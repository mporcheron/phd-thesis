%!TEX root = ../PhDThesis.tex


% *********************************************************************************************************************
% Abstract
% *********************************************************************************************************************

%\renewcommand{\abstractname}{Abstract}
\pdfbookmark[1]{Abstract}{Abstract}
\begingroup
\let\clearpage\relax
\let\cleardoublepage\relax
\let\cleardoublepage\relax

\chapter*{Abstract}
Technologies such as smartphones have had a profound impact on everyday activities, providing us with instant access to a boundless archive of information and real-time communications with others.
Additionally, new devices such as standalone `smartspeaker' devices, which are controlled entirely through voice, are making an appearance in the home.
This instantaneity and pervasiveness of these technologies has not unfolded without critique, however, as many complain of situations where we are surrounded by friends but become distracted, lose track of the conversation, or have trouble balancing demands both from our devices and the conversational involvement.

\iresubmission{Through the completion of three observational studies, this thesis reveals how people bring their device use into a multi-party conversation as part of socialising together.
Furthermore, these interactions are shown to unfold collaboratively, with co-present others supporting device users to complete their interactional projects occasioned through the conversation.}

\iresubmission{This thesis studies three such technologies and examines how interaction with and around the device situationally unfolds:}
\begin{itemize}
\item device interaction where a touchscreen is used as the primary input and output mechanism,
\item device interaction using touchscreen-based devices that also feature voice-controlled interfaces that can be spoken to, and can respond by synthesising speech, and
\item device interaction using voice only, where the device is controlled using spoken `natural language' as the input mechanism, and the device synthesises speech as its response.
\end{itemize}

\crpagebreak\iresubmission{Each study takes place in what is categorically called a `casual-social' setting, with these settings demarcated as places where people gather to socialise and relax.}
The first two of these studies adopt a participant-observer approach, and all three employ an analytic lens based on ethnomethodology.
The first study observes groups of friends socialising in a pub together and reveals how natural device use becomes occasioned in and through interaction and how the device use is interleaved amongst talk in the setting.
Study two adopts a similar premise but asks participants to preferentially use their device's voice-controlled `personal assistant' in situations where they would typically type into a device.
\iresubmission{This reveals how people accountably and interactionally accomplish this practice.
In both of these studies, participants were not asked to use technology or perform any activity other than to socialise together.}
The final study explores the use of smartspeakers in the home through a longitudinal study.
These devices are designed with `far-field' microphones to allow users to speak to them at a distance, and with speakers to allow the device to respond using a synthesised voice.
\iresubmission{Through capturing their use over a one-month period, the study reveals how devices are used as part of the multi-activity home, alongside other ongoing activities.}

\begin{revisedsubmission}
The thesis makes a number of contributions, such as identifying for what purposes and how people use devices in and through socialising together in a casual-social setting, including the collaborative nature of these interactions.
Dealing with technical troubles with devices such as smartspeakers was identified as being a potentially collaborative activity amongst the groups, and synthesised responses from the \acp{VUI} were ostensibly `resources' for dealing with these troubles and progressing interaction with the device.
Through its presentation of thick description and analysis, this thesis establishes the case for further research to examine and design for gatherings in such settings.
\end{revisedsubmission}


\endgroup

\vfill


% *********************************************************************************************************************
