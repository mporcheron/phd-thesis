%!TEX root = ../PhDThesis.tex



% *********************************************************************************************************************
\chapter{Literature review}\label{ch:background litreview}
% *********************************************************************************************************************



\iresubmission[JR-3b, ER-D: Literature review revised to focus more on detailed studies of technology use]{This chapter introduces existing research on the use of technology while we are engaged in face-to-face interactions with others.}
This literature provides the motivational foundation that led to the development of the research questions posed, as outlined previously in \ref{sec:intro rqs}.
In this regard, this chapter will synthesise the literature that frames the \iresubmission{socio-technical and design} backdrop that informs the current understanding of device use in and around conversation and face-to-face interaction. 
\iresubmission{Understanding how technology fits in and around our interactions with each other, which remains ``the most human thing that we do''~\citep[pp. 3]{Turkle2011}, is the key objective of this thesis.
This literature review introduces work in three key areas in relation to technology use, starting from a `big picture' topic, through to work that this thesis is very much situated amongst: (1) studies of technology use in society, (2) studies of systems designed for collocated interaction, and (3) studies of collocated interaction:}

\begin{revisedsubmission}
\begin{description}
    \item[Studies of technology use in society] \hfill \\
    \iresubmission[JR-3b: This section is new to shift the overall focus onto CSCW studies, but is based upon the previous introduction of work on the Socio-technical perspective]{This first field of work, found in \ref{sec:background litreview society}, unpacks the critically reflective literature on widespread device ownership and use in society from a socio-technical perspective.
    This work examines the role that devices play in public and private settings, such as pubs and the home, and crucially focuses on the \textit{impact} of device use, and people's \textit{reflections} of it, helping to establish the backdrop to which the work in this thesis takes place.}

    \item[Systems design for collocated interaction] \hfill \\
    \iresubmission[JR-3b: This section is new to shift the overall focus onto CSCW studies, but is based upon the previous literature review on Mobile HCI]{This second field of work, found in \ref{sec:background litreview design}, will synthesise literature in which systems are designed for supporting and augmenting interaction with others while we are collocated, with aims to support co-operative working.
    This section will briefly introduce `groupware', before discussing how design work in \acf{HCI} and \acf{CSCW} has moved from  meeting rooms to more diverse settings, supporting new portable technologies.}
    % Early development on ``groupware'' systems~\citep{Ellis1991} coincided with the formation of \acf{CSCW}, with \citet{Grudin1994} detailing how multi-disciplinary foci of the domain to examine practices and design technologies for office environments such as meetings rooms.
    % This later moved out of the home to more diverse settings.
    % Additionally, work began to focus on the design of mobile technologies, such as smartphones, and supporting their use in collocated interaction.

    \item[Close studies of collocated interaction] \hfill \\
    \iresubmission[JR-3b: This section is new]{This final field of work, found in \ref{sec:background litreview f2f}, will introduce detailed studies of face-to-face interaction upon which the methodological approach of this thesis draws.
    This literature seeks to examine device use as a matter of course in and through everyday life, by studying users’ interactions in-the-wild, or rather, \textit{in vivo}.}
\end{description}
\end{revisedsubmission}

By briefly unpacking this literature, this thesis' contribution will be situated as part of this broad and interdisciplinary programme of discourse on the social practices of \iresubmission{technology use in casual-social settings}.
Moreover, this contribution is not based on any theoretical or technical considerations of the design of devices, but on the practices of members in a setting.
As such, this work is \textit{indifferent} to models or theories of interaction (see \ref{sec:background approach em indifference} for an expansion of this point on indifference).
\iresubmission{In this regard, this thesis' focus is on people and how they accountably attend to device use naturally: it is \textit{a study of people and their interaction}, and not technology and its uses.}

This thesis studies three main technological developments: touchscreen smartphones, \acfp{VUI} on touchscreen smartphones, and `screenless' `smartspeakers' that only have a \ac{VUI}.
The commentary in this chapter primarily focuses on portable devices such as smartphones and tablets and includes what little nascent literature exists in relation to `smartspeakers' in the home\footnote{This is, in part, because the devices only became commercially available in the last two or so years of this thesis being produced.}.
%However, the existing literature on the socio-technical study of `conversational machines' will be introduced\footnote{This thesis is intentionally devoid of technical descriptions of `conversational machine' design, which can be found elsewhere, e.g. \citet{McTear2016}, given the focus of this work is on interaction with others around technology use.}.



% *********************************************************************************************************************



\section{Technology use in society}\label{sec:background litreview society}
In this section, a socio-technical perspective of technology use in everyday life is developed to frame the problematic motivational backdrop for the approach and empirical work in this thesis.
The utopian view of how technology is briefly introduced, drawing upon \citet{Weiser1991}'s vision of the computer in the 21st century\iresubmission{, before literature that reveals a \textit{societal impact} of technology use is discussed.
This section then progressively unpacks and synthesises literature that examines and offers a critique of device use, and its `impact' and influence on everyday life.}



% *********************************************************************************************************************



\subsection{Ubiquitous computing}\label{sec:background litreview society utopia}
\resubmission{ER-A, ER-B: This section has been reduced and refocused as a primer onubicomp  only, leaving out the discussions of good/bad device use.}Portable devices take many forms, but presently the most prevalent are smartphones and tablets, which have, for the large part, led the charge in realising \citet{Weiser1991}'s prospective vision of device ownership and ``invisible'' use, as outlined in the introduction of this thesis.
In his work, \citet{Weiser1991} set out a vision of (work) environments, where technology of different shapes and sizes is ubiquitously available and always within reach---its use so finely woven into everyday interactions that it ``disappears''.
This realisation of this is ostensibly led by rapid growth in ownership of smartphones and mobile Internet usage~\citep{Poushter2016}\footnote{The growth of smartphone ownership is remarkable with annual growth rates in ownership of 10\% or higher.}.
\iresubmission{More recent innovations, such as voice-activated `smartspeakers' have also seen rapid growth in the last few years, and can now be found in 32\% of US homes~\citep{AdobeInc2018}.}
%\autoref{fig:background technology weiser}, from \citet{Weiser1991}'s seminal paper, should certainly feel familiar to employees in most work environments.
% One particular technology posed in this work was that of ``pads''.
% These devices, as proposed in the work, will be pervasively available as if they were paper, and we would be able to ``spread [them] around on the desk, just as you spread out papers''.
% In some senses, these devices are not too dissimilar to the commercially-available tablets available of today\footnote{the Apple iPad is an example of a tablet that is similar in size to a stack, or pad, of paper.}.

Yet, of course, the vision is utopian and with all prospective utopian visions, should be treated with caution~\citep{Bell2007}. %---indubitably, the notion that we will make use of tablets candidly as if they were paper may seem fanciful.
However, much of what \citet{Weiser1991} projected bares hallmarks of the reality of today, and as noted by \citet[p. 135]{Bell2007}: ``ubiquitous computing is already here; it simply has not taken the form that we originally envisaged and continue to conjure in our visions of tomorrow''.
Simplistically, it is noted that our work environments feature lots of technology, and owing to the ever-increasing capabilities of wireless communication technologies such as Wi-Fi and Bluetooth, we can interact with devices such as tablets in meetings for multiple tasks.
\iresubmission{Of course, outside of the workplace, technology has also become ubiquitous, thanks to the portability and affordability of smartphones and smartspeakers.
The next section introduces the socio-technical work that examines the use of ostensibly ubiquitous computing devices in society.}
%For example, we readily make use of tablets to send documents digitally to others with relative ease or even hand the tablet around for others to look at.% (c.f. \citet{Luff1998a}).

% \begin{figure}
%   \centering
%   \includegraphics[width=.7\textwidth]{Graphics/2-1-Background-Technology/Ubicomp}
%   \caption[Mark \citet{Weiser1991}'s vision of workplaces in the 21st century.]{\citet{Weiser1991}'s vision for technology in the workplace in the 21st century. This photo shows Computer Scientists at the Xerox Palo Alto Research Centre using multiple different devices, including tablets and a large central screen that the audience is gathered around. This sight is probably familiar to most who work in office environments now.}\label{fig:background technology weiser}
% \end{figure}

% This reality is not quite \citet{Weiser1991}'s vision~\citep{Bell2007}, for several nuanced reasons, and moreover, there remains contention around how much of these propositions have been embraced in the design of currently available devices.
% For example, \citet{Haber2014} challenge the notion that current tablets supplant paper in collocated work environments, irrespective of the increased pervasiveness of tablets (this being one of the key suggestions \citet{Weiser1991} made).
% Through observation, they highlight the nuanced differences in collaborating with and around portable tablets and paper, and in turn demonstrate that people may hold a preference for paper due to qualities that cannot be matched with digital devices.
% Elsewhere, \citet{Plank2017} also found that people had a ``legacy bias'', or \textit{preference}, for using single tablets in interaction, suggesting this vision of ubiquitous, easily-accessible, shared technology has yet to, or might not ever, come to fruition.
% Nevertheless, although \citet{Weiser1991}'s vision seems to be hamstrung by practicalities (such as the cost of devices) and nuanced discrepancies between his projection and the reality, the established opinion is that ubiquitous computing has arrived and that \textit{device use is now embedded in our everyday routine}.
% In this vein, much has been made of the embedded nature of different portable devices and mobile communications on everyday social order, treating devices as ubiquitous and it is this literature that the next two sections review.



% *********************************************************************************************************************



\subsection{Togetherness and isolation}\label{sec:background litreview society together}
There is much praise for the use of portable devices such as cellphones and smartphones, especially with the notion that mobile devices provide or \textit{enhance} our daily lives by helping us to shape our experiences of the world around us, although such critique typically comes with caveats.
Consider social anthropologist Sherry Turkle, for example, who is perhaps one of the most commonly referenced cases.
In her oft-cited work ``Alone Together: Why We Expect More from Technology and Less from Each Other'', which offers a critique of the societal use of technology and weakening `desires' to interact with each other, she finds merit in the unique qualities and ``enhanced experience'' provided by the Internet connectivity of modern devices:
\begin{quote}
    [\ldots{}] connectivity offers new possibilities for experimenting with identity, and particularly in adolescence, the sense of free space, what Erik Erikson called the \textit{moratorium} [\ldots{}] [r]eal life does not always provide this kind of space, but the Internet does.
    \quoteauthor{\citet[p. 152]{Turkle2011}}
\end{quote}
Others have found similar effects, such as notions of togetherness and dwelling established through interviewing participants regarding their use of instant messaging platforms, and specifically WhatsApp~\citep{OHara2014}.
Turkle's praise of technology use remains guarded, however, as she then turns to question the negative effects that mobile device use has in everyday life by noting that, in her view, interactions between people are made problematic by the mere presence and use of devices:
\begin{quote}
    [\ldots{}] face-to-face conversations are routinely interrupted by incoming calls and text messages [\ldots] when someone holds a phone, it can be hard to know if you have that person's attention.
    \quoteauthor{\citet[p. 161]{Turkle2011}}
\end{quote}
\label{line:reclaimingconv}She also extends her criticism of devices in later work by claiming that, with the pervasiveness of devices, we lose the sense of wanting to communicate, and in separate work remarks that we must ``reclaim conversation'' as if it were a dying art form~\citep{Turkle2015}.
This criticism rests on notions that we lose the ability to be empathetic because of our use of technology to mediate communication.
While it is easy to disregard Turkle's problematisation of devices as those of a pessimist\footnote{Indeed, many before her have charged other technologies with a similar critique, e.g. consider \citet{McDonagh1950}'s critique that television has transformed conversationalists in the home into mere spectators.}, her views have held stock in work elsewhere, across different disciplines\iresubmission{, and are, of course, relatable to most people\footnote{In other words, for most people, it is relatively easy to recall a situation where someone was using a device while you were talking to them.}.}

Indeed, numerous surveys and interviews identify the increased portability and functionality of mobile devices as encouraging the acceptability of their use in many settings, such as pubs and social environments.
With this finding comes the implications that the use of devices in public spaces becomes derided and charged as annoying or rude by co-present others, and that interruptions from devices and extended mobile search tasks are a distraction from an ongoing conversation~\citep{Ames2013,Church2012,Campbell2007}.
Contradictorily though, \citet{Ames2013} also identifies through their analysis that ``while students often expected others to be constantly connected, they were not always available themselves''~\citep[p. 1494]{Ames2013}.
\citet{Ames2013}, drawing upon \citet{Turkle2011}'s findings, goes on to highlight the difficulty and ongoing contradiction that exists in relation to device use, as people attempt to balance digital and physical social obligations such as `staying connected':
\begin{quote}
    Many expressed concerns about being tethered to ``electronic leashes,'' able to be yanked at any time out of the present [\ldots] [o]thers adopted the values of those around them: when their families or friends derided them for not being fully present or even just having their phone out, they chose to yield to this social pressure. However, most also felt increased anxiety about what their extended network thought about them as a result.
    \quoteauthor{\citet[p. 1494]{Ames2013}}
\end{quote}
In another case, \citet{Humphreys2013}, also through interviews, highlight findings that suggest that the ease of using the `mobile Internet' potentially exacerbates the problem of ``mis-prioritizing communication through their mobile device over and above face-to-face communication''~\citep[p. 501]{Humphreys2013}.
Additionally, in orienting to the use of mobile devices in public places, and in particular pubs, \citet{Su2015} state that technology can ``threaten conversation by creating the present-but-absent, anti-social, and app-addicted patron''~\citep[p. 1667]{Su2015}.
The next section turns to a specific matter of being connected at all times, in which devices `encourage' their use: device notifications, and the research that has examined their occurrence and influence in our daily lives.



% *********************************************************************************************************************



\subsection{Device notifications}\label{sec:background litreview society notifications}
Moreover, there is a substantive body of work investigating how to `better' deliver mobile notifications to individuals in the face of the potentially disruptive nature of such interruptions from portable devices~\citep{Cutrell2000,Fischer2010a,Lopez-Tovar2015}, and now smartwatches~\citep{Cecchinato2017}, drawing on different methods including the observation of groups completing tasks~\citep{Fischer2013} and conducting contextualised interviews~\citep{Hudson2002}.
Certainly, the existence of such work to tackle interruptions from devices raises the prospect that these interruptions themselves, and the use of devices, in general, led to the issues such as `social isolation'~\citep{Turkle2011}.

This literature on device interruptions should, therefore, establish that many people see the use of technology in everyday settings as \textit{problematic} and \textit{disruptive}, in part because their use is brought about by interruptions \textit{arising from} devices.
But, for example, multiple studies have established that although interruptions from devices, such as notifications,  are characterised as problematic, many still prefer to receive notifications than not, for myriad reasons, \iresubmission{including ``awareness''~\citep{Iqbal2010} and to avoid the feeling of ``being cut off''~\citep[p. 560]{Mark2012}}.
\citet{Pielot2015a}, by asking participants to disable notifications on their devices for 24 hours and interviewing them afterwards, identified that ``many participants were anxious to miss information from significant others and superiors''~\citep[p. 1765]{Pielot2015a}.

In reading this literature, it becomes clear that the accepted view is that there are also positive perspectives in relation to the portability and flexibility of mobile devices.
This advantage allows for their greater use within many different settings, including conversation, and allows such devices to provide a utility in which people can remain in touch with their extended network.
This raises expectations that we should quickly respond to our contacts from our friendship groups, just as we expect them to respond~\citep{Ames2013}.
This immediacy provides users with the sense of being ``always connected, to be accessible at all times and places''~\citep[p. 240]{Peters2005} and removing the ``binding between a fixed space and a person’s information and communication resources''~\citep[p. 324]{Perry2001}.
It seems that people find device use problematic because of the burden of managing constant availability~\citep{Sadler2007}, yet find it indispensable and the notion of `untethering' as undesirable because of this very same quality.

Even regarding `non-notification-instigated' use, devices are identified as potentially being a \textit{beneficiary} to everyday life in the home.
For example, \citet{Lanigan2009}, in a study of the use of technology in family life, had what they even considered a ``surprising finding''.
The research found that ``that the more time families spent engaged with the computer, the higher their level of communication, cohesion, and adaptability'', with a home computer encouraging ``more frank communication'', an increase in ``family time [\ldots] due to efficiencies gained through computer use'', and ``a source of mutual interest'' for the family~\citep[p. 603]{Lanigan2009}.
It seems, then, that devices are likely to remain present in our everyday social interactions.
The next section progresses on to examining literature that unpacks how our experience of space is shaped by this device use.



% *********************************************************************************************************************



\subsection{Device use and space}\label{sec:background litreview society space}
Furthermore, exacerbating this point, the use of mobile devices in public settings has been well documented in literature for a variety of purposes, from how an iPod allows an individual to reshape their experience of time and space~\citep{Bull2005}, to how individuals use new technologies such as cellphones to adapt their social perspective~\citep{Humphreys2005,Oksman2004,Peters2005}, and to the enjoyment and ludic pursuits people explore with devices~\citep{Brown2015a}.
In relation to how we now make use of devices anywhere and everywhere we go, \citet{Geser2006} presents a sociological review of whether the mobile phone is `undermining social order'.
He concludes that, through review of how `time-based scheduling and coordination' has declined because of the ready availability of devices, ``a new, more fluid culture of informal social interaction therefore can emerge''~\citep[pp. 5--6]{Geser2006}.
Furthermore, \citet{Campbell2008} argue in an essay that ``mobile communication around copresent others [\ldots] personalizes the communal experience of being in that space''~\citep[p. 379]{Campbell2008}; which also supports work by others of the practice of using technologies to create private spaces in public places (e.g. \citet{Ames2013,Wei1999}).

This literature, which draws on different approaches, establishes that given the desire, or in some cases, compulsion, to remain connected, there is a need to understand the complex factors around the co-management of both the virtual and physical interactions.
This is, in part, due to the relative ease for individuals to retreat to their phone and ``shield oneself from wider surroundings''~\citep[p. 4]{Geser2006}\iresubmission{, which may unfold as individuals with anxiety use the device to shield themselves from unmanageable situations~\citep{Wei2006}}.
Thus, the fact that mobile devices are always connected, and that devices can provide notifications at any point, a situation may become engendered where virtual interactions can potentially rub up against collocated physical interactions.

% These factors all bring to the fore the nuanced and complicated nature of characterising device use as troublesome, as device use, in all forms, is found to provide upsides as well as downsides.
% For example, smartphones have many obvious benefits, such as allowing people to remain in touch with loved ones while away and for calling emergency services whenever they are needed.
% Perhaps not so obvious, there are also situations where devices are used by people with medical conditions for various purposes.
% Furthermore, they allow those who are anxious to shield themselves from unmanageable situations~\citep{Wei2006}.
% Devices are also used as fitness and health trackers, and with connected smartwatches as tools to detect and monitor health conditions such as hypotension and sleep quality~\citep{Phan2015,Iakovakis2016}.
% This review must not be confused as a critique of such research, but instead demonstrates that many of the arguments surrounding device use are well-rehearsed, revised, and challenged within literature.

However, many of these arguments are derived through sociological critique or interviews, opening a gap in the literature for an examination of device interactions from an observational perspective.
Such a perspective avoids being caught up in narratives of perception and feeling, instead orienting researchers to the observable matters of device use, to identify specifically what is \textit{practically done} when people use a device in conversation.



% *********************************************************************************************************************



% \subsection{User experiences of using devices}\label{sec:background litreview society experience}
% %Given the difficulty in capturing `everyday use', many studies default to unpacking the intricate details of device use through interviews and/or diary studies.
% % Both approaches rely upon self-documentation and recall.%\footnote{Anecdotally, my experience of diary studies is that most people complete the diary some time after the event they are recording.}.
% % The strength of such approaches is that through careful interviewing recall is imbued with critical self-reflection and information, and insight into user experiences of device use can be captured over periods of time that would otherwise be difficult to observe~\citep[p. 273]{Wilson2015}.
% % However, such approaches cannot reveal the practical details of how interaction unfolded from an intersubjective perspective~\citep{Berger1966}\footnote{Intersubjectivity is a notion that sidesteps the objective-subjective dichotomy of research, and features heavily in ethnomethodological perspectives because these studies observe practice that is observable and reportable --- such observations are neither objective fact or subjective interpretation (it is not a matter of opinion that someone does something). In other words, intersubjectivity is the acknowledgement that what is done is seen and understood by those who have the competence to see and understand it.} of naturally accountable actions within the setting.
% % Furthermore, many of these studies orient to such devices as \textit{novel devices}, and not devices that are embedded within the lifeblood of mundanity.
% In addition to understanding the reflections of device use, others have turned to matters of the  user experience of using the devices.
% Given the difficulty in capturing `everyday use', many studies default to unpacking the intricate details of device use through interviews and/or diary studies.
% One of the first studies to examine smartphone use identified the `mess' of the multi-functionality of these devices, and of how they differ from \citet{Weiser1991}'s vision by remaining `personal', unpacking issues of seamfulness~\citep{Barkhuus2011}.
% This work, as with others, was completed through interviews and diary studies.
% \citet{Su2015} also relied on interviews for producing their findings\footnote{Although the study also includes observations, the findings were constructed out of interview data.} of mobile device use in Irish pubs, who identify how the presence of mobile devices transform pubs into ``surveilled places'' where individuals could no longer assume privacy.
% Both of these studies find benefits and downsides of existing smartphones, as discussed above in \ref{sec:background litreview society together}, but what they lack is concrete descriptions of how such interactions unfold, and instead default to scenic descriptions of activities that unfold, e.g. ``[s]everal users had ``weird fact'' apps that generate random trivia they would read out loud to friends—essentially, a conversation generator.''~\citep{Su2015}.
% While such findings furnish researchers with an understanding of the sorts of activities that unfold, they are limited in their usefulness in providing valuable and actionable implications for design.

% \citet{Luger2016}, in what was one of the first studies of `modern' commercially-available \acfp{VUI}, interviewed participants about their experiences of such interfaces.
% Manufacturers pitch these devices as being able to handle a complex range of tasks, driven by input through voice.
% This study, as above, used interviews for critical reflection to find shortcomings with existing voice interfaces on smartphones, and that the features of the devices were promoted in such a way that a ``gulf'' formed between expected functionality and that of reality.
% \citet{Luger2016} were remarkably succinct in generating a wide range of actionable implications for design, but because of the approach still provided little in the way of empirical description of the methods of use done by participants, instead relying upon commentaries by participants for revealing the technical troubles that occur.
% Although their findings are useful as a mechanism for honing in on potential problems with interfaces, they do not provide fine-grained detail used to reveal the social organisation of such device use in a casual-social setting, or while collocated with others.



% *********************************************************************************************************************



\subsection{Summary}\label{sec:background litreview society summary}
This brief introduction should highlight and synthesise the examinations of technology use, especially from the perspective of the `impact' of the use of technology in everyday life.
Crucially, this work draws upon reflections, interviews, and sentiments about how technology is used.
\iresubmission{While this thesis adamantly does not dismiss the validity of such findings---indeed they tell us a lot about the world we inhabit---much of it does not examine how device use \textit{practically} unfolds, as \textit{situated action}~\citep{Suchman1985}, and in doing so loses its ``utility in design'' that ethnography, and specifically ethnography informed by ethnomethodology, proffers~\citep[pp. 879--880]{Crabtree2009}.}

In the context of this thesis, then, it is pertinent to consider that as individuals gather to socialise, device use can impact an individual's orientation to space and other co-inhabitants.
This thesis does not adopt a moral or sociological critique of this use but draws upon this multi-faceted argument to consider the rudimentary notion of how such interactions practically unfold (i.e. \textit{what do people actually do with their devices?}). %\footnote{This was a question posed to me after presenting the general premise of this research. I stood at the front of the room shell-shocked by this question, and answering it took three and a bit years.}).
As opposed to attempting to gloss the use of mobile devices to generalisable problems or benefits, or relying upon reflections and interviews to guide understanding, this work is interested in explicating the unfolding and nuanced nature of this practice, as it happens \textit{in vivo}, and how it can be used for design.
This thesis will add to this existing debate by addressing a gap in literature through the provision of \textit{empirical data} that reveals the interactional work people do to use devices in and through interaction.
The next section introduces academic and design efforts to build technologies that support existing collaborative efforts as well as supporting new collaborations using portable technologies.



% *********************************************************************************************************************



\section{Systems design for collocated interaction}\label{sec:background litreview design}
\iresubmission[ER-1 JR-1, IR-2: Include additional literature from CSCW]{This second tranche of literature introduces the long-standing focus of designing systems that allow multiple co-present users to work together under the label of `groupware', originally primarily addressed in \ac{CSCW}.
More recently this work has moved out of the office settings and become concerned with other public and private settings, and examining both sedentary and \textit{mobile} device use.}
With this, the work now spans the overlapping disciplines of \ac{HCI} and \ac{CSCW} (as does this thesis).
\ac{HCI} is perhaps best described as an ``eclectic interdiscipline'' that was initially primarily concerned with notions of the \textit{user experience} of technology, although now encompasses ``all aspect of human life, from birth to bereavement, through all manner of computing, from device ecologies to nano-technology''~\citep[p. vii]{Rogers2012}.
%It is a result of this that \ac{HCI} research is itself interdisciplinary, and draws many different well-established disciplines including psychology, sociology, design, \acf{E/HF}, and so on.

Research in this area has many interchangeable labels, such as collocated~\citep{Lucero2013}, co-located~\citep{Jarusriboonchai2014}, or co-present~\citep{Cole2003} interaction, and same-time same-place~\citep{Fischer2016} research, although here the first of this list is selected for consistency with publications supported by the work in this thesis.
Essentially, in each case, the premise is identical in that work examines the use of technology in situations where there are two or more people present.



% *********************************************************************************************************************



\subsection{Groupware}\label{sec:background litreview design groupware}
\begin{revisedsubmission}[ER-1 JR-1, IR-2: Include additional literature from CSCW]
Early work in \ac{CSCW} examined how to develop groupware systems~\citep{Ellis1991} and other systems to support multiple co-present users interacting with technology and working together, often also referred to as \textit{collaborative software}.
Primarily, technologies were designed for workplace settings such as meeting rooms that were already sites for collaborative action.
Groupware as a term covers a range of software---from supporting collocated interaction and meeting facilitation, e.g. slideware~\citep{Chattopadhyay2018}, through to technologies to support distributed working, such as email-based technologies.
The motivation behind the development of groupware systems was not just the increasing availability of technology during the 80s and 90s, but also the potential benefits of augmenting existing practices with technology.
For example, systems designed to support decision making were found to \textit{increase} decision quality and equality of participation~\citep{Olson1993}.

Progressively, the settings which collaborative systems were designed for moved ``out of the meeting room''~\citep{Bergqvist1999} to classrooms~\citep{Abowd2010}, museums~\citep{Ciolfi2003}, public spaces~\citep{Reeves2011}, air traffic control~\citep{Hurter2012}, and the home~\citep{Edwards2001,Crabtree2016}.
In addition to new spaces, new technologies became the focus of new interrelated fields, such as tabletop interactions and interactive surfaces~\citep{Gjerlufsen2011,Jones2012} through to mobile devices~\citep{Bellotti1996},  such as smartphones and tablets~\citep{Lucero2012}.
It is the latter of this list that the next section focuses on---this field of work, sometimes referred to as \textit{mobile collocated interactions} or part of the field of \textit{Mobile \ac{HCI}}, is occupied with designing technologies for interactions with personal devices while we are collocated with each other, i.e. the situation for which this thesis seeks to understand the social organisation.
\end{revisedsubmission}



% *********************************************************************************************************************



\subsection{Mobile collocated interactions}\label{sec:background litreview design mobilehci}
This section introduces work from the domain of `Mobile \ac{HCI}'.
There is a duality to the definition of mobile in the sense of `Mobile \ac{HCI}', in that it refers to interaction and physical mobility; or the design and use of mobile devices and mobile device applications, i.e. \textit{portable} devices but in potentially sedentary settings~\citep{Church2011}\footnote{Other work also adopts differing definitions of \textit{mobility}, for example, \citet{Luff1998a} use the term in relation to the micro-mobility of artefacts in face-to-face interaction.}.
The work in this section focuses on this latter definition, considering literature that explores interaction in \textit{collocated settings}, i.e. when multiple people are physically collocated together in the same setting, and often sedentary, but use portable technologies such as smartphones and tablets.



% *********************************************************************************************************************



Mobile \ac{HCI} literature is replete with use cases of collocated mobile device interactions, spun out of academic-led design work, such as photo sharing~\citep{Counts2004,Durrant2011}, video watching~\citep{OHara2007}, and collaborative searching tasks \citep{Church2012,Cole2003,Brown2015}, and often involving interaction with additional screens or multiple mobile devices~\citep{Bergstrom-lehtovirta2013,Lucero2013}.
This literature implicitly attempts to demonstrate the beneficial uses of technology in collocated interactions, which expediently refutes---or at least qualifies---simplistic popular views that mobile devices create `social isolation'~\citep{Turkle2011}.
The growing body of literature is exemplified through the generation of design frameworks for the curation of collaborative collocated experiences with technology (e.g. \citet{Lundgren2015}).
In this, mobile devices are examined as artefacts that can be brought into everyday cooperative interactions, with the work of designers transforming their ``features and functionalities [\ldots] into resources for action''~\citep[p. 1117]{Salovaara2007}.

\label{line:singleuser}Much of the work within collocated interactions literature, as a subset of existing Mobile \ac{HCI} research, has challenged the \textit{single-user} nature of \textit{personal} devices, to provocatively explore how the use of mobile devices could instead be designed as \textit{shared} devices that support \textit{multi-user} interaction. This is best elaborated in an \textit{interactions} article on \textit{`Mobile Collocated Interactions'}:
\begin{quote}
    When using their mobile phones, people have a tendency to hold their devices with one or two hands, with the screen facing toward them. People will usually adopt a particular device position, combined even with a second hand to cover the screen, either to browse private content, such as a confidential email, or to avoid glare [\ldots] For people to fully benefit from mobile collocated interactions, they must open up and start seeing their personal devices as shared, public devices. In mobile collocated interactions, phones are at the intersection of fully personal and fully shared use.
    \quoteauthor{\citet[p. 28]{Lucero2013}}
\end{quote}
Many examples of this work, which seems to hark back to elements of \citet{Weiser1991}'s vision\footnote{
Indeed, the notion of devices switching from `personal' to `shared' use is an embodiment of this vision.} (see \ref{sec:background litreview society utopia}), operate by adopting existing \textit{implications for design} from previous studies to guide the design work of a new prototype.
The prototype developed within mobile collocated interactions research is typically tested with `real-world' users in a field trial (e.g. \citet{Lucero2012}), and the resulting analysis used to generate new implications for design.
Some work also specifically follows routes of creating provocative prototypes, not to solve users' problems, but to find ways of evoking critical reflection by users~\citep{Redstrom2006}.
For example, \citet{Lundgren2013} explore ideas to ``design interventions that investigate how apps for mobile devices can make people interact directly in co-located space instead of enclosing themselves with their own digital device''~\citep[p. 1]{Lundgren2013}, primarily using the idea of games to prompt users to reflect and consider interacting with collocated others.

In the same spirit of creating enjoyable interfaces. but with the idea of using the device as a ``resource'' for conversation, \citet[p. 232]{Porcheron2016b} explored the idea of allowing individuals to collect and share photos and notes for a design project with their mobile phone.
The app allowed users to collocate and share the collections but fundamentally required the group to converse face-to-face with each other to use the application successfully and navigate its features.
This work, along with others such as \citet{Lucero2012} that allowed for large display and cross-device interactions in public settings from mobile phones, provides a semblance of supporting the notion that device interactions can be curated, can be enjoyable~\citep{Brown2015a}, and can be used to enhance people's experiences of space (e.g. as in \citet{Bull2005}) and conversation with each other (e.g. \citet{Lundgren2013}).



% *********************************************************************************************************************



\subsection{Designing the context-aware device}\label{sec:background litreview design mobilehci context}
Related to this, the notion of designing for space and the context within which the device is used is explored elsewhere, with a considerable body of work that looks to make device interactions more sensitive to their environment and context of the world around the device.
For many years, a holy grail of Mobile \ac{HCI} research was the ``context-aware''~\citep{Schilit1994} smartphone, with many designs explored in literature; one such influential example is `ContextPhone' by \citet{Raento2005} although there are many more (e.g. \citet{Siewiorek2003,Gellersen2002}).

The purpose of this research is to establish ways of making devices sensitive to the environment, with \citet{Gellersen2002} drawing on \citet{Weiser1991}'s vision to note that ``[i]n the mobile device user interface, context can be used to facilitate a shift from explicit user-driven to implicit context-driven interaction''~\citep[p. 341]{Gellersen2002}.
In the frame of the `contextually aware device', the view of what \textit{context} is differs perhaps from the sociological or interactional definitions, and is broadly classed as:
\begin{quote}
    [\ldots] any information that can be used to characterize the situation of an entity. An entity is a person, place or object that is considered relevant to the interaction between a user and an application, including the user and applications themselves, and by extension, the environment the user and applications are embedded in.
    \quoteauthor{\citet[p. 217]{Dey2009}}
\end{quote}
The ambitions to create devices that are sensitive to anything that is `relevant to the interaction' have generated numerous design and research challenges.
Many of these ideas draw on the notion that context can be used to ameliorate problematic interactions with devices, and are spurred on by the critique that devices are invasive in everyday life (see \ref{sec:background litreview society}).
One of the practical ways in which notions of ``contextual awareness'' have been realised in research is through notification sensitivity to the environment, i.e. a device will determine when to deliver a notification at the ``opportune moment''~\citep[pp. 57--64]{Fischer2011}.
Work in search of the contextual device has progressed, and has considered and explored design ideas for reducing the impact of notifications on many different portable devices such as smartphones (e.g. \citet{Okoshi2016}), smartglasses (e.g. \citet{Kern2003,Lucero2014}), and smartwatches (e.g. \citet{Lee2016a}) by withholding notifications until a later time.

%Progress in this space is swift and is driven forward by the rapid growth in mobile computing power and sensing capabilities.
\iresubmission{Furthermore, ideas also surround the use of the `continuous speech stream' in design to detect the context.
In other words, devices which listen to the stream of speech around them would make use of a continuous and live transcription of conversation to prepare or sensitise interactions to the context~\citep{McMillan2015}.}
In a prototypical trial of this, \citet{Schulze2016} found their system to be on-the-whole suitable, but then attach caveats that ``if conversational context is employed to determine interruptibility, the prior characterization of the conversation is essential'' and ``concerns of participants that go beyond our measures to preserve privacy and beyond a lack of their being informed about those measures, can’t be addressed by design''~\citep[p. 9]{Schulze2016}, exemplifying that the challenges of the context-aware smartphone are socio-legal as well as technical.

However, in spite of this progress addressing the challenge of devices that tailor their interactions to our setting, manufacturers have been slow to adopt such features.
Some smartphones feature options that enable a silent mode when they are placed face down; however, typically devices defer to the user to configure settings for notification management.
The result of this is that the decision of when to deal with interruptions from a device is left to the user.



% *********************************************************************************************************************



\crpagebreak\subsection{Summary}\label{sec:background litreview design mobilehci summary}
\iresubmission{This section has established the tradition of designing systems to support co-operative working within \ac{CSCW} and (Mobile) \ac{HCI}.
Progressively, the focus of development expanded from meeting rooms through to other settings and technologies, including mobile settings \textit{and} mobile devices.}
The literature on creating collaborative experiences with mobile devices while we are collocated with others was summarised.
Then literature was introduced that examines the parallel efforts to ameliorate the problematised nature of device-triggered interruptions to create mobile interfaces that are sensitive to `contextual' information.
Although notifications play a big part of `life with devices', it is expected that they form only part of the occasioning of devices in social settings.
This is reinforced by~\citet{Sohn2008} and \citet{Church2008}, whom both used diary studies to reveal other many reasons for the use of the Internet on a portable device beyond mere device-instigated notifications.
The next section will pivot from matters of designing technologies to studies of how mobile devices are used in everyday life, from a perspective focused on close studies of face-to-face interaction.



% *********************************************************************************************************************



\section{Close studies of collocated interaction}\label{sec:background litreview f2f}
\begin{revisedsubmission}[ER-1, JR-1, IR-2: Include related CSCW literature]
This final section will synthesise the literature that examines interaction amongst people while they are face-to-face, especially from the \ac{HCI} and \ac{CSCW} domains.
These are typically `close' studies of interaction, i.e. they take an approach that orients to observable matters of technology use to address specifically what is practically done when people use of technology `in the wild', with empirical data being the commodity that establishes the findings of the work.
The first section will discuss the \textit{turn to the social} in \ac{HCI}.



% *********************************************************************************************************************



\subsection{\textit{Turn to the social} in HCI}\label{sec:background litreview f2f turn-to-the-social}
\citet{Suchman1985}'s work represents some of the earliest and oft-cited work in \ac{HCI} on the study of the social organisation of interaction with technology in this vein.
Her work, undertaken in a lab-based setting, examined the use of agent-based photocopiers, adopting ethnographic approaches and drawing on the ethnomethodological perspective (this is unpacked later in \autoref{ch:background approach}).
This work was undertaken in parallel to other key works in \ac{CSCW} studies around the same period which examined settings such as air-traffic control rooms~\citep{Bentley1992}, drawing upon similar perspectives, to unpack the cooperative working practices in order to support design activities~\citep{Bannon1993}.
The domains of \ac{CSCW} and \ac{HCI} were increasingly focused on studies of control rooms; in a range of settings from \citet{Heath1992} in a London Underground control room, \citet{Goodwin1996} in air traffic control rooms, and \citet{Watts1996} in NASA mission control rooms.
\citet{Suchman1997} characterised such settings as ``centers of coordination'', in which ``participants' ongoing orientation to problems of space and time'' is crucial in attending to matters of deployment of people and equipment in response to a planned timetable or emergent requirement~\citep[pp. 41--43]{Suchman1997}.

As discussed above in relation to design efforts shifting out of meeting rooms, ethnographic studies too shifted ``out of the control room'' and other constrained settings~\citep{Hughes1994}, as part of a ``turn to the social'' in \ac{HCI} and \ac{CSCW}.
This turn was ``a primary point of view for analysing the design space under the auspices of groupware and cooperative systems''~\citep[p. 28]{Crabtree2003a}.
In this regard, the practice of studying interaction in constrained settings became re-purposed for examining matters of ``everyday life''.
The next section introduces literature that details studies of face-to-face interaction and technology use.
Suchman's work, and that of the Lancaster \ac{CSCW} studies, transformed the ethnographic practices, and the use of the ethnomethodological perspective, into utilities for design in \ac{HCI} and \ac{CSCW}.
This transformation of studies for design did so to support adoption of an ``analytic orientation to fieldwork, which seeks to uncover the locally organized character of action and interaction [\ldots] [that is] is essential to the ongoing development of computing systems that resonate with, support, and enhance what people actually do in new design contexts and how they organize what they do''~\citep[p. 881]{Crabtree2009}.
The next section will synthesise and discuss the literature from \ac{HCI} and related disciplines on the use of technology while we are collocated with others.
\end{revisedsubmission}



% *********************************************************************************************************************



\subsection{Using technology while collocated}\label{sec:background litreview f2f tech-while-collocated}
Early observational studies of portable device use\footnote{Note that these studies were conducted in the early 2000s before the widespread commercial availability of smartphones, and thus they are studies of what may now be considered `traditional' mobile phones, or cellphones, or non-smartphones.} examined aspects such as how devices were shared amongst groups of friends.
For example, \citet{Weilenmann2002} discuss a study of young people in Sweden making use of phones, with observations collected anonymously in public spaces focusing on how sharing is done between collocated friends.
Elsewhere, \citet{Murtagh2002} describes the grossly observable features of mobile phone use during observations of people on train carriages. %, while \citet{Edwards2001} examine aspects of ubiquitous technologies in the home and what it is to `live with them'.
%Both of these studies draw upon an ethnomethodological perspective, as adopted with this thesis (explained later \autoref{ch:background approach}).
\citet{Krehl2013} followed travellers on public transport journeys and categorised their mobile device use.
This categorisation and resulting model was based on contextual factors relating to the use of devices---such as location, task, and technical details.
Such findings provide a comprehensive insight into the activities of device users while mobile, again, providing rich, actionable foundations for work to design a contextually sensitive device (see \ref{sec:background litreview design mobilehci context}).
Conversely, through the collation of existing literature, \citet{Nakamura2015} presents a model of the actions people employ in looking at mobile phone displays in everyday life as non-verbal communication, bringing to the fore some nuanced and complex issues around mobile device use in public life.
This work positions the ``phone user behind something akin to the `fourth wall'''~\citep[p. 74]{Nakamura2015}, which while beneficial for making sense of generating generalisable models of interaction, also does the work of detracting from the situated and contextually-shaped (and contextually-shaping) nature that interaction with a mobile phone occasions and is occasioned by (discussed later in \ref{sec:background approach em sequentiality}). 
%However, these pieces of work do not examine device use moment-by-moment, inste by orienting to the sequentiality of action (see \ref{sec:background approach em}) and shows that, far from being enacted outside of the social sphere, is directly embedded in and through the activities of the users.
%\citet{Brown2014}, in a separate analysis, go on to quantify that 25\% of their captured video ``involved a conversation that took place around the activities on the smartphone'', rubbing up against the notions that ``human relationships are still in decline'' because of the presence and use of devices~\citep{Slade2012}.

\iresubmission{There is also a growing body of work that reveals the intricate ways in which device use becomes interleaved in a variety of settings, adopting methodological approaches that examine interaction moment-by-moment, ranging from kitchens~\citep{DiDomenico2013} and living rooms~\citep{Rooksby2015} through to collaborative photo-taking activities~\citep{Fischer2013}. 
On people using mobile devices while watching television, \citet{Rooksby2015} remark that ``[w]e should not view the mobile device as being brought into television viewing, but the use of mobile devices, the watching of television, and so on as things being brought into leisure''~\citep[p. 17]{Rooksby2015}. 
The sense here is that the use of mobile devices is part of leisure activities in the home, rather than an isolatable activity.
In another study that recorded and studied participants use of smartphones over an extended period, the authors identified how 25\% of all portable device interactions took place while participants were co-present with others~\citep{Brown2014}, illuminating the notion of how the use of devices routinely occurs while we are around others.}
Through the use of screen-recording and fitting participants with portable ``camera bags'', \citet{Brown2013} provide  empirical and rich insight into the daily use of smartphone users.
The work also goes on to introduce sequential practices of how devices are occasioned through activities such as route finding and how web searches are initiated: by others in talk, by events taking place, and by features of the local environment.
The research also brings to the fore how people make use of multi-modal features (e.g. gaze, orientation) to use the device in conversation with others successfully.
\citet{Pizza2016} follow a similar modus operandi to capture the use of smartwatches, revealing what smartwatches are used for and that people embed the interaction with the smartwatch in conversation, which they demonstrate through detailing ``users’ everyday activity [\ldots] [in one situation] the participant is talking about last night, arranging some ingredients for cooking, and quickly reading a notification on the watch''~\citep[p. 5464]{Pizza2016}.
This attention to the minutiae of interaction provides a detailed and rich insight into the interactional accomplishments of people, as they embed device use in other ongoing activities.

\begin{revisedsubmission}
Furthermore, as \citet{Isaacs2012} remark, ``people attempt to blend their local and remote worlds into
coherent interaction when sharing content and one experiences with friends through their devices [and that these] behaviors are becoming more varied, and possibly more common, because of the prevalence of ubiquitous devices and bite-sized content''~\citep[p. 625]{Isaacs2012}.
Work in \ac{HCI} and \ac{CSCW} establishes the case that device use becomes embedded within and is treated as part of activities in everyday life, from television watching~\citep{Rooksby2015} through to searching the web, as occasioned in and through conversation with co-present others~\citep{Brown2015}.
These practices include people embedding the device in interaction through collaborating on web searches~\citep{Brown2015} or making the screen visible to others during use~\citep{Raclaw2016}.

Overall, these studies took place by observing people `in the wild' naturally (i.e. by orienting to
the naturally accountable methods in which members socially organise their interaction), with the studies taking place in perspicuous settings~\citep[pp. 181--182]{Garfinkel2002} to the technology being considered.
This is not to say that such settings could not be constructed as part of the research, for example as laboratory settings~\citep{Rooksby2013}\label{line:labsettings}, and it is the defence of such studies by \citet{Rooksby2013} that is crucial to this thesis: ``[i]f there is to be a ``turn to the wild'' in HCI, this should not be a turn away from the laboratory but a turn away from research methods that ignore human practice''~\citep[p. 1]{Rooksby2013}.
In other words, the \textit{turn to the social} is not, per s\'{e}, a preference for conducting studies outside of laboratory settings, but ones in which the social organisation of the setting and the members within that is crucial to design~\citep{Crabtree2009}, is the locus of the researcher's concern.
Any study that seeks to examine interaction and technology use for design must attend to the matters of the social organisation of the setting as such details ``[\ldots] matter and not as analytic phenomena'' but as matters which the members of the setting understand and must be designed for~\citep[p. 137]{Crabtree2012}.
This thesis takes a hybrid approach, insomuch that the participants are recruited for research and asked to attend a social setting, but the setting is one which they would typically visit together for a research study, and is of perspicuity for the technology interaction of concern.
Participants were made aware that interaction around device use was being studied in each case, with the second study asking participants to preferentially use the \ac{VUI} on their smartphone instead of the touchscreen and the third study including the provision of the technology as part of a deployment to participants' homes.
The focus of the work in this thesis is \textit{the naturally accountable ways in which members attend to device use as it unfolds }(i.e. to reveal \textit{how} it unfolds), and not to establish \textit{why} device use unfolded.
\end{revisedsubmission}

Indeed, most studies of voice-based interfaces have, partly for technical reasons, consisted of Wizard of Oz studies in which the `computer' or \acf{VUI} has been driven covertly by a human rather than a computer agent, in effect introducing a simulated element to the interaction\footnote{This is not to say that such an approach would impede a study of attending to the social organisation of device use; indeed, it is argued as much on page \pageref{line:labsettings}. This is merely to remark that there are nascent studies of commercially available \acp{VUI}.}.
Such practices are long-standing with \ac{VUI}-design related work and studies, with early research on simulating \acp{VUI} used to demonstrate or provide software to implement such studies~\citep{Klemmer2000} or how to undertake a conversation analytic approach to analysing such interactions~\citep{Fraser1991,Wooffitt1994,Wooffitt1997}.
Specific examples also include studies such as how \acp{VUI} can be sensitively designed for use in safety-critical situations such as during driving~\citep{Martelaro2017}, for use in mixed-reality games~\citep{Dow2005}, and action games~\citep{Hoysniemi2004}.
These authors all show how members make use of devices precisely, alter their voice for the device~\citep{Pelikan2016}, and manage the interaction so that it unfolds when opportune~\citep{Martelaro2017}.
Furthermore, each one adopts the perspective of analysing situated action, to reveal details of how users attend to the devices in a particular context, eschewing notions of a generalisable model of interaction.
These studies demonstrate the suitability of adopting observational perspectives to studying device use in order to reveal nuanced interactional practices, but also demonstrate the relevance of studying device use in a particular context---it is not possible to transgress findings from studies across different contexts.
\iresubmission{Given the recent commercial availability of \acp{VUI} on smartphones and standalone \ac{VUI} devices, the necessity to study interactions with them through the Wizard of Oz technique is mitigated, however, the work in the latter two chapters remains among the first to study the social organisation of the use of actual systems in the wild.}



% *********************************************************************************************************************



\subsection{Summary}\label{sec:background technology vivo summary}
This section has introduced literature that has studied the lived experience of device use \iresubmission{through observational studies.}
This work, in juxtaposition with the design work in mobile collocated interactions (see \ref{sec:background litreview design mobilehci}) that intentionally prototypes designs for precisely these \textit{sorts} of settings, demonstrates a shortcoming in literature.
\iresubmission{This section has brought together literature that has examined collocated interaction while mobile and sedentary, with different devices such as smartphones and smartwatches.
However, at the time the work was conducted, there was no study of how device use unfolds in social settings such as public places like a pub, which bring with them `certain ways of behaving' given the specificity of the setting (discussed later in \ref{sec:empirical pub design setting}).
Furthermore, there remains nascent literature on the use of smartphone-based \acp{VUI} and smartspeakers given the recency with which they were introduced, with most studies to date consisting of Wizard of Oz-based studies.
The empirical work in this thesis will address these gaps.}

\begin{revisedsubmission}
The turn to the social in \ac{HCI} centred around the import of ethnographic practices and the ethnomethodological perspective.
This expansion of ethnographic practices into \ac{HCI}, in turn, allows research to support design efforts by generating significant implications for design, which as \citet{Crabtree2012} remind us, allows designers to ground decisions in `facts'~\citep[p. 137]{Crabtree2012}.
Although it is not the purpose of this thesis to design new technologies, through the documentation of the social organisation of conversation with technology interleaved, a greater understanding of how future technologies could meet the needs of users could be designed.
\end{revisedsubmission}

% Therefore, this thesis unpacks interaction as-it-happens in a collocated casual-social setting, without defaulting to simulated or interview-based studies, and will instead present rich empirical data of interaction as its core finding.
% The studies of touchscreen devices in social settings in this thesis follow on from early \ac{CSCW} work on the micro-mobility of documents~\citep{Luff1998a}.
% This work unpacked how documents on paper are used, shared, and passed around in the process of collaborating together for work-related tasks.
% It is in this spirit that this thesis will show how people use and share mobile device interactions in a similar fashion, to collaborate on activities that are occasioned in and through conversation (see Chapters \ref{ch:empirical pub} and \ref{ch:empirical cafe}).
% The last study continues in the same vein, while studying speech-only interactions (\autoref{ch:empirical home}), but again, uses actual empirical data as the bedrock for the findings, instead of simulating interaction with a computer.

%The interests of this thesis lie in how individuals manage the use of these `always connected' devices in conversation, the observable-and-reportable actions of those in the setting, and just \textit{how} device use is embedded in the social order and enacted in and throughout the ongoing interaction.
%Thus, this thesis is interested in attempting to understand the specific interactions that occur, and how devices become occasioned during a conversation, and how their use is embedded in everyday talk.
% Insights in this space have the potential to help designers with the goal of creating more fluid device interactions for multiple users, allowing others to take into account the uncovered interactional practices in their own future work.
% As such, these studies will build upon literature that empirically examines device use \textit{in vivo}, and more fundamentally, present one of the first studies of \textit{\acfp{VUI}} in-the-wild.



% *********************************************************************************************************************
