%!TEX root = ../PhDThesis.tex



% *********************************************************************************************************************
\chapter{Studying pub talk around smartphone use}\label{ch:empirical pub}
% *********************************************************************************************************************



This chapter presents a study of people using their personal mobile devices while they are socialising together with friends in a local pub.
This work was undertaken to unpack the interactional accomplishment of device use in and through an ongoing face-to-face conversation of three or more people \iresubmission*[JR-2a: Realign the thesis on the topic of use in the given settings]{in a `casual-social' setting}.
The study was naturalistic---participants were recruited as groups of friends with the intent of being accompanied by a researcher and video-recorded during a social gathering in a local pub.
They were not at any point asked to use their mobile device, and the use of their devices that ensued, all of which drew upon the devices' \acf{GUI} touchscreen, was entirely coincidental and arose out of external factors or the unfolding conversation between the friends.

This chapter was previously published and presented at the Comp\-uter-Supported Cooperative Work \& Social Computing conference\footnote{See \citet{Porcheron2016a}.}---a number of changes have been made \iresubmission*[JR-3c: Refocus the analysis on to the interactional accomplishments of members in the setting]{to ensure this chapter addresses the research questions of this thesis}.
%These changes are identified in \appref{app:changes-pub} along with a comment on the change.
%Furthermore, a number of changes have been made to refocus the paper onto the collaborative efforts of members of the setting.



% *********************************************************************************************************************



\section{Introduction}\label{sec:empirical pub intro}
\begin{revisedsubmission}[JR-3a: Adjust the focus of this chapter onto an ethnographic account of pub talk]
Pubs are sites for jovial interaction amongst friends and strangers~\citep{Fox1996}, are described as ``focal stages of sociability''~\citep[p. 25]{Torronen2005}, and provide a casual and social setting in which people can relax with friends while drinking, talking, watching sports, and so on.
The very nature of socialising in a pub imbues an informality to the interaction amongst those present, commonly referred to as patrons.
Through the collection and analysis of ethnographic data of groups of friends socialising together, this chapter will unpack the ways in which the use of mobile devices such as smartphones is accomplished as a mundane feature of conversation in a pub (referred to as pub talk herein).
It is in this sense that pubs are established as `casual-social settings', and from existing literature, are identified as places where mobile device use already naturally occurs.
Therefore, their selection as a site for this ethnographic study will be demonstrated to be a suitable and adequate setting for unpacking how devices are used to address members' problems in interaction.

There have been numerous studies into the use of mobile devices by individuals in situations where the user is collocated with others (as discussed previously in \autoref{ch:background litreview}).
Mainly, these follow in the lines of `reductionist' approaches of \ac{HCI} to identify `causes' and `solutions' to problems, in contrast to the holistic ethnographic approach adopted within this thesis.
These studies often ascribe device use to different factors such as \textit{boredom}~\citep{Pielot2015}, \textit{habit}~\citep{Oulasvirta2011}, or \textit{interruptions} or \textit{notifications} originating from the mobile device~\citep{Park2017}.
\citet{Su2015} present findings from a study based on observations of friends using mobile devices in pubs, through which they remark upon how mobile devices ``alter[\ldots] the prime activity''~\citep[p. 1659]{Su2015} of places such as pubs because of new activities taking place; by this they specifically orient to the use of the devices by pub patrons.
Furthermore, work within academia has explored numerous psychological and emotional factors relating to what leads to device use~\citep{Kushlev2016} and how receptive people are to interruptions in different situations~\citep{Fischer2010, Mehrotra2016}.

The conclusions drawn in the literature that relate the use of mobile devices in a face-to-face conversation ascribe elements of negative impression formation and `interaction quality'~\citep{VandenAbeele2016}, with devices seen as having a profound negative impact upon face-to-face encounters~\citep{Nakamura2015}.
Even more, a perspective of faltering societal development has been used as a rallying cry in the widespread critique of the use of mobile devices during encounters with others~\citep{Turkle2011} to encourage people to ``reclaim conversation'' with each other~\citep{Turkle2015}.
This thesis differs in its orientation to the study of device use by adopting an ethnographic perspective to illuminate how people practically accomplish device use in social settings.
With this approach, this thesis eschews the demarcation of causes and solutions, as well as the contention of the `morality' of the actions of patrons in using a device, and adopts a holistic approach to revealing the interactional accomplishment of how people use devices in a casual-social setting.

%Yet, this thesis adopts an ethnographic approach, orienting to unpacking the nature of how individuals in such settings occasion and use a device in and through conversation.
By studying natural interactions of groups of friends socialising together in a pub, the concept of pub talk with and around smartphones will be explored.
Pub talk is perhaps best described as informal chatter that consists of ``repetition, rhetorical questioning, and apparent irrelevance''~\citep[p. 241]{MassObservation1943}, or in other words, pub talk could be construed as informal and relaxed conversation and could be regarded as the overall \textit{accomplishment} of inhabiting a pub to socialise.
This chapter will seek to explicate the gloss of 'using a mobile device in conversation' and crucially reveal the methodical actions taken to use a mobile device while also engaging in social interaction with co-present others.
\end{revisedsubmission}
%Therefore, this study will seek to explicate the gloss of 'using a mobile device in conversation', to reveal it as a sequential and embedded activity within the broader social activity and context.
%The methods through which members account for and sustain the use of devices in and through a conversation will be drawn out through the analysis.
%This work will be undertaken with respect to the social context within which the practice is embedded, yet the analysis will not seek to orient to or critique any particular `cause' of the device use.
%By intentionally adopting an approach that disregards the technical or psychological causes of a device use, this work will show how members co-manage and collaborate with and through the device use irrespective of the trigger that was identified through the interaction.

%This question is morally ambiguous: device use is not positioned as being beneficial, detracting, or impactful upon another activity; it merely is presented as an accomplishment of individuals in everyday life.



% *********************************************************************************************************************



% \section{Research questions}\label{sec:empirical pub rqs}
% Given that mobile device use is often lambasted as a now common feature of everyday face-to-face gatherings, the question of how this mundane practice unfolds so effortlessly and so frequently becomes an interesting and relevant site of study for ethnographers and designers.
% The purpose of unpacking this interaction is not only to understand more of how individuals and co-present others co-manage such interactions with devices, but also how, fundamentally, the device use is sequentially embedded within  face-to-face encounters.
% The first research question in this research is thus:

% \PrintRQ{1}

% Expanding on this, to understand how mobile device use is embedded within a face-to-face conversation, it becomes necessary to draw out the key sequences of activity (i.e. orienting to the sequentiality of action) conducted by members of using a mobile device in conversation.
% In the following work, the three key stages of this that are oriented to are defined as:

% \begin{itemize}

% \item \textit{Occasioning}: The ways in which mobile device use is occasioned in and through interaction, including the talk and embodied actions that lead up to the use (if observable);

% \item \textit{Sustaining}: How the mobile device use is sustained with respect to both the role of the mobile device use in the conversation and the actions of the members within the broader context;

% \item \textit{Disengaging}: The ways in which the mobile device is disengaged from, either temporarily or (semi-)permanently.
% \end{itemize}

% This work will attend to the observable-reportable actions (i.e. accountable) actions~\citep[p. 1]{Garfinkel1967} of members of the setting, and consider the retrospective-prospective character of interaction at each stage in the sequence.
% This will, in turn, reveal how members within a setting embed the mobile device use, and crucially how this is mobile device use is co-managed by the device user and others within the group.

% As will be revealed, however, although mobile devices are designed as single-user, their use becomes embedded in and through the conversation, and through actions taken by the owner of the mobile device.
% These actions can form collaborative efforts between multiple individuals, forming a situation where multiple members collaboratively engage with the mobile device use:

% \PrintRQ{2}

% By orienting to the retrospective-prospective character of members' actions, i.e. by considering the sequentiality of the how members interactionally accomplish tasks with and through the use of mobile devices, and of the way the mobile device use becomes embedded within the conversation by one or more members, an understanding of how individuals cooperate with and around the contingent mobile device use will be revealed.



% *********************************************************************************************************************



\section{Study design}\label{sec:empirical pub design}
%A study was designed to observe the everyday use of mobile devices in conversation amongst friends while they are socialising in a pub.
%In this chapter, the naturally accountable methods through which individuals make use of mobile devices during focused encounters~\citep{Goffman1968} in a local pub were observed, explicated, and reflected upon.
% \begin{revisedsubmission}[Change EMCA to `study informed by EM'; added rationale of selecting interaction analysis with quote] %JR-2
% To achieve this task, a study informed by ethnomethodology and drawing upon Interaction Analysis~\citep{Jordan1995} was devised.
% This work rests upon the assumption that ``that knowledge and action are fundamentally social in origin, organization, and use, and are situated in particular social and material ecologies''~\citep{Jordan1995}, and that by orienting to the actions of members, and the sequentiality in which they are organised, the practice of how device use is begun, carried out, and ended in the course of conversation can be revealed.
% %To achieve this task, a study and analysis informed by \ac{EMCA} was devised.
% %The work orients to the temporal sequentiality in which device use is begun, carried out, and ended in the course of conversation.
% This common approach to documenting an interaction and the context within which it occurs brings into focus the internal structure of the process through which the work is cooperatively achieved by the members of the setting~\citep{Jordan1995,Heath2010}.
\begin{revisedsubmission}[JR-3a: Change EMCA to `study informed by EM']
This section outlines the design decisions made with respect to the study.
Summarily, a video-supported ethnographic study informed by ethnomethodology~\citep{Garfinkel1967} was undertaken, in which groups of friends socialising together in a naturalistic manner.
To do this, the friends were recruited to go to a pub together to socialise, and be video recorded for the duration of the gathering---groups were not recruited to use devices or guided to complete a given task.
The collected data were reviewed and analysed with an orientation to making sense of the `members' accomplishment' and orientation of device use within the setting.
The videos were viewed, with periods of mobile device activity (`fragments') catalogued for further in-depth review and remarked with notes upon the primary activity or `purpose' of the device use.
These fragments were then viewed and described in a more fine-grained manner---detailing what was done with the device in terms of movement, the visibility of the screen, whether someone mentioned or brought the device or purpose of the device use up in talk, and so on---in order to provide an index into the various members' practices.
This provided a comprehensive oversight of the data and supported the selection of fragments as \textit{vivid exhibits}~\citep[p. 111--112]{Crabtree2012} of how members organised device use within pub talk.
The resulting analysis rests upon the assumption ``that knowledge and action are fundamentally social in origin, organization, and use, and are situated in particular social and material ecologies''~\citep[p. 41]{Jordan1995}, and that by orienting to the actions of members, and the sequentiality in which they are organised, the practice of how device use is begun, carried out, and ended in the course of conversation can be revealed.
 \end{revisedsubmission}



% *********************************************************************************************************************



\subsection{The pub as a study setting}\label{sec:empirical pub design setting}
\resubmission{JR-2b, JR-3a: Changed title to represent text more accurately}A motivating factor for the studies in this thesis is to explore the interactions that have led to the rhetoric around the impacts of technology use in everyday situations, and especially when friends are socialising (see~\autoref{ch:background litreview}) in settings that could be considered both `casual' and 'social'.
This thesis defines such a place as one where individuals purposefully co-inhabit with the purpose of socialising in a relaxed and unimposing environment and where conversation is the main activity.
This stance also does not place restrictions on a venue that is exclusively public or private, or the type of venue.

The definition is perhaps most closely relatable, but not entirely congruent to the notion of ``third places'' proposed by \citet{Oldenburg1989}.
Third places are spaces that are outside the home or workplace, where people can gather, socialise, and relax.
The definition of \textit{casual-social} augments this notion by allowing for places that may be considered homes, or perhaps even social environments in or near to a workplace.
For example, the common-sense experience is that it is routine for groups of friends to meet in public plazas, caf\'es or restaurants, homes, libraries, and comfortable spaces around work environments to `catch up' and socialise with each other.
Each of these places can be a relaxing and social forum supportive of group conversation where the purpose of the gathering in such a place is to support leisure-time.

%This study extends existing work in such settings by orienting to a particular feature of the mundane practice that takes place, specifically that of embedding the use of a mobile device within an existing ongoing face-to-face encounter.
\citet{Laurier2001} demonstrate how settings such as those explored in this thesis still exhibit organisational traits, although they typically lack ``complex articulation and coordination work''~\citep[p. 222]{Laurier2001} found in more formal settings.
However, they still show that caf\'es and ``places of that type''~\citep[p. 199]{Laurier2001} provide a ``common code of conduct''~\citep[p. 210]{Laurier2001} that is informal yet provides guidance of behaviour that is adhered to by members.
This conduct is found to exhibit elements of informality and engenders expectations of how members demonstrably and competently perform the work of socialising together.
However, the work in this chapter will attend to how members' articulation and coordination work still occurs within members' interactions with, around, and through mobile device use and conversation.

In selecting a fieldwork setting, consideration was given to a variety of venues including caf\'es and public squares. 
\iresubmission*[JR-2b: Add references to support case that mobile device use is perspicuous to social gatherings in pubs]{A pub was selected for a number of reasons, some logistical and others sentimental.
\citet{Su2015} remark that with their observational study of mobile phone use in Irish pubs: ``without doubt, the mobile phone is ubiquitous in pubs [\ldots] [participants] overwhelmingly acknowledged that mobiles, used tactfully, was not a breach of etiquette''~\citep[p. 1663]{Su2015}.
Additionally and anecdotally from personal experiences, pub settings would allow for the observation of naturally occurring interactions around mobile device use in an environment in which mobile phone use is common, and sometimes at the derision of co-present others.}

The devotion of spending leisure-time in pubs and bars with friends is a popular British pastime; pubs typically open early and close late, many provide food and drink, and they serve as an environment suited to relaxing and conversing with others.
In describing her observations of English culture, anthropologist and popular social science writer Kate Fox describes pubs as ``a central part of English life''~\citep[pp. 88--108]{Fox2004} and others have also highlighted pubs ``as a social centre for the community''~\citep[p. 693]{Clarke2000}.
These descriptions are also reflected in official statistics which state that 48\% of people aged 16 and over would choose to go to a pub or bar in their free time; this figure is even higher for younger age groups~\citep{Seddon2011}.
\iresubmission*[JR-2b: Conclude the point regarding the perspicuousness of the setting to the activity under study]{Therefore, observing a gathering of friends who are leisure-time socialising in a pub would act as a suitable setting and activity in which mobile device use has previously been identified to unfold.}

\iresubmission{In summary, pubs are a \textit{perspicuous}~\citep[p. 181]{Garfinkel2002} setting in which friends gather to socialise (corresponding with the premise of the study and how data was collected, discussed below in \ref{sec:empirical pub design fieldwork}) and in which mobile device use is \textit{ubiquitous}.
In the words of Garfinkel, a pub makes available the ``material disclosures of practices of local production and natural accountability in technical details \textit{with which to find, examine, elucidate, learn of, show, and teach the organizational object as an in vivo work site}''~\citep[p. 181]{Garfinkel2002}.
In other words, mobile device use is routinely accomplished and regulated in and through pub talk, and by studying the use of devices \textit{in vivo}, the phenomena of inquiry---that is, the use of touchscreen-based mobile devices in conversation---can be explicated.}
Based on up the above literature, statistics, and personal experience, it was natural to conclude that pubs provide a suitable and natural environment for the study of how people embed mobile device use during the conversations in a casual-social setting.



% *********************************************************************************************************************



\subsection{Collecting data in the pub}\label{sec:empirical pub design fieldwork}
After finding a pub that agreed to host the research, participants from the university were recruited using email and word-of-mouth.
Participants were recruited as groups of friends who felt they would ``typically go to the pub with each other'' and were willing to be observed for their ``behaviours around mobile devices'' within a pub.
In total, eleven participants took part (in three separate groups); seven of the participants identified as female, with the remaining four identifying as male.
Each group had at least one female and one male, although this was by chance and not intentional.
Of the recruited participants, four were aged 18--23, five were 24--29, and two were 30--39.
The studies were conducted over a three-month period in the UK, taking place at a time agreed with the recruited participants.
The study was approved by the university's School of Computer Science Research Ethics Committee and participants were reimbursed with an online shopping voucher for their time spent during the study.

The study was \textit{participant-observer} in practice.
%The purpose of the fieldwork was to reveal the interactional methods members employ to accomplish the work of using mobile devices within an ongoing conversation, to answer Research Question 1 (see \ref{sec:empirical pub rqs} above).
To achieve this in a naturalistic way, the only `activity' asked of participants was that they converse as they normally would---no tasks were given to participants (the information sheet that was given to participants prior to the study is in \appref{app:studyinfo-pub infoconsent}).
Video and audio recordings of participants socialising together in groups were collected as part of the approach in an effort to allow for the explication of the interleaved use of mobile devices in the conversation, and to study the observable-reportable actions exhibited by members of the setting through video ethnography.
\iresubmission*[ER-F, ER-G1: Specify how data was collected in practice]{The video data was captured with two fixed wide-angle lens cameras on tripods, and a separate audio recorder on the table for higher-fidelity audio to avoid issues of the noise of the environment `drowning' out the sound of the participants.
This allowed for video data to be collected unobtrusively during conversation without the requirement of camera operators being present.}
Relying upon field notes would have hindered participation, and brought attention to the observation of the conversation while also providing a lower fidelity of data.

Questions were asked after the `observation phase' as an interview (the primary questions are given in~\appref{app:studyinfo-pub interview}) so as not to interrupt the flow of the conversation.
The purpose of the interview was to contextualise the observations and gain an insight into the participant's perceptions of mobile device use in conversation.
%Participants also completed a short questionnaire (found in \appref{app:studyinfo-pub questionnaire}) after the observation phase to gather information on the technology they owned.
Given the evolving landscape of mobile technologies, this acted as a point of curiosity to understand the present situation.

Through the questionnaire, participants were asked about which technology they owned: all participants owned smartphones, and had them present, a majority (seven) also owned tablets (although six of these relied on a Wi-Fi connection); however, none had a tablet with them, and there were no smartwatches.% When asked, only 2 members said they would consider ``using their tablet when socialising with friends''

Overall, the ethnographic record is comprised of video recordings of the interaction, field notes made after the session, individual questionnaires completed by members, and the recording of the informal semi-structured group  interview.



% *********************************************************************************************************************



\subsection{Analysing the collected data}\label{sec:empirical pub design analysis}
\resubmission{ER-F, ER-G1: Add further information about the analysis of the corpus, primarily the selection of fragments}To analyse the corpus of collected video and audio data, video analysis, drawing on ethnomethodology~\citep{Goodwin1990,Heath2010} was performed.
Firstly, shortly following data collection, the corpus of data was catalogued and indexed to identify episodes in which mobile device use occurred.
Timestamps and descriptive language were used to construct a record of the interactions that took place, which allowed for iterative re-examining of prior data with relative ease.
This was in order to aid the discovery of the observable-reportable actions performed by the members of the setting and to help gain an overall impression of the data collected across all the sessions.

In total, 51 episodes of mobile device use in the sessions were identified (some of which were overlapping), with episodes ranging from a few seconds to a few minutes in length.
A substantive review of the episodes was performed to examine the interaction that unfolded, honing in on episodes that represented observable-reportable intersections of mobile device use and conversation for a more in-depth analysis.
\iresubmission*[ER-G1: Add details about fragment selection]{Seven fragments were selected for further analysis where instances of device use lasted for more than `a few seconds' and where this interaction occurred in and through conversation (i.e. device use was interleaved in some grossly observable fashion with conversation).
In each case, these fragments were selected in line with the aims of this research to reveal the social organisation of device use in pub talk and that were deemed to warrant further investigation, with each fragment being reviewed individually and discussed collaboratively with other researchers multiple times.}
These fragments were then transcribed with both verbal (i.e. talk) and non-verbal (e.g. gestures and other interactional resources) being carefully noted.
Situations where, for example, mobile devices were used merely as timepieces for a split-second, were ignored and not used within the corpus.

Following multiple iterative reviews, a collaborative `data session' was performed, with other \ac{HCI} and \ac{CSCW} researchers within the same research group invited to watch, review, and comment on collected video data and analysis, and to provide critical reflection on the findings explicated.
In this session, observations and commentary developed through the analysis were provided by the author along with transcripts of the clips, and all were reviewed in a collaborative and reflective manner.



% *********************************************************************************************************************



\section{Findings}\label{sec:empirical pub findings}
\begin{revisedsubmission}[JR-3a: Introduce the new structure of fragment presentation] %PL-DC-H
Three fragments will now be introduced and presented over a series of `data excerpts'---these fragments are \textit{vivid exhibits}~\citep{Crabtree2012} of the data within the collected corpus.
By this, this thesis considers each of these as exemplars of the activities of members' observed practices in the collected corpus in which members' interactional projects are accomplished in and through the use of devices in a pub.
In summary: the first fragment illustrates typical interactionally unproblematic use of a device to introduce new information to the conversation, the second fragment introduces a more complex case that reveals more features of interaction in which a device is used as a resource to make a joke, and the final fragment introduces an interactionally problematic case in which device use is used to contest an argument~\citep[p. 111]{Heath2010}.
Throughout the explication of the fragments, the practice of how device use is used as a mundane activity in pub talk will be established.

%Clarify how/why these three were selected, e.g., exhibiting some interactional projects accomplished in and through copresent use of devices in a pub. These illustrate a typical/unproblematic case, a more complex one revealing more features, and finally a problematic/troublesome one. See Heath et al., 2010, p. 111.

% The first fragment will focus on an exemplary case where the device use is occasioned through articulation work in order to add new information to the conversation.
% The second and third fragments will then progressively introduce further key aspects of device use in pub talk, while demonstrating how the interaction is organised, will unpack additional aspects of the social organisation through which device use is interleaved in pub talk.

%As introduced while recapitulating the research questions previously (see \ref{sec:empirical pub rqs}), this work orients to the temporal sequentiality of action, and reveals the interactional methods through which mobile device use is occasioned, sustained, and disengaged from within an ongoing face-to-face interaction.
%As introduced while recapitulating the research questions previously (see \ref{sec:empirical pub rqs}), this work orients to the temporal sequentiality of action, and reveals the interactional methods through which mobile device use is occasioned, sustained, and disengaged from within an ongoing face-to-face interaction.
%Crucially, a \textit{machinery of interaction}~\citep{Sacks1984} can be produced from the explicated methods of members to answer RQ1, transforming the notion of 'using a mobile device in a conversation' into a series of methodical interactional accomplishments.


% \PrintRQ{1}

% \begin{revisedsubmission}[RQ2 should be answered within the empirical chapters too.]
% Accordingly, this examination also entails to question, as per the aims of this thesis, whether and how members collaborate in and through the multi-party conversation:

% \PrintRQ{2}
% \end{revisedsubmission}

% Each stage of activity of mobile device use being embedded in the interaction (i.e. the occasioning, sustaining, and disengaging of the mobile device use) takes multiple interactional forms that are context-shaping and context-shaped.
% \begin{revisedsubmission}[Presentation of findings is shifted to be around fragments and not stages of device use]
% In the following sections, a number of episodes of device use, presented as `fragments' of the collected data, will be employed as \textit{vivid exhibits}~\citep[p. 112]{Crabtree2012, Bannon1993} of the work conducted by members to interleave their device use in and through conversation.
% To achieve this, the interactional resources that members employ to accomplish this work are scrutinised and brought into consideration.
% Interactional resources include talk, body movement and orientation, and gestures~\citep{Luff1998}.
% \end{revisedsubmission}
% %In the following sections, each stage is presented along with a number of selected relevant episodes as `fragments' of the collected data that provide the \textit{vivid exhibits}~\citep[p. 112]{Crabtree2012, Bannon1993} of the work conducted by members.
% %In so doing, the interactional resources that members employ to accomplish this work are also scrutinised.
% %Interactional resources include talk, body movement and orientation, and gestures~\citep{Luff1998}, and are intrinsically used in conversation as non-verbal communication.

% Although this work is primarily concerned with situations where mobile device use is used within the routine of conversation, situations when mobile devices provided notification chimes or displays turned on are also reported upon as observable features of the setting.
% Furthermore, numerical figures are given throughout the findings as descriptive indices into the qualitative data corpus, although this is a reference to denote the recurrence of which events occurred and do not represent quantifiable findings.

%This chapter will present a number of fragments that reveal the intricate ways in which device use is introduced and interleaved within the multi-party conversation.
In each fragment, the setting will be introduced, giving a background to the conversation that is unfolding.
%Then, in more detail, each fragment will be discussed in terms of how the device use is occasioned, sustained, and disengaged from in conversation.
By this, this thesis means to demonstrate the ways in which: device use is brought about \textit{in and through} conversation as an interactional accomplishment, how members perform actions throughout the conversation to sustain device use within it (often this involves articulation and accounting practices), and how members stop using the device. 
\end{revisedsubmission}

\appref{app:notation} provides details of the transcription notation used in this thesis.
All names and identifiable information within the transcripts provided are entirely fictional.



% *********************************************************************************************************************



\subsection{Introducing new information for conversation}\label{sec:empirical pub findings newinfo}
\begin{revisedsubmission}[JR-3a, JR-3c: This section is new, reconstructed from prior analysis]
In the first excerpt from the fragment, titled \textit{Miniature Schnauzers}\footnote{The complete fragment is included in \appref{app:fragments-pub schnauzers}.}, given in \autoref{frag:empirical pub findings newinfo-i}, two friends, Cally and Dayna, are discussing their favourite dog breeds.
They are sitting across the table from two other friends who are having a separate discussion.
There is a disagreement between Cally and Dayna, as a matter of personal preference, in relation to their favourite dog breeds.
\end{revisedsubmission}

\newpage
\begin{inlinefrag*} 
    {\fragresubmission{JR-3c, ER-H: Revised fragment that is shorter, with labels on the image for the speakers}
    \begin{transcript*}
        \by CAL {i like miniature schnauzers} \\
        \by DAY {°how big are schn-?°} \\
        \im 2   {Graphics/3-1-Empirical-Pub/FragmentSchnauzer-1.png}
        \by CAL {it's like (.) like (.) they're} \\
        \by     {\emph{so:} cute\vspace{2cm}} \\*
    \end{transcript*}
    \caption{Miniature Schnauzers (i)}\label{frag:empirical pub findings newinfo-i}
    }
\end{inlinefrag*}

\begin{revisedsubmission}
In this opening excerpt, Cally occasions the interactional project of \textit{providing new information to the discussion} in and through the ongoing conversation.
This is to address an information deficit that develops once it is established that Dayna is unaware of the size of Cally's favourite breed.
We first joined the discussion between the pair as Cally establishes her preference for Miniature Schnauzer (line 01) and Dayna seeks clarification on the size of the breed (line 02).
This small snippet provisions the members' occasioning of the forthcoming device use which will be explicated.

Later device use will be occasioned by this established information deficit through the production of two key activities in this fragment.
These will be unpacked separately below:
\begin{enumerate}[label=(\roman*)]
    \item \nameref{sec:empirical pub findings newinfo getting-someone}, and
    \item \nameref{sec:empirical pub findings newinfo collab-finding}.
\end{enumerate}
Each of these actions are unpacked respectively in relation to how they are used to accomplish the occasioned interactional project.
\end{revisedsubmission}



% *********************************************************************************************************************



\subsubsection{Getting someone to look up new information}\label{sec:empirical pub findings newinfo getting-someone}
\begin{revisedsubmission}
The discussion continues below in \autoref{frag:empirical pub findings newinfo-ii} where both Cally and Dayna reflect upon their favourite dog breeds.
Cally previously gestured with her hands for an approximately waist-width-sized dog (line 03, above), and Dayna copies this gesture (line 05, next excerpt) before stating her preference for \QF[07]{big dogs}.
Cally acknowledges this remark with \QF[08]{i know}, and then instructs Dayna to use Google to search for the breed.
Through this instruction, Cally is getting Dayna to use the device in order to provide her with the additional resources to allow her to make sense of Cally’s perspective (i.e. as a request to `try to see it from my perspective').

\begin{fragfloat*}
    {\fragresubmission{JR-3c, ER-H: Revised fragment that is shorter, with labels on the image for the speakers}
    \begin{transcript*}[5]
        \im 1   {Graphics/3-1-Empirical-Pub/FragmentSchnauzer-2.png}
        \by CAL {((briefly looks at her bag to} \\
        \by     {her left before looking back))} \\*
        \by DAY {i like big dogs\vspace{1.9cm}} \\*
        \im 2   {Graphics/3-1-Empirical-Pub/FragmentSchnauzer-3.png}
        \by CAL {i know, but google schnauzer,} \\
        \by     {right?} \\*
        \by DAY {((gets phone out from bag))} \\*
        \by CAL {((leans towards DAY)) } \\
        \by     {the puppies (.) schnauzer} \\
        \by     {puppies are gorgeous} \\
    \end{transcript*}
    \caption{Miniature Schnauzers (ii)}\label{frag:empirical pub findings newinfo-ii}
    }
\end{fragfloat*}

Cally's request is treated as unproblematic in the routine of talk-in-interaction in a pub by the friends.
Whereas it is foreseeable that such a request may be explicitly oriented to by as members \textit{out of place} in other settings, in the pub---and as a constituent activity of pub talk---such a remark is produced and responded to as not out of place.
In other words, Cally’s occasioning of the device use through the request for Dayna to search for information, and Dayna’s acquiescence to the request from Cally, establish that the practice of getting someone to look up new information is mutually regarded as acceptable practice as part of the social organisation of interaction in a casual-social setting such as the pub.
\end{revisedsubmission}



% *********************************************************************************************************************



\crpagebreak\subsubsection{Collaboratively finding new information}\label{sec:empirical pub findings newinfo collab-finding}
\begin{revisedsubmission}
Following this discussion, Cally re-orients her gaze to the others conversing at the table, responding to the continuing discussion amongst the other three people at the table.
Some 12 seconds later she refocuses on the topicalised device use, as exhibited below in \autoref{frag:empirical pub findings newinfo-iii}, by shifting her body posture and gaze to look at the device screen.
In the next excerpt from the data, both Cally and Dayna work together to collaboratively accomplish the task for which the device use was occasioned (introducing new information for conversation).

At the start of this fragment, Dayna tilts her screen marginally towards Cally, and Cally reciprocates by leaning towards Dayna and shifting her eye gaze towards the screen (visible at line 17).
Cally re-prompts Dayna with the search term: \QF[15]{miniature schnauzer}.
Dayna seeks guidance from Cally on the spelling of the terms to type into Google, although she does not complete the utterance asking for this; it is established through Dayna’s phonation of the word \QF[18]{schnauzer} and apparent inability to complete the task.
Cally provides the spelling (line 20), with Dayna enacting Cally’s guidance by typing the dictated letters on the device, established as a Google search through inference of the ongoing interaction.
Following the action of Dayna typing in the search term, as dictated by Cally, Cally proceeds to provide further guidance by instructing Dayna to \QF*[23--24]{go look at schnauzer puppies}.

\begin{fragfloat*}
    {\fragresubmission{JR-3c, ER-H: Revised fragment that is shorter, with labels on the image for the speakers}
    \begin{transcript*}[15]
        \by CAL {so it's miniature schnauzer} \\
        \by DAY {how do you?=} \\
        \im 1   {Graphics/3-1-Empirical-Pub/FragmentCollabSearch-2.png}
        \by CAL {~~~~~~~~~~~~=erm::} \\
        \by DAY {(sccchhhh) (tea) (ee) (ar)} \\
        \by CAL {oh schnauzer (.)} \\
        \by     {it's s-c-h-n-a-u-z- n-a-u-} \\
        \by     {(2.2) schnauzer} \\
        \by DAY {oh, sch\emph{nauz}er\vspace*{.4cm}} \\
        \im 1   {Graphics/3-1-Empirical-Pub/FragmentCollabSearch-4.png}
        \by CAL {schnauzer, go look at} \\
        \by     {schnauzer puppies right\intUp} \\
        \by     {((continues to look at phone))}\\
        \by DAY {°my internet is rubbish so} \\
        \by     {this may take some time°\vspace*{.9cm}} \\
    \end{transcript*}
    \caption{Miniature Schnauzers (iii)}\label{frag:empirical pub findings newinfo-iii}
    }
\end{fragfloat*}

This action is imbued with collaborative efforts from the pair as device use is interleaved within the talk between the two.
Cally \textit{instructs} Dayna to search for Miniature Schnauzer, although remains attentive to the device use---shifting her posture and fixing her eye gaze to the device's screen.
As Dayna seeks clarification on the spelling of the search terms, Cally responds to the clipped request \QF[16]{how do you?} with \QF[16]{erm}, with Dayna clarifying it is the spelling of the term \textit{Schnauzer} that she is seeking assistance with through her phonation of the word (line 18).
Cally responds to this by providing the spelling (lines 20---21).

Cally accountably provides support to---and engages with---the device use by Dayna, as established through her gaze and posture as to be `watching' and participating with the device use.
By participating with Dayna to complete the occasioned activity, Cally’s intention of ensuring the completion of the occasioned task is made visible (i.e. Cally waits for Dayna to do the task requested, and watches her as she goes to complete it).
Following Dayna's completion of the input into the device, Cally provides further instruction to Dayna to examine photos of the puppies: \QF*[23--24]{go look at schnauzer puppies}, however, Dayna rebuffs this remark by stating that her internet is slow and that it \QF[27]{may take some time}.
The act of remarking that her device Internet is slow, and putting down her device, completes this period of collaborative device use, and is followed by both members re-joining the existing conversation taking place between the other members at the table.
Dayna returns to the device approximately two and a half minutes later on and examines the photos with Cally, as concluded in \autoref{frag:empirical pub findings newinfo-iv}. % 2min 31s

\newpage
\begin{fragfloat*}
    {\fragresubmission{JR-3c, ER-H: Revised fragment that is shorter, with labels on the image for the speakers}
    \begin{transcript*}[28]
        \by DAY {((looks down and unlocks} \\
        \by     {phone)) \intDown oh tha\emph{t}: thing } \\
        \im 1   {Graphics/3-1-Empirical-Pub/FragmentSchnauzer-5.png} 
        \by CAL {((leans towards DAY and} \\
        \by     {shifts gaze towards her} \\
        \by     {screen)) yes look at them} \\
        \by     {oo:::\intUp} \\
        \by DAY {\quiet{schanuzer}\vspace*{1.9cm}}
    \end{transcript*}
    \caption{Miniature Schnauzers (iv)}\label{frag:empirical pub findings newinfo-iv}
    }
\end{fragfloat*}

In this final excerpt of the fragment, Dayna has looked down and unlocked her device, resuming the interactional project to retrieve new information, which was halted because her \QF*[26--27]{internet is rubbish so
this may take some time}.
After unlocking her phone, she utters the remark \QF[28]{oh that thing}, reporting that the search results have loaded---and through this alludes to her peripheral awareness of the breed.
Cally, in turn, resumes the posture she held earlier to participate in device use, i.e. that of leaning towards Dayna and looking at her screen; then, while pointing at the images on Dayna's screen, she utters \QF[31]{look at them}, reaffirming her opinion that she considers the breed to be cute, although Dayna fails to acquiesce to this opinion.
Following this fragment, Dayna locks and puts her phone down, and both her and Cally resume conversing with the others at the table.

This fragment thus far shows a typical sequence of members using a mobile device in pub talk unproblematically.
In this case, it was to introduce new information for conversation.
Cally did this by getting Dayna to look up new information and then collaborating with Dayna to complete the task which she instigated.
This fragment should provide an initial sense of the activity of bringing a device into conversation, and how the device use does not necessarily become a topic in of itself and is ostensibly treated as perspicuous to the setting and activity of conversing in a pub.
Device use within pub talk is punctuated by pauses and halts to the device interaction as members wait for responses, or reorient between `conversational floors'~\citep{Edelsky1981} (a pause of 12 seconds between Fragments \ref{frag:empirical pub findings newinfo-ii} and \ref{frag:empirical pub findings newinfo-iii} and over 2 minutes between Fragments \ref{frag:empirical pub findings newinfo-iii} and \ref{frag:empirical pub findings newinfo-iv}).
The next two fragments present an increasingly complex depiction of the phenomena, in which the interactional projects of members consist of using a device to make a joke and to contest a response in an argument, and in which the problems being dealt with in interaction deviate from collaborative efforts exhibited by members thus far.
\end{revisedsubmission}



% *********************************************************************************************************************



\subsubsection{Methodical accomplishments in this fragment}\label{sec:empirical pub findings newinfo methods}
\begin{revisedsubmission}
Before turning to the next fragment, a quick recap is provided on the methodical accomplishments of the members in this fragment in successfully using a device to add new information to a conversation.
Firstly, conversation progresses on to different dog breeds, and it is established how Dayna is unfamiliar with a given breed.
This information deficit occasions a member, Cally, to \textbf{instruct Dayna to use her device} to use Google to search for the breed.
Once ready to perform the task, Dayna seeks clarification on the search terms to use.
Cally \textbf{clarifies the activity} by spelling out the words to use, telling Dayna to look at images, and all the while looking at Dayna's phone screen.
Due to her slow Internet connection, Dayna \textbf{temporarily puts her phone down} while the results load and then talks with the others at the table.
A few moments later, Dayna unlocks her phone and \textbf{looks at the screen} to see the results, Cally joins in too by leaning towards Dayna and \textbf{examining the images displayed}.
They both \textbf{acknowledge the information sought has been found}.
\end{revisedsubmission}



% *********************************************************************************************************************



\crpagebreak\subsection{Making a joke}\label{sec:empirical pub findings joke}
\begin{revisedsubmission}[JR-3a, JR-3c: This section is new, reconstructed from prior analysis]
The first fragment has shown how device use can be occasioned in and through the conversation to introduce new information.
The next fragment of data, titled \textit{Font Size}\footnote{The complete fragment is included in \appref{app:fragments-pub fontsize}.}, exhibits how members are able to retopicalise conversation and introduce device use to the floor, in this case to make a joke at the expense of another person.
In this fragment, which commences below in \autoref{frag:empirical pub findings joke-i}, there are five friends including the researcher.
They are currently discussing Christmas food and as we join them in \autoref{frag:empirical pub findings joke-i}, Lawrence has recently returned to the table from buying another drink from the bar and began using his device.
Jayne is currently recalling a pub she visited in September, which had Christmas `stuff' out.

\begin{inlinefrag*}
    {\fragresubmission{JR-3c, ER-H: Revised fragment that is shorter, with labels on the image for the speakers}
    \begin{transcript*}
        \im 2   {Graphics/3-1-Empirical-Pub/FragmentEmail-2.png}
        \by JAY {beginning of september they} \\
        \by     {had their (.) all their} \\
        \by     {christmas stuff out (.) and I} \\
        \by     {was °like oh my god nobody} \\
        \by     {(\qquad \qquad \qquad )°} \\
    \end{transcript*}
    \caption{Font size (i)}\label{frag:empirical pub findings joke-i}
    }
\end{inlinefrag*}

In this excerpt, Lawrence's device use remains not explicitly accounted for thus far, as the group continue to discuss what is deemed by the members to be the absurdity of Christmas paraphernalia set out in a pub in September. Lawrence will then introduce a new topic to the conversion to make a joke unrelated to the conversation at hand.
As has been established above, conversation in pubs is informal and vacillatory, allowing members to alternate to-and-from device use.
Pub talk is shown to feature disjunct topic shifts that are immediate and disconnected from each other, and ultimately, topics in conversation do not require formal resolution or agreement.
Following on from this, Lawrence undertakes two actions:
\begin{enumerate}[label=(\roman*)]
    \item \nameref{sec:empirical pub findings joke floorprep}, and
    \item \nameref{sec:empirical pub findings joke delivery}.
\end{enumerate}
These actions are unpacked in the next two sections to reveal how they are formed as constituent activities of pub talk.
\end{revisedsubmission}



% *********************************************************************************************************************



\subsubsection{Preparing the floor for a joke}\label{sec:empirical pub findings joke floorprep}
\begin{revisedsubmission}
While the discussion continues, Lawrence has been using his device solitarily.
He then, as seen in \autoref{frag:empirical pub findings joke-ii}, moves the phone close to his face while simultaneously shifting his head such that he is looking closely at his device's screen and utters \QF[06]{jesus}.

\begin{inlinefrag*}
    {\fragresubmission{JR-3c, ER-H: Revised fragment that is shorter, with labels on the image for the speakers}
    \begin{transcript*}[6]
        \im 3   {Graphics/3-1-Empirical-Pub/FragmentEmail-3.png}
        \by LAW {°jesus!°} \\*
        \by JAY {we just booked ours (1.0) we} \\
        \by     {do me and liam and james and} \\
        \by     {malcolm do (one every year and} \\
        \by     {we) just booked it} \\*
        \by MAL {du bois\intUp} \\*
%        \im 1   {Graphics/3-1-Empirical-Pub/FragmentEmail-4.png}
        \by LAW {=sorry (.) have you (.) um (.)} \\
        \by     {((jovially)) jonathan has sent} \\
        \by     {round an email (.) this is} \\
        \by     {great for your study isn't it?}
    \end{transcript*}
    \caption{Font size (ii)}\label{frag:empirical pub findings joke-ii}
    }
\end{inlinefrag*}

In this excerpt, the conversation continues without remark from the others regarding Lawrence's \QF[06]{jesus}.
He then interrupts the group, as Jayne and Malcolm are discussing their Christmas meal plans with the other people at the table, by first issuing an apology for the interruption (\QF{sorry}, line 12), and then chuckling while drawing attention to an email he has received from a third party, Jonathan, by asking whether others have seen an email that has been sent to a mailing list (lines 12---14).
As the friends are all students at the same university, it is possible the email was sent to a mailing list which Lawrence assumes all others will have received too.
Finally, he remarks to the researcher that \QF*[14--15]{this is great for your study}.

Through this interruption to the conversation, Lawrence is introducing a new topic to the group that is unrelated---he acknowledges this with an apology---and then prepares his friends for his upcoming device use.
In this, he is establishing that the new topic he is introducing is about an email that has been sent around.
However, at this point and without further details, others at the table are unlikely to be able to respond to the question without further details of the email in question.
Thus, this utterance, which is delivered while chuckling, occasions a device to be introduced into conversation in order to demonstrate the reason for which the topic was brought up.
Lawrence provides incomplete detail of the email by remarking that it was Jonathan who \QF*[13--14]{has sent round} the email, but the incompleteness of his description establishes that he will give further detail in a forthcoming utterance.
\end{revisedsubmission}



% *********************************************************************************************************************



\subsubsection{Delivering a joke}\label{sec:empirical pub findings joke delivery}
\begin{revisedsubmission}
While Lawrence has established that the email is remarkable by bringing it up, and has alluded to a humorous aspect through the jovial delivery of his interruption, he has yet to establish the specifics of the email to others.
In this final excerpt, below in \autoref{frag:empirical pub findings joke-iii}, Lawrence establishes and delivers the joke, using his mobile device, to demonstrate what he perceives as an absurd font size.
  %As has been established above, conversational topics chop and change with a certain incipience in such a setting
  %Lawrence’s interruption is responded to by members through engage with the proposed topic — Zoe leans in to examine the screen, made available by Lawrence rotting the device, and Malcolm proposes a reasoning for this
  %This final fragment establishes how device use can be used as a resource to occasion and sustain new conversation topics as part of pub talk

\begin{inlinefrag*}
    {\fragresubmission{JR-3c, ER-H: Revised fragment that is shorter, with labels on the image for the speakers}
    \begin{transcript*}[16]
        \im 1   {Graphics/3-1-Empirical-Pub/FragmentEmail-4.png}
        \by RES {going to have to zoom in for} \\
        \by     {the camera (.) it's only set} \\
        \by     {to 720p!} \\
        \by JAY {mu:::::a::h} \\
        \by LAW {yeah (.) that's (.) that (.)} \\
        \by     {that's the email!-} \\
        \later  {\ldots}[4] \\
        \by MAL {is that him or is that your} \\
        \by     {phone fitting the line in?}
    \end{transcript*}
    \caption{Font size (iii)}\label{frag:empirical pub findings joke-iii}
    }
\end{inlinefrag*}

As Lawrence rotates his device to others to show them the email, the researcher comments upon him having to \QF*[16--17]{zoom in for the camera} as it becomes clear to others the remarkable feature of the email, for which is the basis of the joke, is the small font size.
Lawrence continues to joke about the font size---reiterating that \QF[21]{that's the email}. 
A discussion unfolds as Malcolm verifies the basis of the joke, establishing whether that it is a display issue or the email itself.
The conversation, as before, then continues to be around the newly introduced topic of the email formatting.
Through his preparation of the floor for his joke, and his delivery through rotating his device screen, Lawrence has established a joke by ridiculing the sender of the email and the formatting of the email message.

This fragment reveals how Lawrence, in using his device in pub talk, is able to establish a joke using the device as a resource in conversation and occasioning a conversational topic shift to his device use.
This fragment exhibits how a device can be used by members as part of previously undisclosed matter in pub talk and that it is not solely a resource for previously established conversational topics.
The informality of pub talk means that such occasioned interaction is not treated as a problematic interaction by other members by acquiescing to the topic change and engaging with the delivered joke.
In the next fragment, this understanding of pub talk will be further extended with an exhibit of how device use is used to contest an argument, but in lieu of a resolution, a topic change unfolds.
\end{revisedsubmission}



% *********************************************************************************************************************



\subsubsection{Methodical accomplishments in this fragment}\label{sec:empirical pub findings joke methods}
\begin{revisedsubmission}
In this fragment, a member, Lawrence, is already using his device, although the specifics of this engagement are unaccountable to the others in the setting.
Lawrence then \textbf{interrupts the existing conversation}, firstly with an apology, and then \textbf{asks an ostensibly rhetorical question} to those who are co-present regarding an email he has received.
He does not provide the specifics of the email at this stage but then \textbf{rotates his phone around} so that the other members at the table can see the email on his screen, and jovially remarks that \QF{that's the email}.
This remark is predicated on the small font size with which the email is displayed on his device, which is accountably visible to other members as a result of him rotating his phone.
Through the delivery of this joke, and given the vacillatory nature of pub talk, Lawrence \textbf{introduces a new conversational topic} about the email he has received.
\end{revisedsubmission}



% *********************************************************************************************************************



\subsection{Attempting to contest an argument}\label{sec:empirical pub findings contest}
\begin{revisedsubmission}[JR-3a, JR-3c: This section is new, reconstructed from prior analysis]
The next fragment, called \textit{Shorthand}\footnote{The complete fragment is included in \appref{app:fragments-pub shorthand}.}, focuses on what, at a glance, may seem to demonstrate device use occasioned to attend to the same matter as above (i.e. that of introducing new information to the conversation).
However, through unpacking the data in this section, the actions of members will be shown to be driven by a fundamentally different members' problem.
In this fragment, which commences below in \autoref{frag:empirical pub findings contest-i}, the same friends are returned to; they are currently discussing the notion of observational studies, ethnographies, and making fieldnotes.
This, in turn, leads to a discussion about shorthand notations.
The fragment begins with Lawrence questioning Jayne whether shorthand notations are \QF[01]{mainly phonetic}, with Jayne's indirect response occasioning a disagreement between the two co-interlocutors.

\begin{inlinefrag*}
    {\fragresubmission{JR-3c, ER-H: Revised fragment that is shorter, with labels on the image for the speakers}
    \begin{transcript*}
        \im 1   {Graphics/3-1-Empirical-Pub/FragmentShorthand-1.png}
        \by LAW {isn't it mainly phonetic?} \\
        \by JAY {it's like:\vspace*{1.65cm}} \\
        \later  {3.2} \\
        \im 1   {Graphics/3-1-Empirical-Pub/FragmentShorthand-2.png}
        \by JAY {there's various versions so} \\
        \by     {the one she tried to teach me} \\
        \by     {first so i could start going} \\
        \by     {is missing out all the vowels} \\
        \by  LAW {((briefly looks at JAY while } \\
        \by      {picking up his phone, he then} \\
        \by      {(begins to use his phone once } \\
        \by      {he has it in his hands))} \\
    \end{transcript*}
    \caption{Shorthand (i)}\label{frag:empirical pub findings contest-i}
    }
\end{inlinefrag*}

In this opening exchange, Lawrence establishes his perception that shorthand notations are \QF[01]{mainly phonetic}, to which Jayne fails to provide a direct response to the question (i.e. of \textit{yes} or \textit{no}).
Instead, her response implies \textit{no} through the use of the opening phrase \QF[04]{there's various versions}, which she follows with an explanation of a shorthand notation based on omitting vowels (i.e. a notation that is \textit{not} phonetic).
Through her response to the question, Jayne establishes a disagreement with Lawrence that shorthand notations are mainly phonetic.
Lawrence, instead of acquiescing, in turn, picks up his device and (as is revealed later on) begins to search for information on shorthand notations (lines 08 onwards), demonstrating his disagreement, or least dissatisfaction, with Jayne's as-yet-incomplete response.
What follows on from this opening activity is two key activities that further demonstrate the complexity and intricacy in which device use is interwoven within conversation in the pub, and exemplifies how troubles occur as a result of device interaction:
\begin{enumerate}[label=(\roman*)]
    \item \nameref{sec:empirical pub findings contest halting}, and
    \item \nameref{sec:empirical pub findings contest resuming}.
\end{enumerate}
Both of these actions are unpacked respectively, with relation to how they are formed as constituent activities while using a mobile device in pub talk.
\end{revisedsubmission}



% *********************************************************************************************************************



\subsubsection{Halting the conversation to attempt resolution}\label{sec:empirical pub findings contest halting}
\begin{revisedsubmission}
A continuation of the fragment is given in \autoref{frag:empirical pub findings contest-ii}.
The conversation between the group continues with Jayne giving an explanation of a second shorthand notation;
Malcolm makes a remark that cannot be deciphered but which is cut off by Lawrence through his utterance of \QF[22]{hang on}.
Through this interjection, Lawrence attempts to stop the progression of the conversation between the other members.
He then begins to phonate the word ``shorthand'' while typing on his device (lines 22---24), accountably searching for the information in response to his question.

\begin{inlinefrag*}
    {\fragresubmission{JR-3c, ER-H: Revised fragment that is shorter, with labels on the image for the speakers}
    \begin{transcript*}[12]
        \by LAW {\intDown{}yeah} \\
        \im 1   {Graphics/3-1-Empirical-Pub/FragmentShorthand-4.png}
        \by JAY {and once you get good at that} \\
        \by     {you just write a lot quicker} \\
        \by     {(0.7) but then she had one} \\
        \by     {which was literally like (.)=} \\*
        \by LAW {\intDown{}yeah} \\
        \by JAY {=swiggles and just didn't look} \\
        \by     {like anything and i don't know} \\
        \by     {if that's phonetic or::::} \\
        \by MAL {(\qquad \qquad \qquad \qquad )} \\
        \im 1   {Graphics/3-1-Empirical-Pub/FragmentShorthand-5.png}
        \by LAW {hang on! ((typing on phone} \\
        \by     {with thumbs)) schuh::::::::ort} \\
        \by     {(.....) hand (..)} \\
        \by     {my mum's regular handwriting} \\
        \by RES {i know some people who miss} \\
        \by     {out vowels (.) like the e\vspace*{.4cm}} \\
        \im 1   {Graphics/3-1-Empirical-Pub/FragmentShorthand-6.png}
        \by JAY {that's how i do it (.) missing} \\
        \by     {out vowels is very very good} \\
        \by     {but there's a squiggly one i} \\
        \by     {don't understand\vspace*{0.5cm}} \\
    \end{transcript*}
    \caption{Shorthand (ii)}\label{frag:empirical pub findings contest-ii}
    }
\end{inlinefrag*}

Lawrence’s issuance of a command to the co-present others to \QF[22]{hang on}, followed by elongation of the phonation of the word \textit{short} (line 23) ensure he continues to hold the floor in conversation.
As is revealed through a post-observation chat, he was using his device to search for information on shorthand notations in order to support his point.
In other words, Lawrence stops the progression of conversation and holds the floor, to retrieve information in order to resolve the disagreement that occasioned the current conversational topic.

Through his device use and his verbal articulation, Lawrence exhibits how he is attempting to resolve the disagreement---by searching on his device for information to support his point.
Yet, he does this concurrently while an answer to his question is being given; he then interrupts others' talk and holds the floor while he completes his occasioned device use.
Such actions reinforce the notion that conversation within the setting is treated as an informal activity.
Through the mundanity with which he and other members treat his device use, which is ostensibly not interleaved in the ongoing multi-party conversation, and his holding of the floor while using the device, it is established how device use and conversation are treated with a sense of informality both by Lawrence \textit{and} his interlocutors.

In this situation, not only is Lawrence's device use not ostensibly interleaved with the conversation, there is also an attempt to halt the progression of conversation while he completes the device use.
Nevertheless, this request is not honoured by the conversationalists and the talk between the other members continues.
Initially, the researcher first renews the topic of shorthand notation at the table before others continue the discussion while Lawrence uses his device away from the group.
\end{revisedsubmission}



% *********************************************************************************************************************



\subsubsection{Resuming without resolution of the disagreement}\label{sec:empirical pub findings contest resuming}
\begin{revisedsubmission}
The next part of this fragment also further brings to the fore the demonstrable ways in which members' interactions establish the notion of the pub as a casual-social setting and that of pub talk as being imbued with informality.
Thus far, device use is showed as being used as a resource in conversation to retrieve further information (as per the prior Miniature Schnauzers example in \ref{sec:empirical pub findings newinfo}).
Whereas in the prior case, device use is shown to be used to address the members' problem, in this fragment, continued in \autoref{frag:empirical pub findings contest-iii}, conversation is shown to continue without the intended purpose of device use being concluded, with a disjunct topic shift occurring.
%Jayne (indirectly) answers the question of whether shorthand is “mainly phonetic” by given two non-phonetic examples, although Lawrence’s halting of conversation while searching for shorthand information on his device exemplifies a distinctly non-collaborative device use achievement 

\begin{inlinefrag*}
    {\fragresubmission{JR-3c, ER-H: Revised fragment that is shorter, with labels on the image for the speakers}
    \begin{transcript*}[32]
        \by MAL {this is why i didn't do} \\
        \by     {ethnography (1.8) just get the} \\
        \by     {participants to fill} \\
        \by     {everything out} \\
        \by LAW {i didn't do ethnography either} \\
        \by     {either!} \\
        \by MAL {yeah\intUp{} you do it- i'm not- i'm} \\
        \by     {not doing=} \\ \newpage 
        \im 1   {Graphics/3-1-Empirical-Pub/FragmentShorthand-7.png}
        \by LAW {=oo\intUp{} that's got (2.0) that's} \\
        \by     {cinnamon in it or something} \\
        \by     {something (..) smells amazing\vspace*{1cm}}
    \end{transcript*}
    \caption{Shorthand (iii)}\label{frag:empirical pub findings contest-iii}
    }
\end{inlinefrag*}

Malcolm's remark that he \QF*[32--33]{didn't do ethnography} is responded to by Lawrence with a claim that he \QF*[35--36]{didn't do ethnography either}, at which point he stops using his device (although still continues to hold it).
Malcolm responds to this retort but is interrupted by Lawrence who shifts the topic of conversation on to Zo\"{e}’s food by shifting his posture towards her, leaning in towards her plate, and remarking that \QF*[40--41]{oo that's got cinnamon in it}.
Following this, a conversation about food, and Christmas food in particular, unfolds, with the disagreement remaining unresolved and not returned to by any members.

% The continuation of conversation after Lawrence's \QF{hang on} is typical of pub talk---the lapse in conversation that occurs as a result of his request is readily dealt with through the resumption of talk.
% Members do not---as Lawrence requests---\QF{hang on}, instead continuing the conversation at hand.
% There is a lack of formality in pub talk, and the relatively small number of people in the gathering, show how conversation continues while 

This fragment, in its entirety, establishes how using a device while someone is talking to you, and interrupting a conversation to use your device, although highlighted as potentially problematic, may not be treated as such by members in and through pub talk.
The first fragment of interaction in this chapter, which focused on Miniature Schnauzers, concluded with the members returning to the device to complete and resolve the problem at hand.
The second fragment further brought into play how conversation topic changes occur within pub talk and render prior discussion points amongst friends as unresolved.
Moreover, this fragment further demonstrates that given the fleeting nature with which conversation progresses and how the topic being discussed changes, reasons for which a device use was occasioned may not be satisfied---resolution of topics or discussions may not necessarily occur in pub talk, and throughout the corpus, was not treated as problematic.

Furthermore, the continuation of talk, and the later retopicalisation of the conversation on to Zo\"{e}'s food demonstrate the vacillatory nature of pub talk, i.e. the topic at hand in a conversation can change in rapid succession, with members changing between activities such as device use and talking with relative ease and wavering between the two.
Although Lawrence began using his device to resolve the disagreement in talk, he then changes the topic readily, without recourse, and without others calling for him to resolve the disagreement.
In other words, the topic of shorthand is not returned to---Lawrence provides no results from his device use, and nor is this sought for by other members.
\end{revisedsubmission}



% *********************************************************************************************************************



\subsubsection{Methodical accomplishments in this fragment}\label{sec:empirical pub findings contest methods}
\begin{revisedsubmission}
In this final fragment for this chapter, a conversation about shorthand notations is ongoing.
Lawrence asks Jayne a question regarding different notations but \textbf{commences to use his device} as an indirect answer is given by Jayne.
Lawrence interrupts Jayne's answer by \textbf{requesting that she waits for him to use his device} and then he \textbf{phonates the word shorthand} while typing on his phone's screen, accounting for the specifics of his device use being to search for information to contest his point.
After his phonation, Lawrence continues to use his device however the \textbf{conversation amongst the others at the table continues} without waiting for him to complete his self-occasioned task.
Lawrence eventually \textbf{returns to talking with others} without resolution of the contested point by \textbf{stopping the use of his phone}.
\end{revisedsubmission}



% *********************************************************************************************************************



\section{Chapter summary}\label{sec:empirical pub summary}
\begin{revisedsubmission}[JR-3a, JR-3c: This section has been revised in accordance with the new findings section]
This study not only provides a basis for understanding how mobile device use is embedded within a multi-party conversation in a pub, but also for how the interactional accomplishment of sustaining the device use is achieved when the mobile device relies entirely on line-of-sight and touching the device screen.

These three fragments of interaction have increasingly introduced the intricacy through which mobile device use unfolds as an interwoven activity within pub talk.
Across each fragment, device use is shown to move quickly, with conversational topics changing quickly, to and from device use, and disjunctively to new topics.
This device use is interleaved within this conversation---members were shown to bring a device into conversation to introduce new information, to resolve disagreement, or to make a joke; and to stop using a device due to sluggishness of the device, once the problem for which the device use was occasioned is `resolved', or as a new topic is introduced.
Crucially, this interleaving is shown to be treated as a consistent and inseparable activity of pub talk and that the device use by members is interleaved within the sociality of the setting.
The device is used \textit{as part of} conversation in a pub, and is used in and through that conversation by members.

Furthermore, device use was shown to be engaged with collaboratively in searching for new information together, or  individually and separate from conversation.
While this latter case could be construed as problematic (and indeed has been by others as discussed above in \ref{sec:empirical pub intro}), it is noteworthy how in this video-supported ethnography that members performing such actions were seemingly not identified as problematic in interaction by others.
In the three fragments presented, as exhibits of the complete corpus, device use was not explicitly oriented to by members as unnecessary, rude, or distracting in talk, even in cases where conversation was interrupted to allow for device use to unfold.
While of course, some people will argue that device use is any or all these things (again, see \ref{sec:empirical pub intro}), the people observed in this ethnography reveal that this is not a universal case.
The resulting conclusion of these observed practices is that device use, as supposed prior to data collection, is treated as perspicuous to the pub as an activity interleaved within conversation.

In summary, device use was occasioned as a result of---and in spite of---conversation, and treated by members as a mundane activity within conversation in a pub, even if that use was not topicalised in conversation.
Furthermore, the informality of the setting and that of pub talk further mean that resolution of the occasioned device use is not necessarily made accountable.
\end{revisedsubmission}



% *********************************************************************************************************************



\subsection{Methodical accomplishments}\label{sec:empirical pub summary methods}
\begin{revisedsubmission}
This chapter has stepped through three fragments introducing the pub talk in which members have occasioned and brought device use into the conversation.
This occasioning was done to introduce new information to the conversation, to make a joke, and to attempt to contest an argument.
Practically, this occasioning practice took the form of instructing others to use their device, choosing to use the device oneself, or using a device as ostensibly unrelated to the conversation at hand.
Through the interleaving of device use in interaction, members accounted for their device use---i.e., they made the specific nature of their interaction observable and reportable---by rotating their phone, by holding it between people to make the screen visible, or by phonating what they were doing on the phone.
Device use ended (temporarily or entirely for the occasioned task) as an interleaved activity as the conversation topic changed, as devices took time to respond, or that the interactional project of using the device was accountably completed.
\end{revisedsubmission}



% *********************************************************************************************************************



\crpagebreak\subsection{Outlook}\label{sec:empirical pub summary outlook}
\begin{revisedsubmission}
Study two, in \autoref{ch:empirical cafe}, considers how such gatherings unfold when the use of the device occurs through speech using a \acf{VUI} on the smartphone in a similar setting: a caf\'{e}.
\acp{VUI} are different from \acfp{GUI} in that they provide a mechanism for people to interact with a device through their voice, although in the case of smartphones the \ac{VUI} also displays its computation and output on the screen of the device.
It is so posited that the consequences of interaction occurring through this paradigm are that the tasks being completed using a \ac{VUI} become hearable to those within earshot, and that through this, the device use interleaved in conversation in a casual-social setting is made naturally accountable.
While interactions around the device use in this chapter were shown to be collaborative at times, it is proposed here that by requesting members to make a preference for voice-based device interaction, the interaction with the \ac{VUI} may lend itself to occasioning further collaborative efforts amongst the co-interlocutors.
How members attend to conversing with each other while device use unfolds through voice as well as the use of the touchscreen, and how members attend and orient to this device use, will become a focus of the next chapter. % when there is no control over who can observe with the device interaction due to the accountable nature odf talk, 
\end{revisedsubmission}

% The purpose of leisure-time socialising is that of conversing and spending time with friends in a group.
% Through the actions of the members of the group, mobile devices can become occasioned in a number of different ways.
% This work identified two key forms of mobile device use occasioning within the context:
% \begin{itemize}
%     \item occasioning that was related to and done \textit{in and through} the conversation
%     \item occasioning that was ostensibly \textit{unrelated} to the conversation
% \end{itemize}
% These two forms in which mobile device use is occasioned will now be brought to the fore\begin{revisedsubmission}, and woven into the this thesis' work of unpacking the various fragments of data presented.\end{revisedsubmission}

% \begin{revisedsubmission}[Introduce a new section on ‘Using your phone’ with a previously used fragment]
% The first episode of interaction to be unpacked here is of a conversation with five friends present at the table: Lawrence, Malcolm, Z\"{o}e, Jayne, and the researcher.
% \end{revisedsubmission}
% In the episode, given in \autoref{frag:empirical pub findings contest}, the group are discussing shorthand notation.
% Lawrence has expressed an interest in learning shorthand notation, and Jayne explains that she had previously been taught one form of notation although also discusses another form of which she is aware.
% \begin{revisedsubmission}As will be revealed, Lawrence begins to use his device and makes accountable this device interaction to those in the setting through the use of body (co)orientation, gaze, and talk to highlight their device interaction.
% This is a commonly observed approach in the data collected in which one member continually articulates their actions to collocated others.
% This practice turns upon the device user offering a verbal account for the device use, either in the form of specific detail such as verbalising what they are typing, or through the production of an abstracted description of the task they are attempting to complete.
% \end{revisedsubmission}

% \begin{inlinefrag*}
%     \resubmission{This fragment also exemplifies a common use of using a mobile device successfully as intended}
%     \begin{transcript*}
%         \by LAW {\intDown{}yeah\vspace*{0.2cm}} \\
%         \im 1   {Graphics/3-1-Empirical-Pub/FragmentShorthand-4.png}
%         \by JAY {and once you get good at that} \\
%         \by     {you just write a lot quicker} \\
%         \by     {(0.7) but then she had one} \\
%         \by     {which was literally like (.)=} \\*
%         \by LAW {\intDown{}yeah} \\
%         \by JAY {=swiggles and just didn't look} \\
%         \by     {like anything and i don't know} \\
%         \by     {if that's phonetic or::::} \\
%         \by MAL {(\qquad \qquad \qquad \qquad )} \\
%         \im 1   {Graphics/3-1-Empirical-Pub/FragmentShorthand-5.png}
%         \by LAW {hang on! ((typing on phone} \\
%         \by     {with thumbs)) schuh::::::::ort} \\
%         \by     {(.....) hand (..)} \\
%         \by     {my mum's regular handwriting} \\
%         \by RES {i know some people who miss} \\
%         \by     {out vowels (.) like the e\vspace*{.4cm}} \\
%         \im 1   {Graphics/3-1-Empirical-Pub/FragmentShorthand-6.png}
%         \by JAY {that's how i do it (.) missing} \\
%         \by     {out vowels is very very good} \\
%         \by     {but there's a squiggly one i} \\
%         \by     {don't understand} \\
%         \by MAL {this is why i didn't do} \\
%         \by     {ethnography (1.8) just get the} \\
%         \by     {participants to fill} \\
%         \by     {everything out} \\
%         \by LAW {i didn't do ethnography either} \\
%         \by     {either!} \\
%         \by MAL {yeah\intUp{} you do it- i'm not- i'm} \\
%         \by     {not doing=} \\
%         \im 1   {Graphics/3-1-Empirical-Pub/FragmentShorthand-7.png}
%         \by LAW {=oo\intUp{} that's got (2.0) that's} \\
%         \by     {cinnamon in it or something} \\
%         \by     {something (..) smells amazing\vspace*{1cm}}
%     \end{transcript*}
%     \caption{Hang on! Shorthand\ldots}\label{frag:empirical pub findings contest}
% \end{inlinefrag*}

% \begin{revisedsubmission}
% This episode presents several methods in the accomplishment of embedding device use within the conversation.
% At the start of the fragment, Lawrence is holding his phone in his hand and has been since returning to the table from a toilet break.
% The conversation about different shorthand notations is progressing, with Lawrence asking Jayne \QF[01]{isn't it mainly phonetic}.
% He keeps his eye gaze and fixed on Jayne as she shifts from describing notation in words to drawing shapes with her hands on the table.
% Throughout this exchange. as Jayne begins to explain the different notations of which she is aware, Lawrence's back remains against the wall watching her as she puts down her glass and gestures with her hands (lines 02---08).
% He then acknowledges her statement about omitting vowels (line 08), and then as Jayne continues to explain the benefit, he shifts head and gaze downwards to his phone, which is between his hands.
% This short segment of interaction exemplifies how device interaction can be \textit{occasioned in and through} in which an individual \textit{chooses to use their own phone}.
% In other words, the conversation topic at hand and the vague description offered by Jayne are seen to occasion Lawrence's device use (although this is not made accountably clear until later in the sequence, as is discussed below).

% Following Jayne's initial description and Lawrence shifting his focused gaze to his device, Jayne continues to expanded her description of the shorthand notations of which she is aware (lines 09-12), while Lawrence acknowledges again with \QF[13]{\intDown{}yeah}.
% Here, Lawrence \textit{demonstrates his continual social presence} when he responds to Jayne's explanation of the first shorthand notation that she learnt with his remark \QF[13]{\intDown{}yeah}.
% While Lawrence demonstrates that he is maintaining awareness of the verbal aspects of the conversation with this utterance, he fails to respond to gestures used by Jayne to exemplify her explanations.
% Jayne makes two visual contributions to the interaction, first demonstrating the form of shorthand in the air (as seen in the second image in the fragment) and then on the table as she explains the second form of shorthand.
% This accountable omission suggests that Lawrence has perhaps withdrawn from at this point.

% However, Lawrence addresses this momentary withdrawal and reasserts his involvement in the conversation when he takes the floor (line 18) following Malcolm's turn (line 17, untranscribable); first by uttering \QF[18]{hang on!}, and then by accountably spelling out the word \textit{shorthand} while typing on his phone (lines 18---19).
% Although he does not provide visual confirmation to the other members, or indeed, specific explanation of what his exact interaction is with his device, the clear implication of asking others to \QF[18]{hang on} and his spelling out the phonetic sounds is that he is presently typing the word into a search engine.
% A later conversation confirmed that he was using a search engine to find information about shorthand.

% Therefore, this fragment brings to the fore the accountable practice of device use being occasioned in and through conversation, and that of how members accountably articulate device use within a conversation.
% Lawrence's no-response to Jayne's visual demonstration and explanation raises the prospect that Lawrence had temporarily withdrawn from the conversation, with his gaze and body posture demonstrating his focus upon his device use.
% However, he then \textit{embeds} this device use in conversation by occupying the floor while using the device, first calling for others to \QF[18]{hang on}, and then by phonetically spelling out search terms that he is typing with his device (lines 18--19).

% In addition to articulating actions, the process of providing a continual account of actions can include more visual aspects such as making the device screen visible (i.e. available for glancing), as shown elsewhere in data.
% This could be to allow others to engage as a spectator during the device use, or alternatively to engage in a collaborative task that allows one or more other members of the setting to contribute.
% Both of these behaviours are documented as members were searching for information to contribute, or to corroborate members' opinions.
% \end{revisedsubmission}

% \resubmission{This title was changed to match the titles introduced above}
% \subsubsection{Device use ostensibly unrelated to the conversation}\label{sec:empirical pub findings unrelated}
% \begin{revisedsubmission}
% While the above two sections discuss how device use is brought into conversation, occasioned in and through the conversation, this is not the only method to which device use becomes interleaved.
% \end{revisedsubmission}
% Members can choose to engage with their mobile device for a variety of factors external to the conversation.
% For example, device use may be prompted by a device notification, or the individual can choose to use the mobile device at will.
% During the observations, 14 instances in which there were observable-reportable notifications and subsequent device use during the study were recorded, which equates to 51\% of all non-conversation related occasioning.
% There are also a number of excluded episodes from the corpus in which device use is neither accounted for nor topicalised in some way in the conversation.

% %[move para down?] We also observed members apparently withdrawing their attention from the conversation as the result of tending to the device after a notification, which on occasion was even followed by a change of the topic of conversation.
% %It is also noted that members also acknowledged, or dismissed notifications, before re-engaging in the conversation at multiple times.

% \autoref{frag:empirical pub findings email} presents an example in which occasioning that is apparently unrelated to the conversation is followed by topicalisation related to mobile device use in the conversation.
% In this episode, the owner of the mobile device, Lawrence, is using his mobile phone while a conversation is ongoing.
% Lawrence has recently rejoined the group after leaving for some time, with his phone left at the table.
% Upon his return to the table, he picks up his phone and begins to use it.
% He then brings up an email he has recently received, but he does this by interrupting an existing conversation around Christmas meals.
% In total, this episode includes four friends: Lawrence, Malcolm, Z\"{o}e, Jayne, and the researcher.
% The group are joined 14s after Lawrence returned to the table.

% \newpage
% \begin{inlinefrag*}
%     \begin{transcript*}
%         \by     {((LAW is leaning on the table} \\
%         \by     {using his phone))} \\
%         \im 1   {Graphics/3-1-Empirical-Pub/FragmentEmail-1.png}
%         \by JAY {\quick{no no no} i'm just saying} \\
%         \by     {mulled wine is not just} \\
%         \by     {christmas} \\
%         \later  {\ldots}[9] \\
%         \by JAY {i think we went in (.) like} \\
%         \by     {(.) quite late i tweeted i} \\
%         \by     {took a photo and i} \\
%         \by     {tweeted no:\intUp~ cos it's-} \\
%         \im 2   {Graphics/3-1-Empirical-Pub/FragmentEmail-2.png}
%         \by LAW {((stops using device, looks} \\
%         \by     {towards to JAY)) what?} \\
%         \by JAY {beginning of september they} \\
%         \by     {had their (.) all their} \\
%         \by     {christmas stuff out (.) and I} \\
%         \by     {was °like oh my god nobody} \\
%         \by     {(\qquad \qquad \qquad )°} \\
%         \im 3   {Graphics/3-1-Empirical-Pub/FragmentEmail-3.png}
%         \by LAW {°jesus!°} \\*
%         \by JAY {we just booked ours (1.0) we} \\
%         \by     {do me and liam and james and} \\
%         \by     {malcolm do (one every year and} \\
%         \by     {we) just booked it} \\*
%         \by MAL {du bois\intUp} \\*
%         \im 1   {Graphics/3-1-Empirical-Pub/FragmentEmail-4.png}
%         \by LAW {=sorry (.) have you (.) um (.)} \\
%         \by     {jonathan has sent round an} \\
%         \by     {email (.) this is great for} \\
%         \by     {your study isn't it?} \\
%         \by     {((chuckles)) have you seen} \\
%         \by     {the font size?} \\
%         %\by     R {Going to have to zoom in for the camera, it's only set to 720p!}
%         %\by     J {muuuuuaaah}
%         %\by     L {Yeah, that's, that, that's the email!-}
%         \later  {\ldots}[4] \\
%         \by MAL {is that him or is that your} \\
%         \by     {phone fitting the line in?} \\
%     \end{transcript*}
%     \caption{Have you seen the font size?}\label{frag:empirical pub findings email}
% \end{inlinefrag*}
% \newpage

% %In order to identify the method of occasioning within this episode, 
% It must first be considered that although Lawrence has almost continually used his mobile phone since returning to the table, this has ostensibly \textit{not} been occasioned in and through the conversation.
% He briefly takes a hiatus from using his mobile phone to have a drink and perhaps to clarify what the topic of the current conversation is: \QF[17]{what?}, looking at Jayne as he does so.
% However, despite this brief interjection, Lawrence then turns back to using his mobile phone whilst the conversation runs its course, without looking at Jayne as she concludes her explanation.
% In turning back to his phone, he also lifts the device closer to his face, as can be seen within the imagery in the fragment.
% This posture suggests to those present that he is engrossed in studying the contents of his device's screen, which may be seen to display considerable involvement in his mobile device use.

% Lawrence's displayed orientation to the device is crucial to accounting for the later topicalisation of the contents of his screen in the conversation.
% Lawrence interrupts the current conversation, first with an apology for doing so (\QFt[29]{sorry}).
% He then makes his previous actions of holding his phone close to his face accountable by explaining that the email he received appears in a small size on his mobile phone.
% He also further corroborates this explanation by performing a screen-sharing gesture allowing others to see the email on his screen, providing further evidence for both his previous actions and his articulation, as exemplified when he observably shares his phone with others at the end of  \autoref{frag:empirical pub findings email}.

% Finally, this episode also demonstrates how some practices of holding the device, such as `close to face', are remarkable and therefore call for an explicit account to be offered to the co-participants.
% The member's display of holding his mobile device close to his face is accounted for in retrospect by introducing it to the group.
% This practice of accounting for the use took place through interrupting the ongoing group conversation and bringing up the device use-related issue (small font size) that occasioned his holding of the device in a non-naturally accountable way and thus, both a verbal and visual (`showing-and-telling') account was offered.



% *********************************************************************************************************************



%\subsubsection{Occasioning in and through the conversation}\label{sec:empirical pub findings occasioning innt}
% The purpose of leisure-time socialising is that of conversing and spending time with friends in a group.
% Through the actions of the members of the group, mobile devices can become occasioned in a number of different ways.
% In this work, perhaps unsurprisingly given the nature of the context, the analysis revealed that roughly 47\% of occasioning instances, occasioning of device use was related to the conversation.
% That is, nearly half of the time, a member chose or acquiesced to use their device \textit{in and through} the conversation.

% The first exhibit of methodical work to occasion mobile device use in and through the conversation is presented in \autoref{frag:empirical pub findings ballad}.
% The mobile phone use in this episode is as a result of confusion amongst the members of the setting over what the exact definition of a \textit{ballad} is.
% In this episode, four friends are joined: Dayna, Jenna, Cally, Kerr, along with the researcher, as Dayna offers the definition that she believes to be the case.
% Before this discussion, Jenna has her (locked) mobile phone on the table, while the others in the group all have their mobile phones either in their bags or pockets.

% \begin{inlinefrag*}
%     \begin{transcript*}
%         \im 1   {Graphics/3-1-Empirical-Pub/FragmentBallad-1.png}
%         \by DAY {°i thought a ballad is a poem} \\
%         \by     {(.) like twen--twelve lines (.)} \\
%         \by     {or something (.) you know?°} \\
%         \by JEN {(\qquad \qquad \qquad ) sorry?} \\
%         \by     {°POem° oh it's a it can be a-} \\
%         \by KER {oh, poem?} \\
%         \newpage
%         \im 1   {Graphics/3-1-Empirical-Pub/FragmentBallad-2.png}
%         \by DAY {poem with twenty lines no?} \\
%         \by     {it's a ballad!} \\
%         \by JEN {yeah (.) we call it as well a} \\
%         \by     {ballad} \\
%         \by KER {a \loud{ballad}!} \\
%         \by CAL {°oh, right°} \\
%         \by JEN {balla\emph{d}=} \\
%         \by DAY {~~~~~~~=balla\emph{d}} \\
%         \by KER {ah: (.) right (.) OK} \\
%     %%      \by  K {is that the definition for ballad? A poem with twenty lines?}
%     %%      \by  D {\quick{Twelve}}
%     %%      \by  K {Twelve lines, sorry}
%     %%      \by  D {°I think?°}
%     %%      \by  J {Really? It could be (\qquad \qquad \qquad)?}
%     %%      \by  R {I know love ballads are just-}
%     %%      \later  {0.7}
%     %%      \by    {-sort of, slow love songs that are-}
%         \later  {\ldots}[10] \\
%         \by JEN {yeah sometimes there are} \\
%         \by     {romantic songs that can be} \\
%         \by     {called ballad} \\
%         \im 1   {Graphics/3-1-Empirical-Pub/FragmentBallad-3.png}
%         \by DAY {°i think° cos i remember we} \\
%         \by     {were said, we told ((gets} \\
%         \by     {out of bag, holds in lap} \\
%         \by     {phone and then begins to use))}\\
%         \by     {°i think we were told that (.)} \\
%         \by     {to google°}
%     \end{transcript*}
%     \caption{Defining a Ballad}\label{frag:empirical pub findings ballad}
% \end{inlinefrag*}

% This discussion, which lasts 34s from the first utterance to the completion of the last remark, quickly leads to the introduction of a mobile phone to resolve the group dilemma.
% In fact, from the first definition of a ballad, as proposed by Dayna (who is also the member who begins to use her phone), it takes just 27s for her to begin the process of retrieving her phone from her handbag.
% In the episode, Dayna, who in the informal interview later confessed to using her mobile phone \textit{``all the time''}, retrieves her phone from her handbag, and as she does so, continues to clarify her confusion over the exact definition of \textit{ballad}.
% Then, just prior to the commencement of her mobile device use, she provides the confirmation to the group of the task she is about to perform by articulating her intention with \QF[34]{°to google°}.
% This declaration confirms that the purpose of her retrieving the device is that of resolving the group dilemma.

% There are further examples of this as, following this fragment, Dayna uses her phone within her lap, below that of the table edge.
% While using her mobile device, Dayna then ostensibly disengages from the conversation through which the use was occasioned in the first place.
% The conversation amongst the friends quickly reverts to a previous topic that was taking place before this particular tangent occurred.
% Dayna remains disengaged from the conversation for a short  time while she continues to use her phone.

% There are various points within the sequence that, when combined, contribute to the occasioning of the mobile device.
% When Dayna utters \QF[33]{°i think°} signs of self-doubt are emanated, which is then followed by further conflicting definitions from other members.
% This helps to establish a `state of confusion' within the group which is then followed by Dayna's act to retrieve her phone and make the statement \QF[34]{°to google°}.
% By declaring her intent in this way, Dayna justifies her device use by making it accountable to the situation at hand.
% Dayna's actions are accountable to the members as they offer a way of dealing with the confusion, which in turn has contributed to the occasioning of the mobile device use.
% Similar situations in which mobile devices were used to retrieve information using search engines, and resolve conflicts within conversations, occurred in all the groups that took part in the study. %This use positions the phone a mobile device that can deliver an authoritative source of information through mobile search, and the conversation as occasioning the mobile device introduction.



% *********************************************************************************************************************



% \subsubsection{Occasioning ostensibly unrelated to the conversation}\label{sec:empirical pub findings occasioning unrelated}
% Members can choose to engage with their mobile device for a variety of factors external to the conversation.
% For example, device use may be prompted by a device notification, or the individual can choose to use the mobile device at will.
% During the observations, 14 instances in which there were observable-reportable notifications and subsequent device use during the study were recorded, which equates to 51\% of all non-conversation related occasioning.
% There are also a number of excluded episodes from the corpus in which device use is neither accounted for nor topicalised in some way in the conversation.

% %[move para down?] We also observed members apparently withdrawing their attention from the conversation as the result of tending to the device after a notification, which on occasion was even followed by a change of the topic of conversation.
% %It is also noted that members also acknowledged, or dismissed notifications, before re-engaging in the conversation at multiple times.

% \autoref{frag:empirical pub findings email} presents an example in which occasioning that is apparently unrelated to the conversation is followed by topicalisation related to mobile device use in the conversation.
% In this episode, the owner of the mobile device, Lawrence, is using his mobile phone while a conversation is ongoing.
% Lawrence has recently rejoined the group after leaving for some time, with his phone left at the table.
% Upon his return to the table, he picks up his phone and begins to use it.
% He then brings up an email he has recently received, but he does this by interrupting an existing conversation around Christmas meals.
% In total, this episode includes four friends: Lawrence, Malcolm, Z\"{o}e, Jayne, and the researcher.
% The group are joined 14s after Lawrence returned to the table.

% \begin{inlinefrag*}
%     \begin{transcript*}
%         \im 1   {Graphics/3-1-Empirical-Pub/FragmentEmail-1.png}
%         \by JAY {\quick{no no no} i'm just saying} \\
%         \by     {mulled wine is not just} \\
%         \by     {christmas} \\
%         \later  {\ldots}[9] \\
%         \by JAY {i think we went in (.) like} \\
%         \by     {(.) quite late i tweeted i} \\
%         \by     {took a photo and i} \\
%         \by     {tweeted no:\intUp~ cos it's-} \\
%         \im 2   {Graphics/3-1-Empirical-Pub/FragmentEmail-2.png}
%         \by LAW {what?} \\
%         \by JAY {beginning of september they} \\
%         \by     {had their (.) all their} \\
%         \by     {christmas stuff out (.) and I} \\
%         \by     {was °like oh my god nobody} \\
%         \by     {(\qquad \qquad \qquad )°} \\
%         \im 3   {Graphics/3-1-Empirical-Pub/FragmentEmail-3.png}
%         \by LAW {°jesus!°} \\*
%         \by JAY {we just booked ours (1.0) we} \\
%         \by     {do me and liam and james and} \\
%         \by     {malcolm do (one every year and} \\
%         \by     {we) just booked it} \\*
%         \by MAL {du bois\intUp} \\*
%         \im 1   {Graphics/3-1-Empirical-Pub/FragmentEmail-4.png}
%         \by LAW {=sorry (.) have you (.) um (.)} \\
%         \by     {jonathan has sent round an} \\
%         \by     {email (.) this is great for} \\
%         \by     {your study isn't it?} \\
%         \by     {((chuckles)) have you seen} \\
%         \by     {the font size?} \\
%         %\by     R {Going to have to zoom in for the camera, it's only set to 720p!}
%         %\by     J {muuuuuaaah}
%         %\by     L {Yeah, that's, that, that's the email!-}
%         \later  {\ldots}[4] \\
%         \by MAL {is that him or is that your} \\
%         \by     {phone fitting the line in?} \\
%     \end{transcript*}
%     \caption{Have you seen the font size?}\label{frag:empirical pub findings email}
% \end{inlinefrag*}

% In order to identify the method of occasioning within this episode, it must first be considered that although Lawrence has almost continually used his mobile phone since returning to the table, this has ostensibly \textit{not} been occasioned in and through the conversation.
% He briefly takes a hiatus from using his mobile phone to have a drink and perhaps to clarify what the topic of the current conversation is: \QF[17]{what?}, looking at Jayne as he does so.
% However, despite this brief interjection, Lawrence then turns back to using his mobile phone whilst the conversation runs its course, without looking at Jayne as she concludes her explanation.
% In turning back to his phone, he also lifts the device closer to his face, as can be seen within the imagery in the fragment.
% This posture suggests to those present that he is engrossed in studying the contents of his device's screen, which may be seen to display considerable involvement in his mobile device use.

% Lawrence's displayed orientation to the device is crucial to accounting for the later topicalisation of the contents of his screen in the conversation.
% Lawrence interrupts the current conversation, first with an apology for doing so (\QFt[29]{sorry}).
% He then makes his previous actions of holding his phone close to his face accountable by explaining that the email he received appears in a small size on his mobile phone.
% He also further corroborates this explanation by performing a screen-sharing gesture allowing others to see the email on his screen, providing further evidence for both his previous actions and his articulation, as exemplified when he observably shares his phone with others at the end of  \autoref{frag:empirical pub findings email}.

% Finally, this episode also demonstrates how some practices of holding the device, such as `close to face', are remarkable and therefore call for an explicit account to be offered to the co-participants.
% The member's display of holding his mobile device close to his face is accounted for in retrospect by introducing it to the group.
% This practice of accounting for the use took place through interrupting the ongoing group conversation and bringing up the device use-related issue (small font size) that occasioned his holding of the device in a non-naturally accountable way and thus, both a verbal and visual (`showing-and-telling') account was offered.

%\begin{figure}[!ht]
%   \centering
%   \includegraphics[width=0.9\columnwidth]{Graphics/3-1-Empirical-Pub/FragmentGrpA-frag2.png}
%   \caption{Lawrence demonstrates the email to the group, naturally accounting for his previous intense gaze on his mobile phone screen.}~\label{fig:email photo}
%\end{figure}

%Another form of self-section can be where members choose to use their mobile device, for one reason or another, where sometimes this action is not objectively observable in the setting; this could be as a result of attempting to alleviate boredom~\citep{Wei:2006jh}, for example.



% *********************************************************************************************************************



% \subsection{Turn Allocation With Respect to Mobile Device Use}\label{sec:empirical pub findings occasioning turn}
% It is imperative to also consider the relevance of this work in relation to the systematics of turn-taking allocation techniques, as given by \citet{Sacks1974}.
% Accordingly, turn allocation works as either (1) an individual chooses to take the next turn in the conversation, or (2) the current speaker selects who will take the next turn within the conversation.

% These two possible techniques are both observable within situations where mobile devices are occasioned in and through the conversation.
% For example, in the previous episode, it can be noted that it was Dayna who self-selected to use her phone.
% Equally, however, it is possible for a member to allocate device use to another member.
% Here, an episode taken from a later stage of the conversation of the same group is presented.
% \autoref{frag:empirical pub findings newinfo}, which includes a small segment of a larger conversation around dog breeds, is between two of the group's members: Cally and Dayna.
% In this episode, Cally attempts to describe the size of the breed of dog by gesturing with her hands, however, she then follows up this by instructing Dayna to use a phone to look up more information.

% \begin{fragment}
%     \begin{transcript*}
%         \by CAL {i like miniature schnauzers} \\
%         \by DAY {°how big are schn-?°} \\
%         \im 2   {Graphics/3-1-Empirical-Pub/FragmentSchnauzer-1.png}
%         \by CAL {it's like (.) like (.) they're} \\
%         \by     {\emph{so:} cute\vspace{2cm}} \\*
%         \im 1   {Graphics/3-1-Empirical-Pub/FragmentSchnauzer-2.png}
%         \by CAL {((briefly looks at her bag to} \\
%         \by     {her left before looking back))} \\*
%         \by DAY {i like big dogs\vspace{1.9cm}} \\*
%         \im 2   {Graphics/3-1-Empirical-Pub/FragmentSchnauzer-3.png}
%         \by CAL {i know, but google schnauzer,} \\
%         \by     {right?} \\*
%         \by DAY {((gets phone out from bag))} \\*
%         \by CAL {((leans towards DAY)) } \\
%         \by     {the puppies (.) schnauzer} \\
%         \by     {puppies are gorgeous}
%     \end{transcript*}
%     \caption{But, Google Schnauzer, right?}\label{frag:empirical pub findings newinfo}
% \end{fragment}

% This episode presents a straightforward example of mobile device use occasioning in and through the conversation, in which the speaker allocates the device use to another member.
% Dayna, who in the conversation had previously stated that she likes big dogs, enquires about the size of Cally's favourite dog breed (which is later revealed to be the Miniature Schnauzer).
% Cally seems to glance around at her bag before turning back to face Dayna and then acknowledging Dayna's remark about big dogs before directly instructing her to \QF*[08--09]{google schnauzer, right?}.
% In review, this conversation both demonstrates the occasioning of the mobile device use for purposes to research information, and that turn allocation of mobile device use is not restricted to self-selection.
% While Dayna was willing to use her phone, Cally identified herself as someone who uses her phone less often than Dayna, potentially contributing to the factor of allocating the device use to Dayna.



% *********************************************************************************************************************



% \subsection{Sustaining mobile device use within conversation}\label{sec:empirical pub findings sustaining}
% Members routinely engage in the activity of \textit{sustaining} their co-presence within a social setting while interacting with a mobile device.
% Given the nature of the gathering, individuals may try to maintain a level of interaction with others in the collocated group, however, the form this interaction takes varies, given that the individual must also balance their focus with their mobile device and the demands it places on their attentional orientation~\citep{Oulasvirta2005}.

% %As part of this task, individuals may try to maintain a level of interaction with others in the collocated group, given the nature of the gathering but the form this interaction takes varies given the shared focus of the individual.

% Two primary foci were observed that this sustaining activity could take: managing one's relation to and within the social situation and co-managing the interaction with the mobile device within the situation.
% The social norms that govern acceptable behaviour for the current setting (i.e. the pub) provide the framework for members to manage both of these relationships~\citep{Su2015}.
% The analysis identified two methods through which members sustained their mobile device use and co-presence within the collocated group: the first, demonstrating continual social interaction, is where members work to maintain a presence (to a varying degree) within the conversation; and the second, performing accountable actions, is where members make their device interaction both observable and reportable to the other members of the setting.

% %Although we can not determine as to how members chose to manage this situation through the observations, it can, however, define the extreme forms this management could take.


% % *********************************************************************************************************************


% \subsubsection{Demonstrating continual social interaction}\label{sec:empirical pub findings sustaining demonstrating}
% The first method observed adopted by individuals to sustain their device interactions is to account for the use while also actively and accountably doing interactional work in and through the conversation.
% The episode in  \autoref{frag:empirical pub findings nudge} briefly presents a small part of a conversation between Jayne and Z\"{o}e, as Z\"{o}e is looking for a photo on Facebook on her mobile phone.
% In this episode, Jayne is observed talking to Z\"{o}e about the photo during this task, with Z\"{o}e's gaze remaining fixed on her phone.

% \begin{fragment}
%     \begin{transcript*}
%         \im 0   {Graphics/3-1-Empirical-Pub/FragmentNudge-1.png}
%         \by JAY {there's an ama\emph{zing} <well no } \\
%         \by     {no no> there's an amazing} \\
%         \by     {photo of dunno if you've got} \\
%         \by     {it actually of of erm (tt)} \\
%         \by     {(0.5) malcolm with \emph{richard}} \\%{Jayne remains looking at Z\"{o}e, who is looking down at her phone}
%         \later  {2.1} \\
%         \by JAY {you know richard from my year?} \\
%         \im 1   {Graphics/3-1-Empirical-Pub/FragmentNudge-2.png}
%         \by ZOE {°(hm::)°\vspace{2.1cm}} \\
%     \end{transcript*}
%     \caption{You know, Richard from my year?}\label{frag:empirical pub findings nudge}
% \end{fragment}

% During this episode, which only lasts a few seconds, Jayne makes her comment as she looks towards Z\"{o}e, but does not receive a response.
% She follows this up with a question: \QF[07]{you know richard from my year?}, to which she receives a subtle and succinct acknowledgement of her comment (\QFt[08]{°(hm::)°}).
% This response is typical of a perfunctory response members give when they are preoccupied with something else.
% As Z\"{o}e provides her response, she maintains her posture of keeping her head down and her gaze fixed on her mobile phone.
% However, Z\"{o}e's response may be seen to indicate that she is still listening by the accountable nature in which she verbally, albeit minimally, responds to the information given to her.
% This episode is an example of work by a member to maintain their presence within a conversation, even at a minimal level, while also continuing to use a mobile device.



% % *********************************************************************************************************************



% % \subsubsection{Accounting for device use}\label{sec:empirical pub findings sustaining accounting}
% % Another methodically accountable feature transpires when individuals made use of interactional resources, such as body orientation, gaze, and talk to highlight their device interactions by making it observable-reportable to those in the setting.
% % One common observed approach is to continually articulate one's actions to the group, thereby offering a verbal account for the device use, be it in a specific detail such as verbalising what you are typing, or an abstract definition of the task you are attempting.
% % An exemplar of how individuals account for their actions is in the fragment that follows, featuring Lawrence, Jayne, Z\"{o}e and Malcolm again.
% % In the episode, given in  \autoref{frag:empirical pub findings contest}, the group are discussing shorthand notation.
% % Lawrence has expressed an interest in learning shorthand notation, and Jayne explains that she had previously been taught one form of notation although also discusses another form of which she is aware.

% % \begin{fragment}
% %     \begin{transcript*}
% %         \im 1   {Graphics/3-1-Empirical-Pub/FragmentShorthand-1.png}
% %         \by LAW {isn't it mainly phonetic?} \\
% %         \by JAY {it's like:\vspace*{1.6cm}} \\
% %         \later  {3.2} \\
% %         \im 1   {Graphics/3-1-Empirical-Pub/FragmentShorthand-2.png}
% %         \by JAY {there's various versions so} \\
% %         \by     {the one she tried to teach me} \\
% %         \by     {first so i could start going} \\
% %         \by     {is missing out all the vowels} \\
% %         %\by     L {((briefly looks at Jayne while picking up his phone, he then begins to use his phone once he has it in his hands))}
% %         \by LAW {\intDown{}yeah\vspace*{0.2cm}} \\
% %         \im 1   {Graphics/3-1-Empirical-Pub/FragmentShorthand-4.png}
% %         \by JAY {and once you get good at that} \\
% %         \by     {you just write a lot quicker} \\
% %         \by     {(0.7) but then she had one} \\
% %         \by     {which was literally like (.)=} \\*
% %         \by LAW {\intDown{}yeah} \\
% %         \by JAY {=swiggles and just didn't look} \\
% %         \by     {like anything and i don't know} \\
% %         \by     {if that's phonetic or::::} \\
% %         \by MAL {(\qquad \qquad \qquad \qquad )} \\
% %         \im 1   {Graphics/3-1-Empirical-Pub/FragmentShorthand-5.png}
% %         \by LAW {hang on! schuh::::::::ort (.....) hand\vspace*{2.4cm}}
% %     \end{transcript*}
% %     \caption{Hang on! Shorrtttthand}\label{frag:empirical pub findings contest}
% % \end{fragment}

% % This episode presents both methods of sustainment: Lawrence demonstrates his continual social presence when he responds to Jayne's explanation of the first shorthand notation that she learnt with his remark \QF[13]{\intDown{}yeah}, and later the provision of an account in which Lawrence begins to spell out the word \textit{shorthand} while typing on his phone (lines 18---19).
% % Although he does not provide visual confirmation to the other members, or indeed, specific explanation of what his exact interaction is with his device, the clear implication of his spelling out the phonetic sounds is that he is presently typing the word into a search engine.
% % A later conversation confirmed that he was using a search engine to find information about shorthand.
% % Unfortunately, while Lawrence demonstrates that he is maintaining awareness of the verbal aspects of the conversation, he fails to respond to gestures used by Jayne to exemplify her explanations.
% % Jayne makes two visual contributions to the interaction, first demonstrating the form of shorthand in the air (as seen in the second image in the fragment) and then on the table as she explains the second form of shorthand.

% % In addition to articulating actions, the process of providing a continual account of actions can include more visual aspects such as making the device screen visible (i.e. available for glancing).
% % This could be to allow others to engage as a spectator during the device use, or alternatively to engage in a collaborative task that allows one or more other members of the setting to contribute.
% % Both of these behaviours are documented as members were searching for information to contribute, or to corroborate members' opinions.



% % *********************************************************************************************************************



% \subsection{Disengaging from mobile device use}\label{sec:empirical pub findings disengage}
% The disengaging of mobile devices can itself be brought about by a number of different factors pertaining to the conversation and the mobile device use (e.g. search task completion).
% External factors that are not related to the conversation or the mobile device may play a part in this disengaging, although these have not been examined in this research.
% Additionally, a process of ``re-entering the conversation'' may follow disengaging from mobile device use.
% Disengagement from mobile device use concludes the work of sustaining `concurrent' device interaction and conversation (as outlined in the previous section).
% This work further identified that the disengagement of mobile device use can also be either temporary or semi-permanent in nature, where temporary disengagement is defined as where a task is still ongoing, but the user halts their mobile device interaction, and semi-permanence to be the completion, or failure to complete, a particular task.
% Typically, this work uses the term semi-permanence in the latter case as an individual may later use their phone for some other cause, irrespective to the outcome of the previous interaction.



% % *********************************************************************************************************************



% \subsubsection{Disengaging in and through conversation}\label{sec:empirical pub findings disengage innt}
% A number of episodes have already been presented in which mobile device use is occasioned in and through the conversation, both in terms of resolving debates or enhancing one's explanation or viewpoint.
% When looking at disengagement, relevant situational features include the interaction before and after the disengagement occurs, including how the member `rejoins' the conversation.
% Mobile devices place a demand on an individual's attentional resources, and shifting focus from one task to another is likely to be problematic.
% For example, from the temporary disengagement highlighted in  \autoref{frag:empirical pub findings email}, when Lawrence asks \QF[17]{what?}, he articulates that he was unable to maintain a full awareness of the conversation while using his mobile phone.
% In this case, however, Lawrence quickly resumes use and opts not to engage in the conversation, but his initial question emphasises that time spent engaged with the mobile device can impact awareness of the conversation at hand.

% One example of disengaging from mobile device use through the evolution of conversation is presented next, focusing on the point at  which the member suspends their mobile device use to reorient their focus back to the conversation.
% In this episode, given in  \autoref{frag:empirical pub findings distraction}, a third group of friends is observed: Leonard, Christine, Janice, and the researcher.
% The participants have only recently arrived for the study, and although they have been through the process of informed consent, the researcher uses this opportunity to recap the information that was given to participants in the initial email that was used to promote the study.
% Before the included transcript begins, Leonard caught a glimpse of the researcher's mobile phone and clarifies the specific model of phone he owns.
% As the group are joined, Leonard, who is holding his phone, moves to hold it next to the researcher's phone as a visual comparison.

% \begin{fragment}
%     \begin{transcript}
%         \by LEO {yeah, it's like an ipad (.) er (.) a small ipad mini} \\
%         \by RES {there's no need to have both (.) is there (.) em (.) no (.) um} \\
%         \by LEO {no compare ((laughs and holds phone next to the iphone six plus))} \\
%         \by LEO {it's quite big! ((as he stops holding the phones together, begins} \\
%         \by     {to use his own phone))} \\
%         \by CAL {yes::} \\
%         \by LEO {°yeah it's (\qquad \qquad \qquad ) lot bigger°} \\
%         \later  {\ldots} \\
%         %\im    0 {Graphics/3-1-Empirical-Pub/FragmentCollabSearch-2.png}
%         \by RES {so yeah (.) the study is basically focusing on peoples behaviours} \\
%         \by     {(.) their interactions around mobile phones and- ((LEO who is} \\
%         \by     {holding his phone, but looking at RES))} \\
%     \end{transcript}
%     \caption{It's like a an iPad, a small iPad mini}\label{frag:empirical pub findings distraction}
% \end{fragment}

% The change in conversation topic by the researcher leads to Leonard sitting back in his chair and temporarily halting his mobile device use.
% As the researcher starts talking, Leonard is observed placing his (unlocked) phone down on the table, face up, and shifting his gaze towards the researcher.
% This may be seen to display an acknowledgement of the importance of the social (research) setting within which he finds himself.
% However, it is noteworthy to highlight that the action of leaving the mobile device unlocked may declare his intention to resume use later, and perhaps it points to the untimely nature of the topic change that interrupted the task he was attempting to complete.

% Furthermore, it was cursorily observed that conversations could lead to disengagement as other co-present members can disrupt individuals using their mobile devices through talk-in-interaction.
% This is possibly as a result of the conversion providing greater demands on the members' attention, thus leading to the disengagement of the mobile device use.
% In effect, in the words of \citet{Goffman1968},  the conversation is demonstrably the dominant involvement of the member.



% % *********************************************************************************************************************



% \subsubsection{Disengaging from the device as a result of task completion}\label{sec:empirical pub findings task}
% Finally, the last fragment rejoins Cally and Dayna following on from the episode in  \autoref{frag:empirical pub findings newinfo}, this time as an example of where the purpose for which the device use was occasioned has been satisfied.
% In the previous episode with the group, Dayna used her phone to provide an enhancement to Cally's explanation of her favourite dog breed.
% The continuation of the episode, given in  \autoref{frag:empirical pub findings google}, takes place roughly 11s after the fragment.
% In the time between the fragments, Cally had briefly engaged with the main conversation before returning to help coordinate the search task with Dayna.

% Instead of simply leaving Dayna to complete the task alone, the mobile device becomes an artefact, or \textit{resource}, embedded within the conversation between the pair, and a collaborative search task forms.
% Using terminology by \citet{Brown2015}, Dayna could be described as ``driving'' the search task with Cally as a ``passenger'', or \textit{back-seat driver}, providing support to Dayna throughout.
% This reveals that both members continually take turns to engage with each other, while Dayna acts as the operator of the mobile device as the pair work together to complete the task.
% This episode contains a number of notable observations that corroborate findings in collaborative search literature, as Dayna is observed re-orienting the mobile display towards Cally, and Cally re-positions herself to engage with the mobile search task~\citep{Brown2015}.
% This task provides an exemplar of cooperation between members and using the former analogy, could be said to be \textit{co-steered} by both members.
% However, the use is cut short and the episode ends with an articulated apology by Dayna: \QF*[12--13]{°my internet is rubbish so this may take some time°}.
% In this utterance, Dayna removes her mobile device from the conversation in which it had become embedded because of the slowness of her mobile Internet.
% Although technological progress continually improves the responsiveness of user interfaces and device connectivity, issues still persist in situations where a mobile device is embedded with a conversation.
% This may be due to the pace of talk---conversations can quickly move on, especially when more than two individuals are co-present and engaged.% This notion is supported by noting that observed pauses in conversation were rarely greater than 2s.

% \begin{fragment}
%     \begin{transcript*}
%         \by CAL {so it's miniature schnauzer} \\
%         \by DAY {how do you?=} \\
%         \im 1   {Graphics/3-1-Empirical-Pub/FragmentCollabSearch-2.png}
%         \by CAL {~~~~~~~~~~~~=erm::} \\
%         \by DAY {(sccchhhh) (tea) (ee) (ar)} \\
%         \by CAL {oh schnauzer (.)} \\
%         \by     {it's s-c-h-n-a-u-z- n-a-u-} \\
%         \by     {(2.2) schnauzer} \\
%         \by DAY {oh, sch\emph{nauz}er\vspace*{.4cm}} \\
%         \im 1   {Graphics/3-1-Empirical-Pub/FragmentCollabSearch-4.png}
%         \by CAL {schnauzer, go look at} \\
%         \by     {schnauzer puppies right\intUp} \\
%         \by     {((continues to look at phone))}\\
%         \by DAY {°my internet is rubbish so} \\
%         \by     {this may take some time°\vspace*{.7cm}} \\
%     \end{transcript*}
%     \caption{Go Look at Schnauzer Puppies, Right?}\label{frag:empirical pub findings google}
% \end{fragment}

% Following this episode, Dayna leaves her phone unlocked in her lap, although a short while after the device automatically locks itself.
% Later on, Dayna unlocks her phone and, following confirmation of the result from Cally, shares the photo with others in the group.
% She then locks the mobile phone and puts it back on her lap, given the task of identifying the dog breed, and reinforcing Cally's opinion, has been completed.
% The entire process from Cally's initial instruction to the demonstration to the group takes 3m 14s, although possibly could have been performed quicker had the mobile device not been temporarily removed from the conversation.

% To recap, the analysis reveals there were two forms through which mobile device use was disengaged from and two exhibits that demonstrate these interactional methods were presented.
% These also highlight the importance of considering how disengagement occurs and the interactional trouble that may follow afterwards.



% *********************************************************************************************************************



% \section{Machinery of interaction}\label{sec:empirical pub moi}
% \resubmission{Updated fragment references to remaining fragments, all existing findings are used in the selected data.}
% The findings above present exhibits that demonstrate the methods through which members employ interactional resources to accomplish the work of \textit{embedding} the use of mobile devices in conversation.
% These findings will now be unpacked to reveal \textit{the machinery of interaction}~\citep{Sacks1984}.
% It is at this point at which the need to transfer, as Sacks put it, from ``our view of `what happened', in particular interaction done by particular people, to a matter of interactions as products of a machinery''.

% In \textbf{\textit{occasioning}} device use, members exhibited two fundamental methods: \textbf{occasioning in and through conversation}, and \textbf{occasioning ostensibly unrelated to conversation}.
% With the former, mobile device use was both a tool to answer questions and resolve disagreements within the group, and as a utility to help reinforce or corroborate a member's point.
% Furthermore, where the occasioning warranted sustained use, mobile device use became embedded in a number of situations in the setting: for example, the mobile device interactions around providing information to the group on Cally's favourite dog breed led to a collaborative search task (see Fragments \ref{frag:empirical pub findings newinfo}).
% In reviewing exhibits of the latter of the two occasioning methods, it was highlighted that unrelated mobile device use itself is accountable to the group, which may be done verbally and/or visually by bringing up the mobile use-related topic in the conversation (see  \autoref{frag:empirical pub findings email}).
% As with the former occasioning method, this form of occasioning succinctly leads to the management of a mobile device allowing for the conversation to re-orient to and sustain the use.
% In both of these situations, the management of the mobile device was dependant on that of the method through which the use became occasioned.

% \textbf{\textit{Sustaining}} mobile device use whilst the conversation continues is done in and through members \textbf{performing actions to continue to display their attention to the conversation} in line with the social norms of the setting and purpose of the gathering by \textbf{interleaving the device use in the conversation} (e.g. \autoref{frag:empirical pub findings newinfo}).
% The findings also show that members visually shared their devices with others, be it either by \textbf{making the device use visible} (e.g. \autoref{frag:empirical pub findings email}) or \textbf{making their actions audibly accountable} (e.g.  \autoref{frag:empirical pub findings contest}).
% There were also a number of ways members \textbf{demonstrate attentiveness to the conversation} while using a mobile device, such as by looking and glancing at others or by responding to `verbal nudges' (e.g.  \autoref{frag:empirical pub findings contest}).

% \textbf{\textit{Disengaging}} from mobile devices could also occur through a number of different methods, \textbf{disengaging in and through conversation   } on the one hand, or by \textbf{satisfying the purpose of mobile phone use} on the other.
% For example, if the mobile device was occasioned for a particular purpose, then once the purpose of its use has been met, the need for the mobile device may be lost and use is halted (e.g.  \autoref{frag:empirical pub findings newinfo}).
% The way in which disengagement is achieved depends on how the mobile device use was occasioned and sustained, as well as the present social situation.
% Furthermore, mobile device disengagement may only be temporary, as was witnessed in a number of cases (e.g.  \autoref{frag:empirical pub findings newinfo}), if the actual operation of completing has not been brought to a close.
% For example, in some cases, the task was ongoing but would take some more time, or in other cases, members would re-orient their focus of attention to the conversation.
% In the cases presented, this could be considered temporary disengagement in and through the conversation.



% % *********************************************************************************************************************



% \section{Discussion}\label{sec:empirical pub discussion}
% The findings from the fieldwork in a casual-social setting have been presented above as vivid exhibits to explicate the interactional methods through which the work of embedding mobile device use in conversation is accomplished.
% Whilst it is a well-known challenge in the \ac{HCI} and \ac{CSCW} community to design new systems that support collaborative interactions within collocated groups, this thesis pivots from this by instead focusing on the detail of the interactional work that members undertake to embed mobile device use while engaged in collocated interactions in everyday life.
%  The work focused on documenting the accountable methods that members performed in occasioning, sustaining, and disengaging from device use.
% A number of related pieces of the academic literature have looked at intentionally designing new coherent user experiences in collocated groups, e.g. \citet{Counts2004, Lucero2012}.
% There is also rhetoric and critique of existing mobile device use as leading to `social isolation' within social settings~\citep{Turkle2011}.
% These findings and the analytic orientation adopted forbid such simplifications both in the analysis and in suggested design solutions\footnote{For example, the findings show members make use of mobile devices collaboratively and work together to embed the device use in conversation.
% Furthermore, the findings suggest that rather than start out with a design exercise to encourage collaboration, focus should be drawn to the nuanced problematic features of the contingent interactions accountably addressed by members of the setting---members already collaborate with mobile device interactions \textit{as-is}.}. Instead, they show the ways in which embedding practices are situationally \textit{context-shaped and context-shaping}~\citep{Goodwin1990}, and \textit{interactionally problematic}, as evidenced by features such as apologies, interruptions, and inattentiveness.



% % % *********************************************************************************************************************



% \subsection{Embedding mobile devices in conversation}\label{sec:empirical pub discussion embed}
% In the analysis, mobile device use was demonstrably occasioned and became embedded in conversations for several purposes, one of which was information seeking, as highlighted by similar literature~\citep{Brown2015, Church2012}.
% Information seeking-related mobile device use was occasioned in and through the conversation, for example, to clarify points or to resolve disagreements.
% Information seeking in collocated settings is arguably a practice enabled by smartphones; a practice that could in future be provided or enhanced by the introduction of differing technologies, such as wearable mobile devices and interactive tables~\citep{Fikkert2009, Lucero2014}.
% Although this work did not examine how people felt about the use of mobile devices in detail, others have discussed the loss of ``authentic banter'' due to the introduction of mobile phones into conversations~\citep{Su2015}.
% However, this work did in fact witness humour arising from topics actually instigated as a result of mobile device use. %Furthermore, although we saw the use of the mobile Internet impact social interaction~\citep{Humphreys2013}, we did also see reparation work by members following a period of use (see  \autoref{frag:email}).

% It was further found that members often took great care to articulate their actions when devices are embedded to sustain their use and social presence, be it through utterances while their gaze was fixed on their device or an announced statement of intent.
% The use of interactional resources and social cues by members of the setting also allowed them to purvey their current focus and task.
% In  \autoref{frag:empirical pub findings google}, body co-orientation accountably displayed participants working together and cooperating~\citep{Sauppe2014}.
% Furthermore, members' orientation towards their mobile device screen whilst visibly typing messages provides a non-verbal, yet observable-reportable account of their actions.

% In addition, this study also echoes the literature on interruptions that have found an impact on members' attentional orientation~\citep{Fischer2011, Hudson2002}.
% There were also several instances in which members responded to notifications; such a `readiness to respond' may perhaps be related to the informal nature of the setting in which the fieldwork was conducted~\citep{Krehl2013}. \citet{Tolmie2008} have suggested that although there is a need to be accountable towards collocated members, there is equally a duty to manage accountabilities to those people are remotely connected.
% This may have helped to develop a contentious situation where individuals feel that they need to constantly respond to the virtual interruptions that permeate their physical surroundings~\citep{Ames2013}.



% % *********************************************************************************************************************



% \subsection{Collaborating in and through mobile device use}\label{sec:empirical pub discussion coop}
% The episodes above highlight distinct examples of collaboration occurring in, around, and through mobile device use.
% Collaborative searching for information has been revealed to be relatively common on mobile devices~\citep{Morris2013}, suggesting that observed practices are routine for collocated individuals.
% Consider the episode given in  \autoref{frag:empirical pub findings newinfo}.
% Prior to this episode, Cally selects Dayna to involve her in an Internet search, by instructing her to \QF[08]{google schnauzer}.
% She does this to help answer the question posed by Dayna about the size of Schnauzers.
% Dayna states she prefers larger dogs, but Cally resorts to searching for pictures to reinforce her view that the dogs are \QF[04]{cute}.
% Dayna, then in the process of searching, relies on Cally for the spelling (\autoref{frag:empirical pub findings google}).
% Cally leans over, and begins to spell the word, and then also instructs Dayna to look for the photos of puppies.

% However, this use that unfolds in this situation is facilitated by the physical positioning of the two members, i.e. that they are both seated next to each other. \citet{Morris2006a}, on a collaborative table-top based search system, remark that the table-sized display ``provides group members with a shared context and focus of attention, conversation, and activity, and is a common paradigm for supporting co-located cooperative work''.
% Mobile devices intrinsically differ because of their diminutive size and ability to be held in positions that conceal the display from others.
% Therefore, the physical adjacency of Cally and Dayna facilitates the collaborative search that ensues and allows Cally to lean in, monitor the progress of Dayna's actions, and guide her use of the device in the search task.

% Conversely, situations where members are not physically adjacent, other members construct questions and interruptions aimed at the device user to gauge their involvement in an interaction, as exemplified in \autoref{frag:empirical pub findings nudge}.
% In this situation, Z\"{o}e is searching for a particular photo referenced by Jayne, but because there is no common focus of attention and context for both members to orient to, the work by Jayne in determining Z\"{o}e's actions, and whether they are cooperating to share a photo with a group, is entirely achieved as articulation work.
% Furthermore, as Z\"{o}e attempts to find the photo referenced, Jayne provides an account to the other members not involved in the device interaction for the reason of this search task: that the photo being searched for is \QF[02,  \autoref{frag:empirical pub findings nudge}]{amazing}.
% In this short stint, Jayne's articulation ensures Z\"{o}e's involvement in a cooperative task and accounts for the use to the other members allowing Z\"{o}e to focus on the task at hand.

% Z\"{o}e later shares the photo with others by rotating the screen, first revealing the photo to Jayne for confirmation and then to others in the group. \citet{Lucero2013} remark that ``when using their mobile phones, people have a tendency to hold their devices with one or two hands, with the screen facing toward them.
% People will usually adopt a particular device position, combined even with a second hand to cover the screen, either to browse private content, such as a confidential email, or to avoid glare''.
% However, the data presented above augments this perspective by revealing that although individuals keep the screen private at times, they do, on occasion, physically share the screen by leaning in, rotating the screen when the contents of the screen are contingent upon realised or to-be-realised conversation and the device use is embedded within the social interaction.

% In summary, the lack of line-of-sight for Z\"{o}e's device interaction transforms Z\"{o}e into a gatekeeper to the interaction---she must maintain a continual social presence for the device interaction to remain embedded, and other members must perform articulation work to collaborate with the device work undertaken.
% In the case of Cally and Dayna, the ability for the device use to be private is diminished by Cally's line-of-sight of the device, although other members still must rely on the observing and engaging in the talk between the two members to interpret the work being undertaken on the device.



% % *********************************************************************************************************************



% \subsection{The problematic nature of embedding mobile device use}\label{sec:empirical pub discussion problematic}
% This thesis does not concern itself with making moral judgements such as whether device use in social settings is socially acceptable, however, it does want to critically examine the embedding practices observed in terms of \textit{interactional trouble} it causes for the co-participants.
% The episodes above reveal that embedding device use with social interaction places continual demands on the member to remain engaged, or at least \textit{display} attentiveness to the conversation while using their device (e.g.  \autoref{frag:empirical pub findings nudge}).
% However, further evidence suggests it may, in fact, be difficult to pay attention to a conversation whilst using the device (e.g. Fragments \ref{frag:empirical pub findings nudge} and \ref{frag:empirical pub findings contest}), in part because using a mobile device requires the user to look and interact with the touchscreen.
% This is further corroborated by literature on divided attention, with performance factors including task difficulty and practice~\citep[p. 38]{Lundgren2008}.  For example, individuals may find it difficult to read and understand information on webpages while also engaging with a conversation.

% %and a number of  multi-tasking and multiple resources that highlight the difficulties with performance %~\citep{Wickens:2008ho}.

% Although this work would not disagree with characterising the embedding practices demonstrated in episodes in which members co-orient to one member's mobile device use in and through the conversation (e.g. Fragments \ref{frag:empirical pub findings email} and \ref{frag:empirical pub findings newinfo}) as successful, this co-orientation does not come without interactional trouble.
% In  \autoref{frag:empirical pub findings email} the device user interrupts the conversation in the topicalisation of his screen contents, and later in \autoref{frag:empirical pub findings google} the co-oriented device use is disengaged from with an apology \QF*[12--13]{°my internet is rubbish°}---and the purpose for which the device use was instigated in the first place (looking up a dog breed) remains (temporarily) unsatisfied.

% In summary, while the trouble observed was subtle, and in all cases repaired swiftly in and through interaction, it may nevertheless be fair to say that embedding practices are \textit{interactionally problematic}.
% Embedding practices frequently feature interruptions, recapitulations of the conversation for members re-joining, displays of attentiveness despite ostensible inattentiveness, and prompts of absent-minded members.
% The analysis above also exemplifies how individuals apologised for their device use, or the slowness of their device, and for bringing device use-related topics into the conversation.

% The interactional problems revealed in this data suggests current mobile device use is perhaps ill-suited to be embedded with social interaction in unproblematic ways.  An attempt to alleviate or reduce demands upon members to undertake potentially problematic practices might explore ways of combining or introducing utilities to reduce the demands upon members to account for their device use.
% For example, devices could broadcast limited information about the on-screen interaction to others nearby or could rely upon the naturally-accountable production of speech for members to complete information-seeking or other input tasks on the device.



% % *********************************************************************************************************************



% \subsection{The role of mobile devices in interactional trouble}\label{sec:empirical pub discussion role}
% While work has been done to support mobile device interactions in collaborative tasks, there still remains an issue with the speed (e.g. `sluggishness') of device use that makes aligning it with the social interaction challenging: conversations ebb and flow, they may get faster and slower, whereas device interactions do not.
% Sluggish device responsiveness and inflexible alignment to conversational pace make it extremely difficult for device use to remain in step with the conversation.
% It arguably increases the potential to disrupt the conversation.
% For example, this work also shows that mobile device use was eventually given up because of the lack of responsiveness (during an Internet search), opening up design challenges around speed and alignment that need to be addressed in order for devices to enable unproblematic embedding in conversations.

% Generally, such interactional troubles are unlikely to be solved with one `solution', just as there is not a single definable `problem', but much smaller, nuanced issues with mobile device interactions in conversation.
% As devices increase in processing power and sensory input, and the ability to offload functionality to the `cloud' increases, it is likely that a number of these issues that contribute to interactional troubles will be solved in the near-term.
% In terms of disruptions, while manufacturers have implemented simple controls to mute device notifications, such actions are not (yet) automated or linked to sensory data.
% The same questions that others have asked about the impact of mobile devices in conversations could be posed here based on the above findings, for example, whether mobile device use is fundamentally detrimental to conversation~\citep{Brown2015}.
% However, as literature has also highlighted the potential of mobile devices to be utilised to avoid social interaction~\citep{Humphreys2010, Srivastava2005}, even instances where device use hinders social interaction may not be entirely construed badly.
% Another factor to consider is that the negative reflections on mobile device use in social settings has been highlighted in literature as being a symptom of ``double-standards'', where members were critical of device users' actions, but also engaged in those same actions themselves~\citep{Ames2013}.
% Furthermore, while some say that individuals use mobile devices to avoid awkward situations, describing the phenomena as social isolation~\citep{Turkle2011}, it has also been noted that many with anxiety disorders or who are shy may shield themselves from unmanageable situations~\citep{Wei2006}.


% % *********************************************************************************************************************


% \section{Summary and outlook}\label{sec:empirical pub summary}
% This study not only provides a basis for understanding how mobile device use is embedded within a multi-party conversation, but also for how the interactional accomplishment of sustaining the device use is achieved when the mobile device relies entirely on line-of-sight and touching the device screen.
% The first, and primary research question was:

% \PrintRQ{1}

% The findings explicated through the analysis reveal how members routinely occasion, sustain, and disengage from mobile device use within their face-to-face conversation.
% The use of devices can become intrinsically embedded, as revealed in the machinery of interaction, despite nuanced interactional troubles evidenced through apologies, interruptions, and inattentiveness.
% These factors, although demonstrable of potential `problems' are attended to by members of the setting---they account for their actions, they prompt each other if necessary, they provide recaps if needed, and so on.
% In essence, although the work of embedding a mobile device in conversation could be seen to be `messy', this mess is part of the fabric through which it is occasioned, and is so too routinely dealt with by all members.

% In line with RQ2, the study further revealed how members were even able to engage in collaborative tasks while searching for information, relying upon two or more members viewing a screen if possible and articulation work:

% \PrintRQ{2}

% When maintaining a shared view of a device screen was not possible due to seating positions, members default to articulation work and rotating the phone for other members to see the screen.
% This revealed the ability to understand and engage with a device interaction and to engage in collaborative action is dependant upon line-of-sight or articulation work.
% In other words, although members attended to a device being used within a conversation, the fact that the device screen is private and invisible to some members precludes involvement in the interaction and positions the device interaction owner as a gatekeeper.

% Study two, in \autoref{ch:empirical cafe}, considers how such interaction unfolds when the device interaction occurs through speech.
% Speech is naturally accountable to those within earshot, removing the ability for the interaction to be private.
% How members cooperate and manage a device interaction in such a setting when the owner of the device has diminishing control over who can inspect the device interaction, and how members attend and orient to this, will become a focus of the work.



% *********************************************************************************************************************
