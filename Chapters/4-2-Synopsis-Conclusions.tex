
%!TEX root = ../PhDThesis.tex



% *********************************************************************************************************************
\chapter{Conclusions}\label{ch:synopsis conclusions}
% *********************************************************************************************************************



\begin{revisedsubmission}
This chapter summarises and brings together the conclusions of the machinery of members' conduct in using a device in and around conversation in a casual-social setting, the collaborative nature of this device use, and the methodological contributions of this thesis.
Finally, this chapter concludes with a summary of the potential future work that could follow on from this thesis.
\end{revisedsubmission}



% *********************************************************************************************************************



\section{Summary}\label{sec:synopsis conclusions summary}
This thesis made a case for studying device use in conversation (see \ref{sec:intro probdef}), adopting a practical approach that eschewed judgements on the values of device use and instead focused on the practical accomplishment of how devices are used during multi-party gatherings in casual-social settings.
\iresubmission*[ER-B: Remove unnecessary narrative regarding good/bad device use]{This case was motivated by a paucity of data on how and for what purpose devices are used in such settings, against a backdrop of existing literature that characterised such use as problematic.}
\iresubmission*[JR-1, JR-3a, ER-1, IR-2: Remove sections discussing and reflecting upon findings related to attention literature and references to EMCA]{The research in this thesis was fundamentally interdisciplinary in its approach and drew upon existing work in Mobile \acf{HCI}, ethnomethodology, and \acf{CSCW}.}

\begin{revisedsubmission}
Through the analytic lens of ethnomethodology, this thesis explicated both the problems that occasioned device use in and through the ongoing multi-party conversation, and how members brought their devices into a conversation to address these problems.
These problems form the foundation for the interactional projects---ranging from information deficits through to desires to play games---occasioning the use of the device as an activity that unfolds within the setting.

This thesis consisted of three studies, each with a focus on the social organisation of groups of people socialising together and making use of a particular technology.
The first study examined conversation around the use of touchscreen portable devices such as smartphones, the second studied conversation around the use of the \acf{VUI} on touchscreen portable devices, and the third studied conversation around the use of screenless \ac{VUI} devices, marketed as smartspeakers.
Each set of studies took place in a specific setting: the first was in a pub, the second in a caf\'{e}, and the third in participants' home, all of which were considered to be `casual-social'.

Through explicating the members' methods of using a device in conversation in these settings, this thesis shows how members accountably organise this use and take part in collaborative activities to complete these interactional projects.
This collaboration was identified as turning upon the natural accountability of device use, which members accomplished through making the specifics of their use visible and reporting upon it in interaction.
When device use was done using a \ac{VUI}, the use itself was made naturally accountable through the talk to the device interleaved amongst the conversation that occasioned it.
In the case of \acp{VUI} that feature no \acf{GUI} (i.e. the smartspeakers), the responses from the device were oriented in the conversations, with members demonstrably attending to---and substantiating the natural accountability of---these responses through the coherence of the initial request and ongoing conversation.
Through the examination and reflection of the data presented in this thesis, the discussed has illuminated how members ostensibly treat the use of devices perspicuously.
Furthermore, it was identified how such device use, rather from detracting from the ongoing conversation, became embedded within the activity of socialising together.
In this sense, the device use did not unfold instead of---or `isolated' from---the social activity, but was performed as an activity within the work of socialising together as a group in a casual-social setting.

These studies inform and further encourage \ac{HCI} work to examine ways of creating technologies to support collaboration on portable and readily-available technologies while people are socialising in groups.
Crucially, by demonstrating the intricate ways in which members accomplish collaborative action to complete interactional projects, drawing upon the resources provided by the \ac{VUI}, this thesis identifies a topic for further examination within both \ac{HCI} and \ac{CSCW}.
Secondly, by exploring members' collaborative actions, this thesis remarks upon moments of when interaction was undertaken with a \ac{VUI} smartspeaker, it was the \ac{VUI}'s responses that reported upon the success---or not---of requests made to the device.
This identified a key concept to consider for future work, reinforcing findings published as a result of the research in this thesis (see \citet{Porcheron2018}), that classify the responses from \acp{VUI} as \textit{resources} for further action as members' problems are addressed.
\end{revisedsubmission}



% *********************************************************************************************************************



\section{Contributions and key conclusions}\label{sec:synopsis conclusions contribs}
This thesis makes three key contributions, corresponding to the research questions posed in \ref{sec:intro rqs}:

\begin{revisedsubmission}
\begin{enumerate}
    \item Explication of the \textit{members' methodical accomplishments} of using a device in casual-social gatherings, detailing how they bring devices into an everyday multi-party conversation, and offering an insight into the differences of how device use is used to address the members' problems that arise in such settings,

    \item Development of the \textit{methodological approach} in this thesis, both in terms of the application of ethnomethodology and of the nature in which \textit{these technologies} were studied \textit{in these settings}, and

    \item \textit{Conceptual insights} of \icorrection{the nature of studies in casual-social settings and `device talk' to \acp{VUI}}.
\end{enumerate}
\end{revisedsubmission}

\noindent These contributions are summarised in the following sections.



% *********************************************************************************************************************



\subsection{Members' methodical accomplishments}\label{sec:synopsis conclusions machinery}
The methodological approach taken with this thesis included capturing `real-world' empirical data of groups of friends or family conversing in a casual-social setting.
\iresubmission{This thesis defined the conceptual choice of setting as one in which friends and family gather to socialise and relax.}
Through the analysis of the captured data, the empirical chapters in this thesis progressively explicated the accountable practices of the members of each setting as they \iresubmission{interleaved} their device use in and through conversation.
The orientation to accountable actions of members in the setting allowed for the formation of thick descriptions that detailed the interactional methods people employed as they attended to the interactional projects for which device use was occasioned.

\iresubmission{Each study identified members' methodical accomplishments that revealed exactly \textit{how} members of the setting brought a device into the ongoing conversation.}
These were brought together in the discussion to reveal how the overall accomplishment of using the three different technologies in conversation is done (see \ref{sec:synopsis discussion conduct device-use}).
\iresubmission{Crucially, the findings reveal how, through using a device in conversation, the natural accountability of members' actions is accomplished as an outcome.
With touchscreen devices, the interactional work to account for device use was done through methods such as articulating what was being done with the device, or making the device screen visible to others.
In cases where the interaction was with a \ac{VUI}, the interaction with the \ac{VUI} (i.e. the \textit{talk to the device}) typically made the specific nature of a member's device use both observable and reportable within the context of conversation within which it was occasioned.
Nevertheless, members accounted for this use by providing further information on what occasioned their use through talk (e.g. they recalled a recent news story), or by providing preparatory accounts for what the interactional project they were about to undertake concerned.
This accountability provided the resources for other collocated members to ostensibly self-select to become involved in the interactional projects, and in turn, at situations where members had technical trouble, collaborate with the device interaction.
In the case of \acp{VUI} on portable devices such as smartphones, in contrast to standalone smartspeakers, the response from the \ac{VUI} was typically returned via the built-in touchscreen.
Through using a device in these settings, members made this response accountable by sharing visibility of the screen, providing verbal reports of responses from the \ac{VUI}, or through further interactions with the \ac{VUI} (e.g. repeating a request), which recognisably establish the case for members reasoning that the device failed to respond to a prior request.}

\begin{revisedsubmission}
Through these three studies, the notion that device use is a mundane phenomenon in the settings is established.
Members' problems, established through the conversation of the setting, occasioned the use of technology---ranging from instances of using a device to find new information in the conversation through to playing a game together---such that the use of the device was \textit{part of} the multi-activity in the setting.
Conversation topics were also brought about as a result of device use, with members introducing an email to the conversation to make a joke, through to projects such as testing the functionality of the \ac{VUI}.
In these instances, again, device use was occurrent alongside---or interleaved with talk---and was brought into the activity of socialising as an ostensibly mundane activity within the setting.
Therefore, this thesis concludes that device use is fundamentally \textit{embedded} in the activity of socialising in a casual-social setting.

Furthermore, as a result of the naturally accountable nature of interacting with a device, dealing with technical troubles and issues in completing device tasks were shown to unfold as collaborative activities.
In this regard, members collaborated on mobile search tasks using touchscreen devices, attempted to find new information with \acp{VUI} on smartphones, and attempted to start games using \ac{VUI} smartspeakers in the home.
In both of the studies involving \acp{VUI}, this collaborative practice unfolded whereby co-present others involved themselves in others' device use \textit{without invitation} or a request for assistance.
This was revealed to turn upon the natural accountability of the use of the \ac{VUI}, which made the specifics of the interaction hearable and reportable within the context of which the use was occasioned.

%\correction{Remove summary of the summary}
% In summary, there are three contributions arising from the machinery members' conduct in a casual-social setting:
% \begin{itemize}
%     \item Device use was occasioned in and through conversation as part and parcel of socialising in a casual-social setting,
%     \item The hearable nature of talk to the \ac{VUI} was made naturally accountable through the use of the device, as occasioned in and through the ongoing conversation, and
%     \item Technical troubles arising from device use were attended to collaboratively, and where the specifics of the interaction were made accountable, members self-selected to involve themselves in these interactional projects.
% \end{itemize}
\end{revisedsubmission}

\begin{corrections}[Clarify the substantive contributions]
Crucially, this thesis challenges existing notions in socio-technical studies that the use of portable devices in such settings is a distraction from ongoing socialising (see \ref{sec:background litreview society together}).
Specifically, this thesis' findings implicate a call for reconsidering how device use is articulated in publications: rather than treat it as incongruous or distinct from social interaction, this thesis emphasises a call for analytic orientations which regard it as accomplished  \textit{in and through} members' efforts to socialise.
Even more pointedly, the findings suggest that device use is part of the \textit{social order} in these settings (see \ref{sec:background approach em work}).
This does not invalidate prior findings by others, as such findings were perhaps indicative of the time in which portable devices were more nascent, compared with the ubiquity of device ownership at the time of these studies (see discussion of device ownership in \ref{sec:intro devices}).
In retrospect, for example, Turkle's more recent calls to `reclaim conversation'~\citep[see p. \pageref{line:reclaimingconv}]{Turkle2015} implicates as much: the work of conversing in these sorts of settings \textit{has changed} as a result of technology use, which now ostensibly permeates it.
However, to regard this change as negative or positive---which is a categorisation based upon morals---is something this thesis intentionally does not speak to.

Secondly, this thesis' work on voice interfaces---both mobile and in the form of smartspeakers---took place as the technologies were coming to mass-market.
Literature that examines their use \textit{in vivo} was nascent and thus this thesis contributes the first empirical account of making such technologies at home.
However, studies of other activities in the making of technology at home bare analogous methods to the use of the smartspeaker.
For example, in the same regard that \textit{multi-screening} became embedded in the work of \textit{leisure time} through sitting on a sofa in front of a television while making use of mobile devices to augment television watching~\citep{Rooksby2015} (see \ref{sec:background litreview society utopia}), smartspeaker devices were placed in \textit{activity centres} in the home---such as the living room or a kitchen/diner---and members made the use of the smartspeaker naturally accountable as part of the interactions in those settings (see p. \pageref{line:activitycentres}).

Finally, the technologies under study in this thesis could still be considered to be single-user---even smartspeakers, which operate on a one-at-a-time person-agnostic operation.
However, research in \textit{mobile collocated interactions} has attempted to address this by exploring ideas for how to embody notions of groupware from \ac{CSCW} in portable devices to support collaboration (see p. \pageref{line:singleuser}).
This thesis specifically contributes the notion of how users' interactions with voice interfaces are transformed into multi-user experiences \textit{by users} through interaction in spite of a lack of specific design cues.
Members does this by accounting for device use and occasioning the device use in conversation.
This was demonstrable in this thesis in many cases, but of note are also situations in which technical troubles with devices were collaboratively dealt with through conversation, especially with \acp{VUI}.
For example, the resources made available through interaction with a smartspeaker---the hearable input and output---facilitate the collaborative actions of collocated users.
The ambitions of the research agenda to design portable technologies for collaborative action could be extended by further exploring the use of voice interfaces, with the findings of this thesis demonstrating the potential of such technologies.
\end{corrections}



% *********************************************************************************************************************



\subsection{Methodological approach}\label{sec:synopsis conclusions method}
\begin{revisedsubmission}
This thesis adopted the methodological perspective of ethnomethodology (see \autoref{ch:background approach}) to unpack members' actions in the settings.
The first two studies took place in the semi-public settings of a pub and caf\'{e}, with participants recruited for the purpose of socialising together, and to be recorded for their interactions.
Participants were informed that the focus of the study was on the behaviours around the use of devices, but that there was no requirement to use any device during the study.
In the case of the second study, participants were asked to use the \ac{VUI} on their smartphone instead of typing where possible, but were again told that this was not a requirement and that the use of the device itself was not required for the study.
These settings were selected as they were perspicuous in regards to the technology being studied, i.e. existing literature and personal experience had already established that technology use unfolds within pubs and caf\'{e} among groups of people socialising.

These two studies were, in some regards, an ``uncontrolled experiment''~\citep[p. 114]{Suchman1985} of sorts, in which people were in the setting for the purposes of being participants in a research study, but that the focus of that study was not on the completion of a specific task or following a protocol.
The researcher was present and socialised with the participants for all the gatherings in these two studies and followed a participant-observer approach throughout the gatherings.
The gatherings took between 60 and 90 minutes, with the study ending at an agreed time with the participants---the uncontrolled nature of the study was such that the study ended where it was ostensibly deemed appropriate by the researcher.
In these two studies, data were collected by video recording the study using fixed wide-angle cameras on tripods, audio recording using a voice recorder, and writing of fieldnotes after the study.

The third study took place in participants' homes and was longitudinal in approach, taking place over one month.
Homes were selected as the site of study because these were the sorts of places that \ac{VUI} smartspeakers were designed for---they are typically non-portable devices designed for use in homes or offices.
A longitudinal approach was also adopted in this study given the recentness of the technology being introduced to market (i.e. less than one month on the UK), with the focus on the study being how the technology was used in everyday multi-party gatherings.
In this regard, each participating household installed the smartspeaker in a communal area of the home.
For data collection, audio capture was selected given the ethical and technical concerns of collecting data in the home (i.e. participant uneasiness about recording video in the home, and the vast amounts of video data that would be generated).
However, a further issue was that it would still remain problematic to continuously record audio in the home (e.g. necessitating an analysis that would be an insurmountable challenge in terms of the length of captured data).
Therefore, an audio recording device was designed and built for this study that captured a minute before and after interactions with the \ac{VUI} only, allowing researchers to make sense of the context within which the device was used (this was called the \acf{CVR}, see \ref{sec:empirical cafe design data cvr}).

Across all three studies, the same analytic orientation was adopted in unpacking the collected data.
Each study adopted an iterative approach, in line with documented practices in ethnomethodology (e.g. \citet{Crabtree2012,Heath2010}), to explicate members' interactional projects, and how they practically interleaved device use within conversation.
Ethnomethodology's focus on the naturally accountable practices of members allows this analysis to present `what is done in the doing', and consequentially support this thesis' underlying goal to reveal how devices are used in multi-party conversation in a casual-social setting.

%\correction{Remove summary of the summary}
% In summary, there are three contributions arising from detailing the members' problems that occasion device use and the machinery of how devices were actually used in a casual-social setting:
% \begin{itemize}
%     \item Recruiting groups of participants to socialise as a group provided rich empirical data to allow for the explication of how technology is accountably used in casual-social settings,
%     \item Selective recording using a purpose-built \ac{CVR} is an effective way of capturing data from a longitudinal study in the home, and
%     \item The ethnomethodological approach taken in this thesis' work has resulted in thick descriptions that report members' naturally accountable actions, and provides the commodity to make sense of and reveal the nature of using technology in a casual-social setting.
% \end{itemize}
\end{revisedsubmission}

\begin{corrections}[Clarify the methodological contributions]
This thesis' contribution in the form of the \ac{CVR} provisions a data capture tool to longitudinally study interaction with a voice-driven technology in the home (see \ref{sec:empirical cafe design data cvr} for the design of the recorder).
Approaches previously adopted to studying technologies in the home include auto-ethnographies, ethnographies, diary studies, log analysis, and interview studies (see p. \pageref{line:prevhomestudy}), with each providing researchers with different levels of insight into the situated use of technology.
This thesis' work in the home is congruent with existing ethnographic approaches and ideas from \ac{CSCW} of placing research technologies in the home~\citep{Tolmie2008a}.
As elaborated on by \citet{Crabtree2003}, the challenge for designing technologies for the home is ``how people live in the home, what they do when they are at home, and the potential role of technologies within the milieu of domestic activities''~\citep{Crabtree2003}; through its selective capture of interaction and the preceding/succeeding use, the \ac{CVR} allows researchers to elicit such an insight with relative ease.

By building and deploying a technology to automatically selectively capture interaction with a smartspeaker over an extended period of time that incorporated ethical considerations of long-term data capture in a personal space, this thesis supports an ethnomethodological analysis by providing rich data that includes elements of the context of interaction.
Other approaches, such as having a fieldworker `on site', can become impractical when studying interaction for extended periods of time, with proposed solutions including asking participants to record their interaction (e.g. \citet{Rooksby2015}).
Alternatives include continuous video capture, as done by \citet{Heath1991} in their study in workplace collaboration, but such work was undertaken among colleagues and under a vastly different regulatory environment.
This thesis took an automated approach that allowed for the collection of data with some semblance of context through the inclusion of data around device interaction.

The richness of the resulting data included in this thesis validates the approach for further studies of ubiquitous computing in the home, and perhaps other sensitive locations such as the workplace.
Such approaches need not be restricted to voice interfaces, but could include other technologies too, detected through properties such as increased electricity consumption.
As ubiquitous computing research continues to examine the \acf{IoT}\footnote{An umbrella term for the routine integration of Internet connectivity to everyday technologies, including sensors and home fittings.}, approaches such as the automated data capture in this thesis are likely to become of even greater benefit.
\ac{IoT} technologies are increasingly incorporating elements of autonomy and portability~\citep{Fuentes2019a,Porcheron2015}, raising challenges for the study of their use: understanding just how people deal with this autonomy in the home will be critical to ensuring systems meet the needs of users~\citep{Crabtree2003}.
%Other research technologies could incorporate selective data capture, or augment existing `\ac{IoT} data hub' concepts (e.g \citet{Crabtree2018}) to support examination of \ac{IoT} use \textit{in vivo}.
\end{corrections}



% *********************************************************************************************************************



\subsection{Conceptual insights}\label{sec:synopsis conclusions concept}
\begin{revisedsubmission}
The final point to address is conceptual insights arising from this thesis.
The insights contribute to existing ideas and work in both \ac{HCI} and \ac{CSCW}.
There are two key concepts this thesis \icorrection{focuses on: the concept of the activity-based settings, and the concept of `conversation' with a \ac{VUI}}.
\end{revisedsubmission}

\begin{corrections}[Refocus the conceptual contributions]
\subsubsection{Casual-social settings}\label{sec:synopsis conclusions concept casual-social}
This thesis set out to study people socialising together in groups in what was categorised as a `casual-social setting'.
This sort of setting is referentially mentioned in a range of existing academic literature (see p. \pageref{line:casualsocial}), as a place for people to gather and socialise together, but rarely is it designated as a site for empirical investigation.
In seeking to clarify this type of setting, this thesis incorporated notions of other `sorts' of settings, such as third places~\citep{Oldenburg1989}.
This provided the backdrop that established the concept of a casual-social setting as one in which people gather to relax and socialise, and that may be public or private.
This definition encompassed each of the three study locations in this thesis: a pub, a caf\'{e}, and a communal area in the home; each of which were \textit{perspicuous} to the device use under study~\citep{Garfinkel1992}.
The analysis in this thesis showed how interaction in each of these settings is replete with articulation work to naturally account for and interleave the use of devices within conversation, and that device use is a recurrent activity as part of socialising in these settings.

As far back as the early 1990s, there have been calls for \ac{CSCW} to examine social settings to inform technology design~\citep{Grudin1990}.
What this thesis does, through this study in these three different settings, is reinforce and renew such a call for studying technology use in places where people are collocated: technology use is replete in such settings and there is a ripe opportunity for \ac{CSCW} and \ac{HCI} to understand and design technologies to meet members' needs.
This should crucially take place outside of workplaces and homes as distinct settings, and instead be situated in a range of settings defined by the interactional phenomena that is of interest.

\citet{Ellis1991} call the use of technology in such settings ``same-time/same-place'', distinguishing it from interaction that is asynchronous or consisting of remote communication.
Research of such technologies has diminished in \ac{CSCW} since the early 1990s, although \citet{Fischer2016} attempt to rekindle this, making the call for studies of ubiquitous technology use in new places.
In this spirit, this thesis proffers the concept of the casual-social setting for further research, as one with which there is nascent understanding of newer technologies and their use within the work of socialising, and one which should be unpacked in future work.

Finally, this thesis identified how conversation in each setting was occasioned for related interactional projects using the same methods as part of the work of socialising together.
%Device use was established as part of the for conversation within \textit{casual-social settings}, and within the activity of socialising in a group, rather than be tied to any specific place.
The use of the devices was done so as part of the already \textit{established social order} of the setting and regulated as such by those members who were co-present through ``whatever organisation''~\citep[pp. 548--549]{Sacks1992b} the world in which the devices inhabited.
People were shown to make use of their devices\----smartphones and smartspeakers---in and through the existing organisation of their lives; as \citet{Sacks1992b} argue, the devices ``[were] made at home with the rest of world [in so much that the introduction of] each new [device] becomes the occasion for seeing again what we can see anywhere''~\citep[pp. 548--549]{Sacks1992b}.
In other words, this thesis demonstrates how the use of these devices is brought into the already organised work of the casual-social setting as part of the already established order of socialising.
\end{corrections}




\begin{corrections}[Clarify the second conceptual contribution]
\subsubsection{Conversation with VUIs}\label{sec:synopsis conclusions concept vui}
The second conceptual contribution offered by this thesis relates to \ac{VUI} use and the question as to whether interaction with a VUI is indeed ``conversation''.
Above in \ref{sec:synopsis conclusions concept casual-social} the notion of the device use being made at home was established through members bringing their devices into the work of the highly organised world.
With the use of \acp{VUI} in particular, interaction was shown to consist of the phenomena of members `talking to devices' through an ongoing conversation with others (see Chapters \ref{ch:empirical cafe} and \ref{ch:empirical home}).
The devices superficially `talk' back to the user and thus at a cursory glance present the illusion that there is indeed an exchange or `conversation' between human and machine.
Adding to the narrative that such use is a conversation is the plethora of literature describing such interfaces as ``conversational'' (e.g. \citet{McTear2016}'s work on `The Conversational Interface').
To examine this claim, this section brings together this thesis' findings to add a perspective grounded in empirical data to the debate, first unpacking the notion of `talk to devices', and then that of `talk by the devices' to reflect upon the concept of `having a conversion with a \ac{VUI}'.

On the use of the telephone, \citet{Sacks1992b} explicates the characteristics of the opening of a telephone call and in doing so reveals that despite the introduction of a new technology, human interlocutors employed the existing methods, routines, and established social order to converse.
Indeed, talk to \acp{VUI} has been shown to consist of specific characteristics in this thesis.
Whereas, conversation typically unfolds with a minimisation of overlapping talk and gaps between speakers' turns~\citep[pp. 704--706]{Sacks1974}, the design of a \ac{VUI}---that it listens from the utterance of a wake word through to a pause in talk---necessitates the production of gaps to delineate the completion of requests from `other talk'\footnote{This echoes \citet{Suchman2006}'s ideas of ``shared understanding'' between device and human, with both exhibiting different ``respective views of the interaction''~\citep[pp. 123--124]{Suchman2006}. In this case, a \ac{VUI} has a different respective view, constrained by only `understanding' that a request is completed by a drop in the relative ambient volume.}.
In the study of a simulated \ac{VUI}, \citet{Wooffitt1994} similarly remarked upon the ``comparatively lengthy silences between system turns''~\citep[p. 104]{Wooffitt1994}.
These system designs are predicated upon one-at-a-time interaction and a typical voice transcription system cannot distinguish between multiple concurrent voices.
As a result of this limitation, talk to \acp{VUI} consists of little overlap, yet is trailed by a gap which signifies the end of the request to the device\footnote{Conversely, however, such a design feature was also shown to be exploitable to disrupt another's device use by talking over the request (see \ref{sec:empirical home findings game preinit}).}.
In addition to this is the notable lapse in conversation both while a request is made and following a request until a response is produced by the \ac{VUI}, or it is established by members that no response is forthcoming.
Such actions are, in essence, demonstrable of others allowing for the user to `get the device to work'.
In this, the use of the \ac{VUI} is shown to be a methodical accomplishment, done through the user talking to the device with the adoption of certain characteristics to get \textit{a} desired output from the device

Now, consider the notion of how \ac{VUI} interaction idealistically proceeds, i.e. through the input of a user speaking some request, and a synthesised voice delivers a response as output.
Such an ideal easily affords the notion of referring to such interaction as conversational.
There are two parts to this interaction with the \ac{VUI}: the input and the output, and in combination they ostensibly map to a subset of formalised abstractions of talk (e.g. of a \textit{command/action} or a \textit{question/answer}).
However, merely exhibiting a similarity to conversation in structure does not make it a conversation~\citep{Button1995}.

With regard to the input to the \ac{VUI}, the system itself has no comprehension of what a complete utterance is beyond a  break in the stream of words and designates all that came before this break as a \textit{request}.
Equally, the device has no notion of social order or grounding of the context of the action that occasioned its use, instead merely operating on a series of words punctuated by a drop in the volume.
As \citet{Button1995} remark, the \ac{VUI} would have to ``have capacity to conduct the social actions constituent of particular activities'' to understand the conversation, however the \ac{VUI} reduces the situated action to that of a recording of an audio stream.
The device then transcribes this captured audio to a series of words and processes it according to a series of pre-programmed rules\footnote{This is evident in the way user manuals present to users the possible ways in which a \ac{VUI} can respond. The \ac{VUI} cannot deal with what it has not been programmed to deal with.}, in turn abstracting away the situatedness of the action with which the input was produced.
In this sense, the pipeline with which \acp{VUI} operate reduces the social action of producing input to a mere textual representation of that action devoid of its context\footnote{You could argue that a human can parse the context of a letter or an SMS and thus words do carry meaning even in written form. \acp{VUI} (at least of today) do not possess the capability to do this, perhaps for a combination of ethical, technical, and legal reasons.}.
This processing is based upon ideas of talk being formalizable to a series of preconfigured rules, whereas the basis of conversation analysis attests to its unformalizability~\citep{Sacks1974}.
With this in mind, it could be argued that for various reasons beyond the scope of this thesis, the design of the \acp{VUI} are \textit{reductionist} in their treatment of input.

In terms of the response delivered by the \ac{VUI}, there too are particulars that fracture the notion of the `conversational' interface.
It is, for benefit of the user in the `here and now', an automated outcome of their actions (which were, as above, shown to be embedded among the established social order).
There is no claim that a lamp engages in conversation when the switch is flicked by a person, for example.
The light may come on, or it may not, and this outcome may depend upon any manner of technical or social reasons (e.g. the wrong switch may have been flicked).
Yet, the seeming complexity, variability, and form through which the response from the \ac{VUI} is made proffers the idea that it is a `conversational' exchange.
A rebuttal against this simplistic argument would be to say that response from a \ac{VUI} differs from request to request for no `obvious' reason (e.g. technical error, variation, software upgrades), it is mostly non-deterministic unlike the lamp and switch, and that it may be imbued with some context (e.g. information about current affairs, sports results, the state of smart home technologies).
Crucially, however, the response from a \ac{VUI} is indiscriminate to the \textit{context within which the response is delivered}.

Therefore, the non-determinism of the \ac{VUI}'s response comes not from locally-produced interaction with which the device finds itself, but rather from pre-designed features.
It is, in this regard, ultimately not conversational as the response from the device adorns none of the qualities one would expect in conversation: the device does not manage its response among the social order, and even more so is `ignorant' of the social action which occasioned the response in the first place.
The work of `attending' to this response is done through which the routine work of the members of the setting in which the device finds itself.
As with the lamp analogy above, the person who turns on the light makes it naturally accountable, and with the \ac{VUI} so too does the person who makes the \ac{VUI} request.
The device does not converse, but rather audibly simulates talk as a direct consequence of human action alone, it takes no account of how that simulation unfolds in the setting---it does not make its actions naturally accountable, nor does it bare such responsibility as part of the social order.

As such, it can be said that the audible response from the device provides a veneer of ``conversation'', but that such a notion is merely a simulation of conversation.
The user talks to the device, but the device cannot converse. As \citet{Button1995} remark:
\begin{quote}
     [Despite] the fact that one may be able to reproduce, on the computer, many sequences of conversation that formally resemble the sequences of conversation, which may indeed be formally indistinguishable from them, [it] does not demonstrate that one has thereby enabled a computer to converse in the way that human beings do.
    \quoteauthor{\citet[p. 111]{Button1995}}
\end{quote}
In this, what becomes evident is that it matters not if the device seemingly produces a response to a question, or follows an instruction with an action, but that the device itself is not conversing because it rests on the false basis of formalised pre-configured conversation, rather than the notion of conversation as locally-produced situated action by the interlocutors.

In reflecting on this state, the lack of the \ac{VUI}'s `competence' in engaging in conversation can be seen as what leads to members of each setting making the device `at home'.
Furthermore, through exploring the concept of `\ac{VUI} use as a conversation', the parallels to \citet{Suchman1985}'s work on the use of an agent-based photocopier interface (see p. \pageref{line:suchman}) become even more evident.
Just as with the photocopier, \acp{VUI} possess only a limited sensitivity of the interaction that unfolds, and it is through this that the common sense basis upon which people approach interaction with the interface is revealed (i.e. the common sense basis in this case knowing how to do talk).
In this regard, there is a mismatch between the very idea of a `conversational interface' and the mundane competences implicated in doing conversation.
The incompetence of systems to attend to the matters of conversation as humans do implicates the user of the \ac{VUI} into resolving these misaligned competences.
Through this, the user makes the \ac{VUI} `at home' through the methods they make any technology at home.
%The quality of the devices' simulation of conversation is not critical to the use of such devices being embedded in the home.
\end{corrections}



% In summary, there are three conceptual contributions arising from the study design and analysis of this thesis:
% \begin{itemize}
%     \item Device use is ostensibly treated as mundane and recurrent in social gatherings in \textit{casual-social settings}, underscoring some existing calls for \ac{CSCW} to further examine social settings for how technology design could be shaped by people's conduct,
%     \item Members undertake \textit{collaborative action} to complete their interactional projects, and this turns upon the natural accountability of action in the setting, and
%     \item The responses from \acp{VUI} are conceptually \textit{resources for further action} with the \ac{VUI}, by both the original \ac{VUI} user and others who are co-present.
% \end{itemize}
%\end{revisedsubmission}



% *********************************************************************************************************************



\section{Critical reflection}\label{sec:synopsis conclusions reflection}
\begin{revisedsubmission}
The research questions posed in the thesis were developed out of personal intrigue into the sources of the consternation amongst popular press and academic literature of devices being used in face-to-face gatherings.
This thesis drew upon three distinct fields of work: 1) Mobile \ac{HCI} literature on creating and studying technologies for `mobile collocated interactions' (see \ref{sec:background litreview design mobilehci}), 2) close studies of interaction from \ac{CSCW} (see \ref{sec:background litreview f2f}), and 3) the ethnomethodological approach to ethnography (see \ref{sec:background approach em}).
This combination of fields provided the backdrop to the studies in this thesis, with the aim of each study being to examine and explicate the interactional projects of members using a device in a casual-social setting.

The three studies in this thesis were undertaken as independent pieces of work but successive pieces of work, with the second and third motivated by the previous one.
In this regard, these studies are examinations of different technologies but with shared characteristics between them, i.e. the first two studies involve the use of portable devices such as smartphones, and the latter two use \acp{VUI} on a smartphone and a smartspeaker respectively.
Perhaps unsurprisingly, the first two studies both feature the device being used for the same type of interactional project, i.e. to introduce new information to the conversation, and the latter two studies both identified members establishing the capability of the \ac{VUI} through its use.

Although the critique of technology use in such settings provided the motivational backdrop for this thesis, the work itself speaks little to many of the arguments raised.
In part, this is because much of this criticism relies upon \textit{a posteriori} methods, and includes people's reflections upon device use, rather than an examination of their device use \textit{in vivo}.
This thesis does little to challenge specific critiques of device use (e.g. of isolation~\citep{Turkle2011}) because this thesis can only speak to the naturally accountable methods upon which members use a device in the setting and not members' perspectives or feelings regarding device use (unless, of course, they reported these as part of interaction in the studies).
Nevertheless, through the approach adopted, what this thesis does accomplish is to show that people account for device use in casual-social settings, and embed it as a constituent activity of socialising together as a group.
Device use was shown to be \textit{regulated as an activity in socialising together}, as an ostensibly mundane feature of that interaction, and not in spite of it, and at least rubbing up against the critiques of isolation.
\end{revisedsubmission}



% *********************************************************************************************************************



\subsection{Limitations of this work}\label{sec:synopsis conclusions reflection limitations}
The interdisciplinary nature of this thesis has resulted in a number of challenges as well as some limitations with the work in this thesis.
This section discusses two key limitations of this work:

\begin{enumerate}
    \item The \textit{limitations of a small n} (i.e. participant/`sample' size) in the empirical studies, and

    % \item The complexity of including an \textit{interdisciplinary perspective with EMCA} work

    \item The \textit{discussion and linking of findings to design} are untested implications based on reflection of the empirical data.
\end{enumerate}

\paragraph{Limitations of a Small \textit{N}} \hfill \\
The study of people socialising and using devices was based on numerous small-scale studies of specific groups of participants.
\iresubmission{This thesis' adoption of an ethnomethodological approach, focused on the naturally accountable actions of members, produced thick descriptions of members' actions that would be recognisable to anyone with a vulgar competency in the setting's work (see \ref{sec:background approach em comptency}).}
In this sense, this thesis did not rely on \textit{interpretation} to construct `scenic' descriptions of people interactions, but a praxeological account of member's interactional work.
However, the unavoidable caveat is that these findings are not objective (nor could they be), and are not quantifiably generalisable to all situations---they are based upon ethnographic accounts of the studied groups of participants socialising in a given context and setting.
\iresubmission{Producing such a larger record of many cohorts or different settings would be an insurmountable task for a single thesis, and to do so} would likely result in the dilution of the richness in which context is established as a factor in shaping interaction.
It is the attention to the minutiae that furnishes the analysis with rich insight into the actions of people, yet also serves as a limitation of this work.

% \paragraph{Interdisciplinary Perspective with EMCA} \hfill \\
% This project was shaped by the presence of multiple views onto the research questions posed, that include perspectives that have historically found critique within each other epistemologically (see \ref{sec:background emca discourse}, where some of the debate around this is partially touched upon).
% These concerns were sidestepped in this work through the careful attention to attending to matters of accountable interaction in the analysis only, as in the ethnomethodological tradition.
% The discussion brought into play how people dealt with factors relating to mental resources \textit{accountably} through the employment of the Multiple Resources Model as a reflective framework.

% Yet, however, the case for further work to deepen the findings into how workload and mental resources are managed within such settings using other measurement devices could be established.
% Although the discussion and realisation of how people avoided situations that would result in potentially overloaded situations in and through talk was discussed, quantifiable findings, as is the norm in such situations, could not be established due to the nature of data collection.
% The established implications, in their nature, are a subjective assessment of the data presented in the empirical chapters to support and complement research across disciplines.

\newpage\paragraph{Discussion and linking of findings to design} \hfill \\
\begin{revisedsubmission}
This thesis explicated the collaborative efforts of members in using a device in conversation, and of how this turned upon the natural accountability of action, yet this thesis did not generate exhaustive implications for design.
Furthermore, this thesis proposed that work in \ac{HCI} to create collaborative experiences using mobile technologies could explore the use of \acp{VUI} in their design.
Although others have remarked upon how it is a key activity for ethnographic studies to yield implications for design (e.g. \citet{Crabtree2012}), the focus on this work was addressing the literature gap of empirical data of people socialising and using technology.
Nevertheless, implications arising from this underlying research in the empirical chapters have been published in the fields of \ac{HCI} and \ac{CSCW}, although these publications differ in parts from the analysis presented in this thesis\footnote{Each empirical chapter corresponds to a single publication: \citet{Porcheron2016a,Porcheron2017,Porcheron2018} respectively.}.
However, neither the notion of collaboration with and around \acp{VUI}, or the resulting implications in the publications, are established or verified through an experiment.
Instead, these notions were derived out reflection of the explicated machinery of interaction, identified through members' naturally accountable actions.
Such a limitation is, of course, an avenue for future research and design to expand upon these findings, rather than a devaluation of the work in this thesis.
\end{revisedsubmission}



% *********************************************************************************************************************



\section{Future work}\label{sec:synopsis conclusions future}
This chapter has brought together and summarised this thesis' contributions and limitations.
Looking forward, this thesis now ultimately concludes with how future research might proceed that builds upon these conclusions and addresses these limitations.

\begin{revisedsubmission}
The naturalistic participant-observer approach used in the first two studies, and the automated selective data capture with the third, allowed for an analysis that oriented to understanding conversation among members in the setting.
However, this analysis was limited in that further work could be done on the data to orient to different matters of interaction, such as the specific nature of how talk to devices is modulated and how this varies over time, including factors such as `recipient design' (i.e. how users adapt their voice to get the device to work~\citep{Clark1996}).
In other words, the corpus of data, especially in relation to that of \acp{VUI} in the home, is rich and ripe for further analysis drawing upon methods such as Conversation Analysis.
Additionally, future studies could orient to different matters of how talk to the devices is constructed, drawing on disciplines such as linguistics to provide greater insight into the language used, as some have already begun to argue for with regards to \acp{VUI} (e.g. \citet{Sutton2019}).
These approaches would potentially generate additional findings in relation to what this thesis has offered in support of design tweaks, especially in the case of \ac{VUI} design.

Another limitation is, as discussed, that the conduct observed is of specific cohorts of people, but that further configurations of cohorts and/or settings might reveal different findings.
For example, friends socialising together in a pub is not the only combination where technology use is identified as problematic, with recent research in \ac{HCI} examining notions of couples using mobile devices in bed~\citep{Salmela2019}, and relating this to literature on using technology while collocated.
With this, the foundations of this thesis hopefully reinforce and support future work in \ac{HCI} to continually examine and pursue the idea of revealing technology in a range of settings and cohorts, all of which provides richer insights for research, and implications for the design of future technologies.
The call for further examination of these settings in \ac{CSCW} serves as a feasibility proposal to identify the ways in which \ac{VUI} technologies could be embedded within the work of these---and other---settings.

Finally, the critical discussion of the methods employed by members in using a device identified the collaborative practices of members, and how these turn upon the accountable nature of interaction.
Such approaches steered clear of `obvious' and redundant challenges such as `better voice recognition' and instead attempted to explicate nuanced and provocative ideas for how to harness or disrupt existing methods, or to ameliorate the difficulty people had in `getting devices to work'.
The thesis proposed that designers and researchers who currently build and study \ac{CSCW} systems for collaboration using portable devices could consider ways of using \acp{VUI} in future \ac{CSCW} systems.
However, further work could build on this idea to examine and determine its applicability to different settings following a \textit{research through design} approach~\citep{Zimmerman2007a}, generating meaningful conclusions about just how such interactions unfold, and what implications these have for the design of future technologies.

In summary, the work in this thesis is not a comprehensive answer to what all device use entails in all settings, and does not offer a checklist of solutions for how technology could be redesigned to suit members' interactional needs.
Nevertheless, it should serve as an empirical primer that can be utilised as a keystone to supporting further studies across disciplines that examine and design for everyday interaction with devices.
\end{revisedsubmission}



% *********************************************************************************************************************



\section{Final remarks}\label{sec:synopsis conclusions final}
In conclusion, this thesis has delivered an \iresubmission{empirical} insight into \iresubmission{for what purposes and} how device use is \iresubmission{interactionally} organised \iresubmission{as an activity in multi-party casual-social settings}.
This was to address a fundamental gap in a crowded body of literature on device use in everyday life, \iresubmission{much of which focuses on} specifically when we are collocated with others.
\iresubmission{Through the description, discussion and reflection upon members' actions in the settings, device use was shown to unfold through the routine of socialising together as an \textit{embedded} activity, with members making their device use a naturally accountable phenomenon.
Members also collaborated on their interactional projects for which device use was occasioned, with this collaboration ostensibly turning upon the accountability of the use of the device.
This thesis makes a number of contributions to the existing literature on the nature of device use in these settings, reflections of the methodological approach of the three studies in this thesis, and of conceptual insights that call for further work in \ac{HCI} and \ac{CSCW}.}



% *********************************************************************************************************************

