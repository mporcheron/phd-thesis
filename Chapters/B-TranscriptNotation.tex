%!TEX root = ../PhDThesis.tex



% *********************************************************************************************************************
\chapter{Transcript Notation}\label{app:notation}
% *********************************************************************************************************************



In general, the orthographic notation for fragments of transcribed data is based upon the system employed by \citet{Heath2010}.
However, a number of differences and simplifications have been made for brevity, clarity, and conciseness.
This notation itself was derived from the notation system originally described by \citet{Atkinson1984a}, devised by Gail Jefferson, and often used in Conversation Analysis-related literature.

Summarily, the notation used for transcribed data in this thesis adopts the following conventions:

\begin{itemize}
\item the volume of talk is denoted as \texttt{LOUD} or \texttt{\degree{quiet}\degree},
\item emphasis is denoted with \texttt{\underline{underlined text}},
\item shifts in intonation are given as arrows, i.e. \texttt{$\uparrow$} for a rising intonation and \texttt{$\downarrow$} for a falling intonation,
\item a single dash (\texttt{-}) when an utterance is cut off,
\item an equals (\texttt{=}) at the end of an utterance and at the start of a following utterance to denote contiguous talk (indentation is used to improve clarity and readability in these cases),
\item elongation of sounds and words are like \texttt{th::is}, where the \texttt{th} sound is two-tenths of a second in length,
\item pauses between words and utterances are given as \texttt{(.)}, where each individual period represents a tenth of a second; or as \texttt{(0.4)}, where this represents a pause of 0.4 seconds,
\item overlapping talk or action is denoted using opening square brackets (\texttt{[}) and closing square brackets (\texttt{]}) where possible or applicable (sometimes this closing bracket is omitted if two concurrent utterances end simultaneously at an end of a turn),
\item indentation is often used with overlapping talk or action to aid readability,
\item actions are given in \texttt{((double parentheses))},
\item utterances to an electronic device as a query (i.e. as input to the device) are given as \texttt{\textbf{bold text}},
\item digitally produced spoken words from an electronic device are preceded and succeeded with two forward slashes and typed in \texttt{\textit{// italics // }}, and
\item Names are typically denoted using the first three letters from the first name of the participant, e.g. \texttt{LIL} for \textit{Lily}; the researcher is identified as \texttt{RES}.
\end{itemize}



% *********************************************************************************************************************
