%!TEX root = ../PhDThesis.tex



% *********************************************************************************************************************
\chapter{Approach}\label{ch:background approach}
% *********************************************************************************************************************



\begin{revisedsubmission}[ER-D, IR-3: Proposed restructuring of background and approach to clearly indicate and justify the adoption of an ethnomethodological lens.]
This chapter brings to the fore and discusses the conceptual side of the methodological approach adopted in this thesis to collecting and analysing empirical data of members' actions in the three settings studied.
This chapter will discuss how this thesis' approach develops an understanding of the interactional accomplishment of conversing and socialising while interleaving the use of a device.
\resubmission{J3-3a, ER-D, IR-3: New approach and methodology framing}This chapter will introduce and situate ethnography as a method of academic inquiry, used across multiple disciplines, and of its relevance and propriety for the studies in this thesis. 
Ethnomethodology ``is the study of the methods people use for producing recognizable social orders [\ldots] to discover the things that persons in particular situations do, the methods they use, to create the patterned orderliness of social life''~\citep[p. 6]{Garfinkel2002a}.
This chapter will establish this thesis' specific form of ethnography---ethnomethodology---and how the analysis is informed by the ethnomethodological perspective, through which an understanding of how those conversing interleave device interactions with talk.
\end{revisedsubmission}
%This chapter will first briefly introduce the cross-disciplinary nature of studying discourse, focusing primarily on spoken face-to-face studies of interaction.
%Then, in greater detail, the chapter will hone in on the underlying philosophical and epistemological perspective of ethnomethodology.
%\textit{Ethnomethodologically informed ethnography} will then be described, with the analytic tool of \acf{CA}  introduced.

%In this thesis, the adoption of the ethnomethodological tradition, which historically draws upon the interactionist perspective, and \acf{CA}, which is often found in studies of pragmatics, are used to inform the design of technologies.
%Such practices are established already in \ac{HCI} and Ubiquitous Computing research~\citep{Crabtree2006,Gilbert1990}.
%This analytic perspective is consistent across each of the three studies, with practical variances outlined in the corresponding empirical chapters.

Of the three studies in this thesis, the first two draw upon video-supported ethnography to aid explication of what is done by members of the settings as they use a device during a gathering in a semi-public setting (see Chapters \ref{ch:empirical pub} and \ref{ch:empirical cafe}).
The third study continues to draw upon the same analytic perspective, although consists of automatically captured audio data from interactions with a device in the home (see \autoref{ch:empirical home}).
The analytic perspective remains the same across all three studies, however, and this chapter helps frame this perspective \iresubmission{but will refrain from discussing the practical matters of how each study was conducted `in-the-wild', in part because they vary across each study; such details are included in each empirical chapter, so as to allow for the explanation of how data were collected and analysed in the context of the specific technology.}
In this thesis, the analysis was not undertaken as a distinct post hoc event but instead occurred throughout the fieldwork-based studies---a practice that this chapter will establish as core to ethnographic work.
Therefore, it becomes all the more relevant to understand Ethnomethodology not just as a tool for making sense of the data collected and presented, but to understand the practical aspects through which these data were collected and selected for presentation in this thesis.



% *********************************************************************************************************************



\section{Ethnography}\label{sec:background approach ethnography}
\begin{revisedsubmission}[ER-D: Introduce the development of the ethnomethodological tradition]
This thesis adopts an interactionist perspective to understanding everyday interaction, drawing upon ethnographic practices and eschewing other methods of work, such as \textit{a posteriori} interviews or self-report methods.
\textit{Ethnography}---an approach to the study of social life developed by \citet{Malinowski1922}---takes many forms and is now found in many different disciplines from sociology through to computer science.
Practically, however, the perspective through which the ethnography is conceptualised and established varies.
There are many different approaches and perspectives under the banner of \textit{ethnography}, this thesis adopts the perspective of ethnomethodology~\citep{Garfinkel1967}.
In this chapter, the \textit{ethnomethodological} perspective will be introduced, and it is through this analytical lens that the primary objective of this thesis---to reveal \textit{how} individuals conversing with others interleave the use of a device in conversation---will be met.
First, however, a summary of the development of the ethnographic approach is included here to situate and rationalise the adoption of the ethnomethodological approach to ethnography.

Malinowski conducted an ethnographic study of the native inhabitants of Guinea and exposed their practices to others through his influential work \textit{Argonauts of the Western Pacific}.
Through this, he influenced the anthropological study of communities and settings by shifting such studies from efforts that focused on the collection of artefacts, stories, and measurements into a pursuit of \textit{immersion} in the setting; to ``grasp the native’s point of view [\ldots] to realise his vision of his world''~\citep[p. 25]{Malinowski1922}.
His work, through circumstance, was achieved not through mere observation of the inhabitants or through a retrospective analysis of collected artefacts, but by immersing himself in the culture of those he was studying, to \textit{experience what they experienced}.
Through his study, Malinowski established ethnography as a prominent scientific method which relies upon not just observation or fieldwork taking place, but one in which the ethnography itself chronicles the \textit{behaviour} of those under study, and not just the tools they used or a description of the environment in which they were used:
\begin{quote}
    In Ethnography, where a candid account of such data [a detailed account of all the arrangements of the experiments; an exact description of the apparatus used; of the manner in which the observations were conducted] is [perhaps even more] necessary, it has unfortunately in the past not always been supplied with sufficient generosity, and many writers do not ply the full searchlight of methodic sincerity, as they move among their facts but produce them before us out of complete obscurity. [\ldots] In ethnography, the writer is his own chronicler and the historian at the same time, while his sources are no doubt easily accessible, but also supremely elusive and complex; they are not embodied in fixed, material documents, but in the behaviour and in the memory of living men.
    \quoteauthor{\citet[pp. 3--5]{Malinowski1922}}
\end{quote}
In this sense, Malinowski established that ethnography is not just observation or the analysis of a corpus of exhibits~\citep{Bittner1973}, but more so requires the ethnographer to understand the phenomena under study as if one were a member, and could see and make sense of the work in the setting from the members' point of view.
Therefore, ethnography is \textit{not just} fieldwork but also the resulting analysis~\citep{Anderson1997}, in which a demonstrably useful account of the lives of others is established~\citep{Button2000,Crabtree2012}.
His work required immersion, time, and devotion to understanding language and practices, and a willingness to partake in the society as if he were a native.
Malinowski's work posited this, and demonstrably revealed how it was necessary to make sense and understand the perspective of the `natives' and chronicle their lived experience.

Of course, Malinowski's study was patently different to the work of this thesis: here the people under study are not in a remote society with which there was nascent knowledge in the Western canon.
Ethnography, as understood in the perspective of this thesis, was first pioneered by the Chicago School of Sociology, which took the matter of \textit{everyday life} to be its locus of study---not ``non-western societies and cultures'' but instead ``the city as its subject matter, and through numerous extensive and detailed ethnographic examinations of urban life subjected the city to an order of examination previously reserved for ‘other’ societies and cultures''~\citep[p. 112]{Button2015}.
In this, one such pioneer, \citet{Hughes1958}, spurred ethnography from a study of another's culture to a study of one's own society, asking his students to study their taxi rides, cleaners, and so forth.
However, one critique of the work from this era was that such ethnographies failed to explicate the `interactional work' (this is discussed later in \ref{sec:background approach em work}) of members of the setting; i.e. they failed to reveal the social phenomena but relied upon `scenic' features of action and in doing so, failed to allow the reader to understand the `work' of the setting~\citep{Crabtree2009}.
With this, it becomes evident that it is not enough to be able to speak the language, or to be vaguely familiar with `what is done'.
Conducting an ethnography---even if you `know the language'---still presents many challenges, not least access to the setting, which may prove challenging as many settings are not readily observable by the public.
With this challenge comes issues of securing the consent of participants to collect data and the development of the researcher's competence to make sense of the work under investigation~\citep[pp. 89--95]{Crabtree2012}.
%How ethnomethodology meets the challenge of addressing some of these issues, such as competence, is discussed in the next section (see \ref{sec:background approach em}).

Bringing the influences of the Malinowski and Chicago School together: ethnography, which can involve the study of things that `might seem familiar' or `common sense'~\citep[p. 160]{Crabtree2012}, requires an ethnographer to embed themselves in the setting to understand the experience of the members of the setting from their perspective.
Moreover, the production of an ethnography is \textit{not only} fieldwork but consists of fieldwork and analysis in unison, in which the fieldwork is guided by the analytic perspective to make sense of members' actions.
Critique of the work of the Chicago School's influence on ethnography varied, but the next section of this chapter expands upon one element of this critique---of a reliance upon scenic descriptions---and introduces how the tradition of ethnomethodology, and its orientation and stance to ethnographic work, can furnish readers with a richer ethnographic record of members' methodical accomplishments.
%However, although Malinowkski's analysis was post hoc, insomuch that the fieldwork was do0jne  demarcation of fieldwork \textit{and} analysis does not suggest that these activities exist as sequential steps, but instead the analysis p
\end{revisedsubmission}



% *********************************************************************************************************************



\section{The Ethnomethodological perspective}\label{sec:background approach em}
\begin{revisedsubmission}[ER-D: Introduce the development of the ethnomethodological tradition]
This thesis adopts the perspective of ethnomethodology in its approach to ethnography.
A study that is ethnomethodological in character focuses on the ongoing ordinary primordially social features of everyday interaction~\citep{Schegloff1987}; in other words, an ethnomethodological study reveals ``the techniques and strategies members of society use in making sense of one-another’s subjective perspective on everyday experience, and through these methods, achieving a significant measure of shared understanding''~\citep[p. 30]{Reeves2011}.
It is through this orientation to everyday routine practices of individuals that ethnomethodology develops its concern with the accountable ways in which members organise their conduct, moment-by-moment, relevant to their context.
There are several key tenets to understanding the ethnomethodological perspective that are introduced here: the explication of the \textit{work} and \textit{interactional what}, the sequentiality of action, the policy of \textit{indifference}, and the notion of \textit{vulgar competence}.
Each point is addressed in turn throughout the remainder of this chapter, and through these points the analytic lens in which each study in this thesis was conducted is assembled.
\end{revisedsubmission}



% *********************************************************************************************************************



\subsection{`Work' and `interactional what'}\label{sec:background approach em work}
In this thesis, interaction is treated as the locus of study, with accounts of what is done by the members of the setting being the primary resource used to establish the ethnographic record.
This record consists of \textit{thick descriptions}~\citep{Geertz1973a} that unpack and reveal the interactional \textit{naturally accountable} methods of members (i.e. \textit{the accountable character} of work in the setting).
\iresubmission{Ethnomethodology, it is argued, suspends the assumption that social ``order is a rare beast to be found in only a few places''~\citep[p. 6]{Crabtree2013} but is instead a constituent feature of the ordinary activities and common-sense reasoning that inhabits and animates it, and it is this that an ethnomethodological lens illuminates.}
In this regard, the ethnographic record, then, will reveal the social order of the phenomena under study, or in other words, will allow readers to make sense of the actions of members as they converse and interleave device use within this conversation.
The thick description, which will be assembled as a result of the fieldwork and analysis, is produced through attention to ``to what is done in the doing of action [through] `thicken[ing] up’ the thinnest level of description to make its accountable character visible and available to others''~\citep[pp 117--118]{Crabtree2012}.
With this, accountability is defined as an action that is observable and reportable~\citep{Garfinkel1967}, i.e. what it is done is observ-\textit{able} and tell-\textit{able} by the other parties who are present~\citep[pp 117--118]{Crabtree2012}.
%Such actions have, in turn, an ``incarnate reflexivity'' to them\citep[p. 28]{Crabtree2012}, as it is through the enactment of the routine actions of the setting, members make it accountable and ``and in doing so reflexively organise whatever it is that they are doing''.
It is this thick description that documents the accountable work of members of the setting.
\textit{Work}, in the sense of ethnomethodology, is not treated as equivalent to paid labour but is considered the achievement of mundane naturally occurring activities~\citep{Schmidt2010,Crabtree2009,Crabtree2006,Button2012}, with Sacks succinctly regarding it as a culmination of `everyday stuff that is done' in and through a person living their ordinary routine:
\begin{quote}
    Whatever we may think about what it is to be an ordinary person in the world, an initial shift is not to think of an ‘ordinary person’ as some person, but as somebody having as their job [\ldots] doing ‘being ordinary’.
    It's not that somebody is ordinary [\ldots] it takes work, as any other business.
    \quoteauthor{\citet[pp. 215--221]{Sacks1992}\hfill\\edited by \citet[pp. 23--24]{Crabtree2012}}
\end{quote}
Although there have been many ethnographic studies of `ordinary activities', the ethnomethodological orientation to ethnography also embellishes qualitative participant-observation approaches with attendance to revealing the ``interactional what''~\citep{GarfinkelMissing}.
This notion of `what' was developed in a commentary by David Sudnow and Garfinkel in response to studies of Jazz singers by Howard Becker, whose work featured heavily in the development of ethnography in the Chicago School of Sociology.
Becker, amongst his work, provided accounts for ``the career structure of the jazz musician, the fraternal organisation of work it gave rise to, the pressures of work and playing to the audience, the dilemma of commercialism versus prestige, and the impact of family on the musician’s life and the conflict it generates''~\citep[pp. 116--117]{Button2015}, yet Sudnow and Garfinkel argue he did not reveal the circumstances in which music was collaboratively accomplished---his work was informative, and well developed, but did not reveal the \textit{interactional work} of a Jazz musician.
These studies, although ethnographic, were found to merely provide ``scenic descriptions'' of what is done and were of limited use in understanding how interaction was specifically achieved as a situated and coordinated action (the notion of \textit{situated action} is elaborated upon later in \ref{sec:background approach em sequentiality}).
The tradition of ethnomethodology is not only crucial to understanding and making available what is done as a gloss, but principally what is done \textit{in interaction} (i.e. the interactional work of Jazz musicians as they did it, from their perspective).
In assembling the ethnographic record of what is done---to unpack this gloss---it becomes necessary to understand and relate the actions of members as a series of particular activities.
As \citet{Crabtree2012} elaborate:
\begin{quote}
    We need to be able to see the activities that produce sequential order in the ‘lived’ details of their production -- i.e., in details of the particular things that members do to accomplish the component activities of a sequence.
    \quoteauthor{\citet[pp. 103--106]{Crabtree2012}}
\end{quote}
\iresubmission{With this perspective it becomes evident that an ethnographic record informed by ethnomethodology should reveal the activities of members through some form of sequential order, to allow the reader to understand the member's perspective and actions, and of how such actions are constituent in the sequential ordering of an activity.
In this, the actions of members become assembled as a series of \textit{sequential accomplishments}, to thicken scenic descriptions so that they allow the reader to make sense of the members' practical reasoning and practical action.
It is this notion of \textit{sequentiality} and the situated\textit{-ness} of action that the next section details.}



% *********************************************************************************************************************



\subsection{Sequential organisation of situated action}\label{sec:background approach em sequentiality}
Firstly, \textit{sequentiality} is defined as ``any kind of organization which concerns the relative positioning of utterances or actions [\ldots] turn-taking [in conversation] is a type of sequential organization because it concerns the relative ordering of speakers''~\citep[pp. 1--3]{Schegloff2007}.
With this definition, it is important to note that sequentiality differs from mere temporal ordering (although it can take advantage of it), not only in that it encompasses actions that occur temporally in tandem (such as overlapped talk), but that the sequential coherence of conversation is a continuous achievement by conversationalists, who are seeking to assemble the retrospective-prospective sense of those actions which are often outside a basic temporal order.
For instance, a speaker might answer a question several turns subsequent to it being posed in a conversation (which might be accounted for by a speaker in various ways, e.g. prefacing ``before I answer your question\ldots'' to their turn).
The notion of retrospective-prospective is key here, as Garfinkel notes:
\begin{quote}
    Many expressions are such that their sense cannot be decided unless one knows or assumes something about the biography and the purposes of the speaker, the circumstances of the utterance, the previous course of the conversation, or the particular relationship of actual or potential interaction that exists between speakers. The sensible character of an expression requires that we wait for what a speaker or speakers say next for the present significance of what has already been said to be clarified. Thus, many expressions have the property of being progressively realised and realisable through the further course of the conversation.
    \quoteauthor{\citet[pp. 35--75]{Garfinkel1967}\hfill\\edited by \citet[pp. 122--123]{Crabtree2012}}
\end{quote} 
With this, the case is established that action is both context-shaped, in that to understand it one must know the context within which it unfolded, and also context-shaping, in that each action carries implications for future actions, and is only realised through those future actions.
In this, actions become coherently and sequentially \textit{organised}.
\label{line:suchman}It is this feature of interaction as being sequentially organised, and further so locally and longitudinally managed by members that provides the basis for Suchman's notion of `situated action' (i.e. the arrangements of this organisation of action are negotiated and established only in and through their production and the context of the interaction~\citep{Button1995a,Nguyen2008}).
Suchman's analysis draws upon observation of everyday interaction with an agent-based photocopier at Xerox PARC, and by drawing upon ethnomethodology, she was able to explicate not only issues with the design of the hardware but also fundamental notions of the mundane achievement of work in using the device. On situated action, she notes that:
\begin{quote}
    That term underscores the view that every course of action depends in essential ways on its material and social circumstances.
    Rather than attempt to abstract action away from its circumstances and represent it as a rational plan, the approach is to study how people use their circumstances to achieve intelligent action.
    \quoteauthor{\citet[p. 35]{Suchman1985}}%   Rather than build a theory of action out of a theory of plans, the aim is to investigate how people produce and find evidence for plans in the course of situated action.
\end{quote}
Suchman's definition builds in the notion that people are `everyday sociologists', and that members of settings can observe and recognise what other members of the setting are doing, and that this stems from the natural accountability of members' actions~\citep{Berger1966}.
\label{line:naturalaccountability}Natural accountability is the notion that the `members' of a setting can observe the work of others around them in that setting, and crucially, \textit{know} what it is that they and others involved in that work are doing~\citep[pp. 1--34]{Garfinkel1967}.
With this, members can unproblematically offer an account of what they are observing, and that the other members of the setting will recognise this account~\citep[pp. 1--34]{Garfinkel1967}.
Specifically, members' actions are naturally accountable in terms of their \textit{practical action and practical reasoning}~\citep{Garfinkel1970}---ethnomethodology is not concerned with `activity', `action', or `agency', but with how these notions are ``ordinarily understood by the members of society from within the settings in which they operate''~\citep[p. 29]{Crabtree2012}.
As \citet{Crabtree2012} remarks: ``[t]he naturally accountable character of everyday activities is an achieved outcome of their conduct, which is to say that in making their activities happen---in the work of assembling and accomplishing them---members attend as a matter of course to making them naturally accountable''~\citep[p. 25]{Crabtree2012}.
Moreover, not only does Suchman's work provide the practical methodological approach for this thesis, but the premise of action as established in and through its achievement as a product of the context within which it is done, is imperative in bringing the empirical findings into the context of design in this thesis.
In other words, Suchman's influential work scopes out a field of work in which research in interaction with systems focuses on the practically and accountably done actions as opposed to theoretical assumptions of action, and so sensitises researchers to the need to include, not abstract, context-shaping and context-shaped implications of action.

In this thesis, the work of members in the setting will be chronicled through the presentation of series of excerpts of data.
Members' interactional accomplishments will be analysed with respect to the coherence and situated nature in which they occur, enabling the analysis to reveal the \textit{interactional what} of how device use is interleaved within conversation.



% *********************************************************************************************************************



\subsection{Ethnomethodological indifference}\label{sec:background approach em indifference}
\iresubmission{Thus far, this section has detailed the ethnomethodological orientation to the interactional work of members in settings, and how this is revealed through an attention to the sequentiality through which their accountable actions are conducted.
Through inference, it should also be clear that this thesis is not concerned with theories of `why' something happened, or indeed theories of interaction or work in general, but rather focuses on the practical situated accomplishment of action.}
In this sense, the thesis adopts the notion of ``ethnomethodological indifference''~\citep{Garfinkel1970}.
As summarised by \citet{Lynch1993}, this consideration allows researchers to pragmatically study the work of people ``[r]ather than addressing whether sociologists ever can achieve adequate or acceptable accounts of the phenomena they study''~\citep[p. 141]{Lynch1993}.
In other words, what matters in this research is explicating the \textit{members' methods} of interaction without \textit{a priori} models of how such interaction unfolds~\citep{Livingston1987}, i.e. there need not be a theoretical unpinning of understanding in how people use mobile devices because this work is primarily concerned with the \textit{accountable} interaction of the setting.

\begin{revisedsubmission}
\citet{Button2015} argue that an account of the setting imbued with interpretation detracts from the work of the setting, transforming the thick description of members' actions into an interpretation of both members' actions and interpretations.
This claim rubs up against others who argue that ``indifference'' is not necessary to derive a valid account.
\citet{Dourish2014} argues that the allowance of researchers to draw upon epistemological notions does not impede analysis, as such analysis is rooted in the researcher's immersion and experience of the setting and that the researcher's account of the work of the setting is not invalidated as such.
However, for this thesis, indifference was adopted insomuch that the goal of the work was to `take a step back' from the critical assessments of device use found in existing literature---both academic and popular press---and instead practically study how such device use is interleaved within conversation, and through this explicate the \textit{members' methods} of how this use is achieved.
Applying \textit{a priori} understanding to the analysis would instead pivot this work from an examination of how device use unfolds as an accomplishment and instead project the existing rhetoric of device use upon the analysis, and in turn, diminish the contribution and motivations behind the thesis.
Thus, in accordance with practice guided by others, the data collection and analysis in this thesis was based on studying settings and analysing data without \textit{a priori} frameworks of what constitutes interleaving practice, and instead allows such notions of how members' actions unfold to be guided by the data~\citep{Crabtree2012,Heath2010}.
\end{revisedsubmission}


% *********************************************************************************************************************



\subsection{Vulgar competency}\label{sec:background approach em comptency}
\begin{revisedsubmission}
The final matter to address in this brief summary of the development of ethnomethodology and this thesis' methodological approach is to consider the issue of \textit{vulgar competency}, which is fundamental to how researchers reliably make sense and present the social order of the setting, i.e. of how members make their actions accountable to each other~\citep{Garfinkel1992a}.
The notion of competency is the antithesis to the interpretation of findings and is developed through the ethnographer attaining a position in the setting in which they not only understand the routines of the setting---a \textit{gloss}, if you will, of what is done---but how the members accomplish that routine.
\citet{Button2009} solidify the necessity of developing competency: ``even if it appears to an outsider that nothing is going on, there will be something that is being done''~\citep[p. 86]{Button2009}.
\citet{Slack2000} further argues that vulgar competency is intrinsically connected to the type of account produced in the analysis: if an ethnographer does not adequately attend to members' everyday work practices through the perspective in which the member lived and undertook them, then those accounts falter and potentially become mere interpretations of phenomena.
In other words, if one cannot see it from the members' perspective, then any account cannot be a true reflection of the members' situated action.

Furthermore, the notion of competency stands in unison with that of ethnomethodological indifference: such competency should come from an understanding of the member's methods rather than through the use of formal sociological methods of inspection.
Through the development of vulgar competence, it becomes possible for the analyst to understand the `inner-workings' of the setting and of the members' perspective, allowing the ``analytic and member concerns [to] merge [such that] the very distinction between the [\ldots] analyst and member is obliterated''~\citep[p. 15]{Pollner2012}.

The specific approach to developing competency, as with all ethnographies, depends upon the setting under study.
This thesis, perhaps more so than others that adopt an ethnomethodological perspective, studies an activity which intrinsically motivated the thesis: the initial ideas for this thesis came from remarks by others of the author's use of a device while they were socialising together in a pub.
This candid observation from a friend was transformed into a research proposal that spawned a thesis that examined just how this device use unfolds in social settings. % and of how this varies when the modality of device interaction varies.
%Insomuch as can be regarded, 
This thesis studies settings in which the researcher was already a part of and had a competency: i.e. people socialising together and who ostensibly use technology in such gatherings.
\end{revisedsubmission}


% To be able to reliably present how the work is organised in a setting and how members make their actions and interactions accountable to one another, researchers need to develop “vulgar competence” in the members’ work (Garfinkel & Wieder 1992). Developing such competence should not be treated as a banal matter, as it is the ticket to understanding not only what members achieve through their daily routine, but also how they carry out their work.

% This routine might seem as if it just happens. However, there is always reasoning and sense-making behind members’ actions; “even if it appears to an outsider that nothing is going on, there will be something that is being done” (Button & Sharrock 2009). The researchers’ job is to reach a position where they are capable of not only understanding the routine but also how “members make their work routine” (Crabtree et al. 2012), and being “in a concerted competence of methods” (Garfinkel & Wieder 1992) with the members.

%\footnote{Action selection is obviously more much complex than simply `choice', but that is for another thesis.}.

% \section{Studies of discourse and interaction}\label{sec:background emca discourse}
% This section briefly introduces numerous perspectives on the study of language, communication, conversation, and discourse.
% This will introduce and establish how the stance taken with the analysis in this thesis differs from similar and interrelated approaches.%\footnote{Perhaps this section should be considered to be ``what this thesis isn't, but what has contributed to the stance taken''.}.

% Studies of language and interaction are inherently interdisciplinary in their nature~\citep{Brumfit1997}, drawing upon differing perspectives and methodologies to generate understanding and language use as a resource in interaction.
% Early studies of language were born out of anthropological ethnographic studies, but within quick succession work began to develop in numerous distinct but related disciplines.
% Not least, disciplines such as linguistics, sociolinguistics, cognitive and social psychology, semiotics, and pragmatics all begun to tackle challenges relating to the formation, analysis, and production of language~\citep{Dijk2011}.
% As each discipline here is the culmination of decades of independent and cross-disciplinary work, no thesis could feasibly and justifiably produce a comprehensive introduction.

% Many of these disciplines orient to similar aspects of interaction and contribute to each others' work and study.
% Herein, only a few disciplines are introduced.
% Each discipline discussed focuses on interaction as a face-to-face contingent conversation that is situationally embedded\footnote{There are also studies of written discourse and face-to-face interaction from different perspectives too, e.g. \textit{discourse studies}, \textit{corpus linguistics}~\citep{Dash2005}, and perhaps more contemporarily, \textit{social computing} research.}.
% \citet{Dijk2011} identifies three main strands to studies of language and discourse:

% \begin{description}
%     \item[Models of Discourse Production and Comprehension] \hfill \\
%     The first strand of work introduced constructs mental representations and processes involved in discourse production and comprehension, and is typically found in \textit{neuroscience}.
%     This work is considerably removed from the language and aims of this thesis, and so is only acknowledged here for completeness.

%     \item[Formal and Cognitive Studies of Discourse] \hfill \\
%     The second route of work is adopted in \textit{Cognitive Science}, and often contributes to the development of \acfp{UI} and \acf{AI}.
%     This work, especially in recent studies in \ac{HCI}, adopts similar notions of studying action and settings, to devise models of human cognition within spatial settings.
%     This work is briefly discussed in \ref{sec:background emca discourse cog}.

%     \item[Pragmatic Studies of Discourse] \hfill \\
%     The third strand of work is the study of \textit{Pragmatics}, and is briefly introduced in \ref{sec:background emca discourse prag}.
%     Pragmatics is an interdisicipline that combines the study of language and semiotics, by orienting to the use of language, the context within which it occurs, and the organisation of language (among other topics).
%     This thesis uses one such approach in studying interaction, \acf{EMCA}, which embodies the study of \textit{Pragmatics}; this is unpacked in the next section of this chapter (see \ref{sec:background emca em}).
% \end{description}


% *********************************************************************************************************************


% \subsection{Formal and cognitive studies of discourse}\label{sec:background emca discourse cog}
% Cognitive Science, like most approaches to studying discourse and interaction, is ``a recognition of a fundamental set of common concerns shared by the disciplines of psychology, computer science, linguistics, economics, epistemology, and the social sciences generally''~\citep{Simon1981}.
% Cognitive Science focuses on the creation of systems that draw on notions of \ac{AI}, and that can interact and accomplish tasks, accounting for humanistic traits and `natural use'.
% The domain was founded around the premise and modus operandi of the reducibility of \textit{intelligence} to symbols and \textit{models} of action:

% \begin{quote}
%     ``At the root of intelligence are symbols, with their denotative power and their susceptibility to manipulation. And symbols can be manufactured of almost anything that can be arranged and patterned and combined. Intelligence is mind implemented by any patternable kind of matter.'' ---~\citet{Simon1981}
% \end{quote}

% The interdiscipline orients to the challenge of representing knowledge and action such that tasks can be planned and ordered~\citep{Bobrow1975}.
% In this sense, cognitive science is akin to a challenge of design, and is used within \ac{HCI} both to evaluate systems (e.g. \citet{Furniss2015}) and to design new technologies~\citep{Hollan2000}.
% As the field has progressed, numerous critiques of this approach have been raised, as \citet{Klatt1981} points out, however, such representation remains useful in studying spoken discourse:

% \begin{quote}
%     ``Linguists are usually careful to point out that a generative grammar is a descriptive device rather than a model \ldots~Nonetheless, the forms and rules developed for linguistic descriptions constitute a good starting point for the study of lexical representation and lexical access.'' ---~\citet{Klatt1981}
% \end{quote}

% Elsewhere, \citet{Suchman1985}, in her seminal work \textit{Plans and Situated Actions},  discusses how the models used in the development of machine interfaces at the time were not sufficient for everyday human interaction.
% The crux of her work was that work that typically relied on the notion of `linear plans' of action, developed under the presumption that people specifically plan and enact routines as if we were machines ourselves, was invalid.
% In contradiction with this notion, her work realises that plans are constructed in and through their enactment and that actions are situationally achieved\footnote{Simply put, her work reminds us that the world is messy, and demonstrates that the idea that an interaction or course of actions can be planned completely is insurmountable --- the doing of a task is entirely achieved as a situated action dependant upon the social order within which the task is done.}.
% \citet{Norman1980} too underscores the need for an interdisciplinary approach in cognitive sciences to address such limitations:

% \begin{quote}
%     ``Cognitive scientists, as a whole ought to make more use of evidence from neurosciences, from brain damage and mental illness, from cognitive sociology and anthropology, and from clinical studies of the human. These must be accompanied, of course, with the study of language, of the psychological aspects of human processing structures, and of artificially intelligent mechanisms. The study of Cognitive Science requires a complex interaction among different issues of concern, an interaction that will not be properly understood until all parts are understood, with no port independent of the others, the whole requiring the parts, and the parts the whole.'' ---~\citet{Norman1980}
% \end{quote}

% Numerous cognitive approaches have been adopted that circumvent this critique by orienting to the sociality and artefacts in a setting to explicate the accomplishment of cognition.
% \citet{Hutchins1995} established the methodological and theoretical framework of \acf{DCog}, which upends the cognitive perspective by bringing into play the social and spatial aspects of cognition, as summarised by \citet{Rogers1994}:
% \begin{quote}
%     ``[With \ac{DCog},] the central unit of analysis is the functional system, which essentially is a collection of individuals and artefacts and their relations to each other in a particular work practice. \ldots~The main goal is to account for how the distributed structures, which make up the functional system, are coordinated by analyzing the various contributions of the environment in which the work activity takes place, the representational media (e.g. instruments, displays, manuals, navigation charts), the interactions of individuals with each other and their interactional use of artefacts.'' ---~\citet{Rogers1994}
% \end{quote}

% Further still, situated cognition~\citep{Brown1989} draws upon and extends the notion of cognition as distributed, by adopting a perspective that orients to the sequentiality of interaction:

% \begin{quote}
%     ``\ldots [C]ognition must be situated, interactive, and flexible. Cognition emerges from moment-by-moment interaction with the environment rather than proceeding in an autonomous, invariant, context-free fashion.'' ---~\citet{Smith2004}
% \end{quote}

% Both \ac{DCog} and Situated Cognition furnishes researchers with the matters of the use of artefacts and space, and how their use interplays with the humans in a setting, to establish the work of cognition.
% Studies in this nature are typically ethnographic in practice, with researchers recording the interaction in a setting and the artefacts used, and applying the analytic framework to document the distributed and coordinated form in which cognition is achieved within a setting.
% Since this early work in establishing the approach, multiple studies have drawn on these theoretical perspectives, with many safety-critical settings such as clinical and healthcare environments (e.g. \citet{Galliers2007,Blandford2006,Rajkomar2011}) and non-safety critical settings such as the classroom (e.g. \citet{Brown1993}),
% The adoption of these theories for use in such generative roles is now widespread and shares cross-disciplinary links with efforts to build contextually aware technology (see \ref{sec:background technology mobilehci context}).
% The ultimate expression of such approaches is the notion that a technology could be devised and developed that is contextually aware of how interaction unfolds with respect to the social and spatial aspects, in addition to the cognition of the device user.


% % *********************************************************************************************************************


% \subsection{Pragmatic studies of discourse}\label{sec:background emca discourse prag}
% Pragmatic studies, although similar to \ac{DCog} and Situated Cognition in terms of practical data collection and initial analysis, instead dispense with notions of modelling cognition and interaction, seeking to work with only what is empirically documentable.
% Pragmatic studies, which were developed alongside the approaches outlined in the previous section, are a combination of many differing approaches to studying discourse, and often use \acf{CA} in their analytic approach to documenting the practical matters of interaction.
% In this thesis, this approach is taken because there exists a gap in the literature of what is done by people to use devices in conversation --- the stance in this thesis does not preclude a cognitive approach\footnote{Merely, it is remarked that alternative approaches exist, and these too could be taken in order to reveal different matters of interaction, such as the cognitive models of members in the setting by employing \ac{DCog}. This is not the locus of this research as this thesis is preoccupied with addressing an established gap in the literature.}.

% Following the progression of anthropological studies into studies of interaction, the distinct study of \textit{semiotics} formed.
% This work, found primarily in the humanities, is pitched as the ``study of signs'' of everyday life.
% Such signs are not limited to literal signs or artefacts, but also include \textit{interactional signs}~\citep{Chandler2007}, such as body language and gaze~\citep{Kendon1967}, and what Goffman refers to as ``body gloss''~\citep[p. 11]{Goffman1971}.
% Such notions are found elsewhere too and were not confined to 'semiotic' studies, with the study of \textit{multimodal interaction} found within both \textit{Discourse Studies} (e.g. \citet{Mondada2007}) and that of \textit{Pragmatics} (e.g. \citet{Levinson1983}). %[p. 64]
% \acf{CA} augments this study of semiotics with the study of the structure and organisation of conversation as a whole.
% This work turns the name `conversation analysis' into a sort-of misnomer, as it works to unpack all facets of practical action in interaction, including conversation \textit{and} semiotics.

% Sociolinguistics, which builds upon linguistics work that studies the use and understanding of language syntax and structure, augments the pragmatic approach by considering societal factors such as gender and race.
% Sociolinguistics is concerned with how these factors are interconnected with the conveyance of meaning information, and how social identity is established in and through communication~\citep{Holmes2017}.
% The studies in this thesis, which draw upon \ac{CA}, are intentionally devoid of analysing the intersectionality of societal structures at play in interaction.
% The applicability of \ac{CA} to study such matters has been debated, although work has attended to issues of gender and feminism with the approach, e.g. \citet{Stokoe2001}.
% Nevertheless, sociolinguistics departs from the matters of documenting the observable action by introducing factors that are not empirical or quantifiable in small-scale studies such as this thesis adopts\footnote{In other words, it would be brazen to represent or make judgements of all human action using the studies in this thesis because they are only the study of a small number of non-representative people. \ac{CA} and pragmatics remains an applicable approach for this, because of the sole attention to empirical matters.}.
% %Nevertheless, this thesis will not study such matters.



% % *********************************************************************************************************************



% % \section{Ethnography}\label{sec:background emca ethnography}
% % There are many different approaches and perspectives under the name of ``ethnography'' --- this thesis follows on from the transitions developed at Xerox PARC\footnote{Xerox Palo Alto Research Center, now simply referrec to as PARC}, and later adopted in the fields of \ac{HCI} and \ac{CSCW} on the study of computing systems and workplaces for design, under the label of ``ethnomethodlogically-informed ethnography'' and the interactional `work' people do in-the-wild\footnote{It should be noted that although many researchers in both \ac{HCI} and \ac{CSCW} have adopted ethnomethodlogically informed ethnography as a collaborative agreed and validated practice for research, there is still much debate within the fields amongst researchers as to this perspective and its suitability to studying interaction beyond work environments.
% % This academic argument has generated numerous papers on this practice, and the use of `work' as a term to describe what-is-done (see \cite{Schmidt2010,Crabtree2009,Crabtree2006,Button2012}).}.
% % Insomuch, this thesis provides rich praxeological accounts of the practices of members in the setting under study as situated and cooperatively-achieved actions.
% % Herein, this section and the next will describe this development of the ethnographic perspective, how it provides researchers with a rich insight into the what-is-seen-and-done by members in setting, and how the underlying findings of this thesis have been constructed.

% % The tradition of ethnography lies in anthropological studies of non-native lands and communities through the collection of new data, popularised and promoted by the publication of \citet{Malinowski1922}\footnote{The prominent feature of this work is that Malinowski shifted the study of communities out of the armchair and into the field, and in doing so established ethnographic fieldwork as a prominent scientific method of the 20th century for documenting cultures and habits of others.}
% % This was done in much the say way as today, accounts were constructed of activities observed by the researcher and methodically documented to present as-true-as-possible representation of what was seen, so as to document what was done:



% % *********************************************************************************************************************



% \section{Ethnomethodologically informed ethnography}\label{sec:background emca em}

% There are many different approaches and perspectives under the name of ``ethnography'' --- this thesis follows on from the traditions developed at Xerox PARC, and later adopted in the fields of \ac{HCI} and \ac{CSCW}, on the study of computing systems and workplaces, under the label of ``ethnomethodologically informed ethnography'' and the interactional `work' people do in-the-wild\footnote{It should be noted that although many researchers in both \ac{HCI} and \ac{CSCW} have adopted ethnomethodologically informed ethnography as a practice for research, there is still much debate within the fields amongst researchers as to this perspective and suitability to studying interaction beyond work environments.
% This academic argument has generated numerous papers on this practice, and the use of `work' as a term to describe what-is-done (see \citet{Schmidt2010,Crabtree2009,Crabtree2006,Button2012}).}.
% Insomuch, this thesis provides rich praxeological accounts of the practices of members in the setting under study as situated and cooperatively-achieved actions.
% This was managed by participant-observation and capture of talk-in-action with audio-video recording equipment.
% Herein, this section and the next will describe the ethnomethodological perspective adopted, how it provides researchers with a rich insight into the what-is-seen-and-done by members of the setting, and how the underlying findings of this thesis have been constructed.
% Ethnomethodology inherently draws upon the pragmatics approach to the study of interaction, in which the occurrence of action is considered to both define and embody the rules contingently, where interaction is considered to be a collaborative achievement amongst interactants, and where the interaction establishes the social order of the setting.

% For brevity, the oft-discussed history of how ethnographic practices, participant-observation, and ethnomethodology were established have been omitted\footnote{\citet{Button2015} provide a detailed account of the establishment and continued evolution of the ethnomethodological tradition, and its use in \ac{CSCW} and systems design.}.



% % *********************************************************************************************************************



% \subsection{`Work' and `interactional what'}\label{sec:background emca em work}
% A study that is ethnomethodological in character focuses on the ongoing ordinary primordially social features of everyday interaction~\citep{Schegloff1987}.
% Interaction is treated as the principle and locus of study, with the accounts of what is done consisting of rich \textit{thick descriptions}~\citep{Geertz1973a} that unpack and reveal the interactional \textit{naturally accountable} methods of members (i.e. \textit{the accountable character} of work in the setting).
% Accountability is defined as an action that is observable and reportable, i.e. what it is done is observ-\textit{able} and tell-\textit{able} by the other parties who are present~\citep[pp 117--118]{Crabtree2012}.
% In other words, a thick description is about, as analysts, attending carefully ``to what is done in the doing of action and have us ‘thicken up’ the thinnest level of description to make its accountable character visible and available to others.''~\citep[pp 117--118]{Crabtree2012}.

% \textit{Work}, in the sense of ethnomethodology is not treated as equivalent to paid labour, but is considered the achievement of mundane naturally occurring activities.
% The notion of work is considered the culmination of `everyday stuff that is done', as established by Sacks:
% \begin{quote}
%     ``Whatever we may think about what it is to be an ordinary person in the world, an initial shift is not to think of an ‘ordinary person’ as some person, but as somebody having as their job \ldots~doing ‘being ordinary’.
%     It's not that somebody is ordinary \ldots~it takes work, as any other business.'' --- \citet{Sacks1992a}
% \end{quote}

% There have been many ethnographic studies of ``ordinary activities'', with the participant-observation approaches existing since the times of \citet{Malinowski1922}.
% Yet the ethnomethodological orientation to ethnography also embellishes qualitative participant-observation approaches with attendance to revealing the ``interactional what''~\citep{GarfinkelMissing}.
% This 'what' was developed in a commentary by David Sudnow and Garfinkel in response to studies of Jazz singers by Howard Becker.
% Becker provides accounts for ``the career structure of the jazz musician, the fraternal organisation of work it gave rise to, the pressures of work and playing to the audience, the dilemma of commercialism versus prestige, and the impact of family on the musician’s life and the conflict it generates'', yet did not reveal the circumstances in which music was collaboratively accomplished.
% These studies, although ethnographic, were found to merely provide ``scenic descriptions'' of what is done and were of limited use in understanding how interaction was specifically achieved as a situated action.
% The tradition of ethnomethodology is not only crucial to understanding and making available what is done as a gloss, but principally what is done \textit{in interaction} (i.e. the `interactional work' of Jazz musicians).
% As summarised succinctly by \citet{Button2015}:

% \begin{quote}
%     ``The interactional what of work is still missing in ethnographic studies more generally.
%     Not only in mainstream ethnographies of work, but also in symbolic interactionist studies and a great many ethnographic studies conducted for the purposes of systems design as well.
%     The latter may well produce findings of interest, but like the studies of the symbolic interactionists they nevertheless treat interaction at the scenic level.
%     The result is that an ethnographic study may at first glance appear to be taking on an examination of work itself in furnishing first-hand ‘insider’ accounts of interaction, but on closer inspection it transpires that the work is missing, supplanted by accounts of the interaction that surrounds work and what can be abstracted from it for the purposes of systems design.'' --- \citet{Button2015}
% \end{quote}

% It is this matter of `interactional what' that informs the empirical analysis within the first two studies in this thesis, and it is delivered through the production of thick descriptions of the naturally accountable work of members in the setting under observation.
% This thesis is not concerned with theories of `why' something happened, or indeed theories of interaction or work in general.
% In this sense, the thesis adopts ``ethnomethodological indifference''~\citep{Garfinkel1970}.
% As summarised by \citet[p. 141]{Lynch1993}, this consideration allows researchers to pragmatically study the work of people ``[r]ather than addressing whether sociologists ever can achieve adequate or acceptable accounts of the phenomena they study''.
% In other words, what matters in this research is explicating the \textit{members' methods} of interaction, and not those of the researcher, i.e. there need not be a theoretical unpinning of understanding in how people use mobile devices because this work is primarily concerned with the \textit{accountable} interaction of the stetting.
% This thesis will explicate the interactional methods of members as the commodity that establishes the findings of this work, and that with such an analytic lens, it is unnecessary to consider `why' a person `decided' to perform an action\footnote{Action selection is obviously more much complex than simply `choice', but that is for another thesis.}.

% %The third study, although it relies upon the analysis of talk-in-action without video capture, is still able to reveal the interactional work done by members to handle the interaction with the voice interface within the multi-party talk.
% %



% % *********************************************************************************************************************



% \subsection{Sequential organisation of situated action}\label{sec:background emca em sequentiality}
% \textit{Sequentiality} is defined as ``any kind of organization which concerns the relative positioning of utterances or actions [...] turn-taking [in conversation] is a type of sequential organization because it concerns the relative ordering of speakers''~\citep{Schegloff2007}.
% Importantly, sequentiality differs from mere temporal ordering (although it can take advantage of it), not only in that it encompasses actions that occur temporally in tandem (such as overlapped talk), but that the sequential coherence of conversation is a continuous achievement by conversationalists, who are seeking to assemble the retrospective-prospective sense of those actions which are often outside a basic temporal order.
% For instance, a speaker might answer a question several turns subsequent to it being posed in a conversation (which might be accounted for by a speaker in various ways, e.g. prefacing ``before I answer your question…'' to their turn).

% It is this feature of interaction as being sequentially organised, and further so locally and longitudinally managed by members, i.e. the arrangements of turn-taking and organisation of action are negotiated and established only in and through their production and the context of the interaction~\citep{Button1995a,Nguyen2008}, that provides the basis for Suchman's notion of `Situated Action'.
% Suchman's analysis draws upon observation of everyday interaction with a photocopier at Xerox PARC, and by drawing upon ethnomethodology, she was able to explicate not only issues with the design of the hardware but also fundamental notions of the mundane achievement of work in using the device. On situated action, she notes that:

% \begin{quote}
%     ``That term underscores the view that every course of action depends in essential ways on its material and social circumstances.
%     Rather than attempt to abstract action away from its circumstances and represent it as a rational plan, the approach is to study how people use their circumstances to achieve intelligent action.'' --- \citet[p. 35]{Suchman1985}.%   Rather than build a theory of action out of a theory of plans, the aim is to investigate how people produce and find evidence for plans in the course of situated action.
% \end{quote}

% Suchman's definition builds in the notion that people are `everyday sociologists', and that members of settings can observe what other members of the setting are doing, and so too members that other members can recognise what other members are doing in the setting, and that this stems from the natural accountability of action~\citep{Berger1966}.
% Furthermore, not only does Suchman's work provide the practical methodological approach for this thesis, but the premise of action as established in and through its achievement as a product of the context within which it is done, is imperative in bringing the empirical findings into the context of design in this thesis.
% In other words, Suchman's seminal work scopes out a field of work in which research in interaction with systems focuses on the practically and accountably done actions as opposed to theoretical assumptions of action, and so sensitises researchers to the need to include, not abstract, contextual shaping and shaped implications of action.



% % *********************************************************************************************************************



% \subsection{Naturalistic studies of action}\label{sec:background emca em naturalistic-studies}
% The work in this thesis dispenses with traditional ethnographic approaches of observing people ``in-the-wild'' --- in each study participants were recruited and asked to visit a setting.
% In this sense, there is a naturalistic element to the studies in that what was observed was done under the guise of a study.
% The question of what the practical implications of recruiting people for observation are, whether it is a valid approach, and whether such a study is `natural', `naturalistic', or neither, is important to consider\footnote{It may almost go without saying that video recording strangers and performing analysis on the data is both illegal and unethical under UK laws and regulations.}.
% The demarcation regarding the application of the label of `natural' and `naturalistic' is unnecessary insomuch the findings of the studies in this thesis are not used to draw conclusions on which activities occur `naturally', or used to provide moral judgements on actions.
% The studies are transparently open about the design, the data collection, the analysis (which is presented in this thesis), and the discussion that is drawn from this data.
% The focus in each study is on the actions of members, in this setting, with the understanding that each member was aware and had consented to data capture\footnote{This thesis makes no promise to deliver findings that represent or support arguments of ``human behaviour'' in the general sense.
% What is presented here is an account of what-was-seen-and-done in the observational studies, and in the case of the third study in \autoref{ch:empirical home}, what-was-said-and-done.
% The contributions from these studies make no claim to their generalisability.}.
% In each study, at no point was their intent to guide or constrain action of members --- what is presented as data did accountably happen in each setting.
% The notion of researcher influence is, as a result, purposeless, as the analysis focuses on practical accountable actions (i.e. this thesis considers \textit{what and how} something happened, and not \textit{why}).

% In this thesis, each study has been purposefully curated to take place in a setting that is natural for the device interaction at hand.
% In the first two studies, these are in the semi-public settings of a pub and a caf\'e\footnote{There has been some discussion through peer review of the publications upon which this thesis is based whether pubs and caf\'es are \textit{public}, \textit{private}, or \textit{semi-public} (note that the word `pub' itself is short for public house).
% Nevertheless, `semi-public' is used here to delineate between definitely public settings like plazas and private settings like the home.}.
% The studies examine the use of portable devices such as smartphones and tablets, and so these settings are appropriate and the use of devices is expected\footnote{Many pubs and caf\'es provide free Wi-Fi connectivity as a convenience to customers underscoring this expectation of portable device use.}.
% The third study in this thesis is of devices which only have a voice interface, and often require being plugged into mains electricity to function.
% Common sense dictates that such a device is not suited to public (or semi-public) settings, and so this study was conducted in the home where such devices are designed for.



% % *********************************************************************************************************************



% \section{Conversation analysis}\label{sec:background emca ca}
% \acf{CA}, which assimilates the ethnomethodological tradition (and so is routinely abbreviated as \ac{EMCA} when analyses draw upon both ethnomethodology and uses \ac{CA}) is an established analytic approach to unpacking both verbal and non-verbal interaction.
% \ac{CA} examines utterances and non-verbal interactional resources as `talk-in-action' as achieved through interlocutors contextually understanding and referencing utterances and other non-verbal actions by participants within the sequential organisation of the setting.
% \ac{CA} itself was established through many publications around the same time, although is most often attributed to Sacks et al.'s work on turn-taking organisation in talk~\citep{Sacks1974}.
% \citet{Atkinson1984} draws attention to how \ac{CA}, however, is not primarily interested in distinct utterances as singular events, but in fact considers the sequences of action in which talk progressively unfolds: the name \ac{CA} could perhaps be construed as a misnomer, as it attends to all matters of contextually accountable action, including the use of interactional resources and artefacts within an interaction.

% \begin{quote}
%     ``For conversation analysts, therefore, it is sequences and turns within sequences, rather than isolated sentences or utterances, that have become the primary units of analysis.
%     This focus on participant orientation to the turn-within-sequence character of utterances in conversational interaction has significant substantive and methodological consequences.
%     At the substantive level, conversation analytic research into sequence is based on the recognition that, in a variety of ways, the production of some current conversational action proposes a here-and-now definition of the situation to which subsequent talk will be oriented.'' --- \citet{Atkinson1984}
% \end{quote}

% Such a perspective correlates with the ethnomethodological tradition of ethnography (as outlined above in \autoref{sec:background emca em}).
% This analogous perspective is deliberate, given \acp{EM} attendance to naturally accountable (i.e. observable-reportable) actions in a setting.
% Conversation is formed of a multitude of naturally accountable actions\footnote{For example, talk, interactional resources such as eye gaze, and body co-orientation, and so on.} and constitutes an interactional achievement in the work of everyday life (i.e. people use talk to get things done in interaction).
% \ac{EMCA} does not employ theoretical reasoning to assess why conversationalists perform specific actions, but uses what-is-said-and-done to reveal what is accountably done in interaction, and how this is accountably attended to by others in and through the ongoing interaction.

% Much work in \ac{CA} seeks to document different methods through which communication and interaction unfold, to document the sequential organisation of conversation~\citep{Schegloff2007}, as demonstrating preferences for answers in multi-party settings~\citep{Stivers2006}, and how conversational floors are socially and collaboratively agreed features of talk~\citep{Edelsky1981}.
% \ac{EMCA} has also been applied in many different settings to understand and document interaction as situationally achieved and collaboratively organised, i.e. the findings of \ac{CA} are used methodically to reveal the interactional work of a setting.
% For example, research has identified how artefacts, such as paper hospital records, are by mobilised and manipulated as an asynchronous communications tool between different hospital staff~\citep{Luff1998a}.
% In other settings, equally safety-critical settings, the approach is used to represent and document conversation in cockpits, revealing difficulties that unfold in communication that lead to operational errors unfolding~\citep{Nevile2005} and in air traffic control~\citep{Bentley1992}.
% This last tranche is where this thesis is situated: the work here does not seek to augment \ac{CA} literature with new findings of conversational organisation but instead employs such findings to document the work of using technology in a face-to-face social encounter.

% The work in the empirical chapters draws upon the `Jeffersonian' orthographic transcription notation~\citep{Atkinson1984a}, as summarised in \appref{app:notation}.
% This notation should be considered a \textit{representation} of the data, as the analysis included in this thesis was itself performed upon the `raw' data as it was collected.
% The first two studies in this thesis captured video, as well as audio, and so in addition to the given transcripts of spoken (and minimal spatial action), stills from videos are included at key moments to provide greater insight into the activities of members.

% The research questions of this thesis (see \autoref{sec:intro rqs}) frame this thesis as unpacking the interactional and collaborative achievement of embedded device use in conversation.
% By drawing upon \ac{EMCA} as its analytic framework, as outlined above, this thesis will make use of thick descriptions of action, and explicate the sequential character of how members embed the use of devices in and through interaction to discuss the collaborative work of using devices in conversation.



% *********************************************************************************************************************



% Thus far, this thesis has concerned itself with the notions of interaction in terms of how interaction and how people, as social characters, perform through mundane routine to accomplish multiple tasks, often in and through ``messiness of every day practice''~\citep{Bell2007}.
% Existing \ac{HCI}literature has been shown to be replete with examples of co-operative and collaborative work conducted with mobile and portable electronic devices, used in scenarios that require individuals to draw upon verbal and visual communication with others in addition to relying upon technology controlled by one or more members, to complete one or more tasks.

% \citet{Bell2007}, through reflection upon the ubiquitous computing vision proposed by Weiser~\citep{Weiser1991}, demonstrate the case for understanding ``how social and cultural practice is carried out in and around emerging information technologies'', given that the use of technology is routinely embedded in routine.
% Their call to study ubiquitous computing as a matter of the present and as a matter of the social world, rather than an abstract futuristic notion, is an underlying assumption of the work conducted in chapters XX.
% The use of technology is practiced -- and embedded -- in routine, using small displays, drawing upon multiple invisible-at-the-point-of-use systems, and as people are mobile~\citep{Crabtree2006}.

% A key narrative of this work has to been to consider the use of technology in everyday `casual-social' settings such as the home, or the pub; where individuals gather to socialise, and routinely, as previously discussed, embed device interactions within a face-to-face encounter.
% To reflect upon how technology becomes embedded in and through everyday interaction within social settings, this thesis will orient to consider the frame of the modality of the device interaction as a remarkable characteristic that contributes to the formation of co-operative encounters in collocated settings.



% % *********************************************************************************************************************



% \section{Multimodal Interaction}



% *********************************************************************************************************************
